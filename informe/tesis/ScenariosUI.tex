\section{Interfaz Gr�fica: Scenarios UI}
	\label{sec:scenariosUI}

	\subsection{Motivaciones para una ``GUI''}
		A�n considerando las mejoras provistas por la extensi�n a Rainbow propuesta en el presente trabajo, uno de los
		escollos m�s notorios para poder utilizar de manera amena, �gil y productiva a ``Arco Iris'' es, sin dudas, la
		ausencia de una interfaz visual para que los \emph{stakeholders} y arquitectos de la aplicaci�n a adaptar puedan
		crear, editar y eliminar escenarios, entornos, artifacts y otros conceptos introducidos en ``Arco Iris''; as� tambi�n
		como otros conceptos ya existentes en ``Rainbow''.
		
		% TODO CONTINUAR! A fin de supEn el presente apartado nos concentraremos en repasar 
	
	
		\todo{Chamuyo de que nos llev� a hacer esto, incluir pq es en ingles}
		
		Se propone el desarrollo de una interfaz de usuario gr�fica (GUI, de sus siglas en ingl�s: Graphical User Interface)
		para que los distintos \emph{stakeholders}, incluyendo usuarios y arquitectos, puedan colaborar creando y editando
		escenarios que luego ser�n importados y utilizados por Rainbow.

	\subsection{Conceptos b�sicos de uso de la herramienta}
		\todo{Explicar que siempre hay por debajo un SHC, que se va actualizando siempre, que la estructura es
		siempre ver todo actualizado en las consultas, etc\ldots}
		
		\subsubsection{Constraints}
			\todo{Mencionar que es extensible el composite pero que por ahora solo implementamos la numeric relational binary
			constraint}

	\subsection{Administraci�n de Artifacts}

	\subsection{Administraci�n de Entornos}
	
		\subsubsection{Pesos relativos de Concerns}
		
		\subsubsection{El entorno ``ANY''}

	\subsection{Administraci�n de Escenarios}

		\subsubsection{Selecci�n de Entornos}
		
		\subsubsection{Selecci�n de Estrategias de Reparaci�n}

	\subsection{Puntos de extensi�n}