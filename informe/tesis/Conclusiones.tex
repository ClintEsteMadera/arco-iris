\section{Conclusiones}
	\todo{Alcances de los resultados obtenidos, ventajas y desventajas, futuros trabajos, etc.}
	\todo{Disertar sobre los roles que una metodolog�a como la presentada puede jugar en el desarrollo de sistemas y
sobre
los beneficios que supone su uso.}

	\subsection{Res�men (copy paste de la propuesta, REVISAR!)}
		En resumidas palabras, este trabajo a�ade las siguientes caracter�sticas a Rainbow, todas ellas tendientes a
disponer de una herramienta gen�rica de auto reparaci�n m�s poderosa y al mismo tiempo, a involucrar a los actores
funcionales del sistema, puesto que este tipo de actores son los que saben (junto obviamente a arquitectos,
dise�adores, etc.) como se debe comportar el sistema ante determinadas situaciones de operaci�n (normal o no):
		\begin{itemize}
			\item Definir las siguientes extensiones:
				\begin{itemize}
						\item Posibilidad de modelar escenarios de atributos de calidad siguiendo los principios de ATAM y QAW.
						\item Posibilidad de relacionar escenarios con componentes de la arquitectura.
						\item Posibilidad de especificar prioridades relativas entre escenarios, a ser utilizadas en la elecci�n
de la estrategia de reparaci�n a ejecutar.
						\item Posibilidad de modelar estrategias de reparaci�n y asociarlas a escenarios.
						\item Posibilidad de definir Entornos de ejecuci�n para el sistema, los cuales tengan un papel clave en el
algoritmo de elecci�n de estrategias de reparaci�n.
				\end{itemize}
			\item Proponer los cambios en el \emph{Manejador de Reparaciones} y el \emph{Evaluador de Restricciones} de
Rainbow, e implementar algunos casos pr�cticos que permitan mostrar c�mo esta estrategia puede funcionar y llevar a un
framework de adaptaci�n m�s flexible y poderoso.
			\item Evaluar la factibilidad de que esta nueva versi�n extendida de Rainbow sea compatible con la original,
es decir que idealmente cualquier usuario que ya posea Rainbow funcionando en su sistema podr�a incorporar estas
extensiones sin tener que modificar la configuraci�n actual.
		\end{itemize}

		Todo lo antedicho tiene como consecuencia que los usuarios tendr�n m�s herramientas para influir sobre lo que el
sistema decida hacer para auto repararse: s�lo con modificar la informaci�n de los escenarios el sistema modificar� su
comportamiento. Como contrapartida, las extensiones propuestas en este trabajo hacen que el sistema se comporte de
manera m�s aut�noma a medida que se agregan escenarios; esto parad�jicamente le quita control al usuario, ya que el
procedimiento de auto reparaci�n se vuelve m�s complejo, dificultando el seguimiento de las decisiones tomadas por el
\emph{framework}.