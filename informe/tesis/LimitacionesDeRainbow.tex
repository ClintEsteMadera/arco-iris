\section{Limitaciones de Rainbow}
	
	\begin{sidewaystable}
		\begin{center}
%				\footnotesize{ AGREGAR ESTO SI SE AGREGA CONTENIDO Y LA TABLA NO CABE EN UNA PAGINA
				\rowcolors*[\hline]{1}{GreenYellow!25}{GreenYellow!10}
				\begin{tabularx}{\textwidth}{|X|X|X|X|}
					\textbf{Problema} & \textbf{Implementaci�n en Rainbow} & \textbf{Implementaci�n en Thesis} & \textbf{Ventajas}\\
					Informaci�n sobre restricciones & Las restricciones son guardadas en el modelo de la arquitectura (expresado en \mbox{ACME}) & Los \mbox{\emph{stakeholders}} usan una interfaz visual para expresar restricciones del sistema en formato de escenarios de QAW. Esa informaci�n es guardada en un archivo XML y luego tomada por Rainbow para tomar decisiones sobre las auto-reparaciones a realizar. & Nuevas res\-tri\-ccio\-nes pue\-den ser a\-gre\-ga\-das din�micamente y de una manera simplificada para los \emph{stakeholders}.\\
					Relaci�n entre \emph{concerns} arquitecturales(e.g. tiempo de respuesta) & El concepto de atributo de calidad no tiene un modelado espec�fico en Rainbow. S� est� presente en ZNN, y son llamados llamados ``Quality Dimensions''. Los mismos se relacionan est�ticamente en una funci�n de utilidad (d�nde el valor de los atributos suman 1) & Los atributos de calidad son parte de los escenarios y se relacionan utilizando la terminolog�a de ATAM. Un escenario puede tener mayor prioridad que otro y esto afecta a los \emph{tradeoffs} que nuestra extensi�n a Rainbow deber� hacer al momento de escoger una estrategia de reparaci�n. & Los \emph{stakeholders} tienen la posibilidad de cambiar esta informaci�n din�micamente y consecuentemente, afectar la manera en que el framework opera.\\
					Decisiones sobre que reparaciones realizar & Las reparaciones a realizar tambi�n se encuentran en archivos de configuraci�n & El sistema eval�a los escenarios cargados y como se afectan entre ellos. Luego invoca a un m�dulo que decide qu� reparaciones pueden llevarse a cabo de acuerdo a las restricciones. & Dicho m�dulo puede ser extendido para inclu�r complejas heur�sticas que aprendan de los efectos de reparaciones pasadas y puedan as� mejorar los resultados.\\
					Entorno de ejecuci�n (e.g. Alta Carga)& El entorno de ejecuci�n es est�ticamente configurado en un archivo de configuraci�n y no se modifica de acuerdo al estado din�mico del sistema que se intenta reparar. Para modificar tal est�tico y limitado valor, es necesario el reinicio de Rainbow. & El modelo de nuestra extensi�n a Rainbow permite al usuario especificar distintos entornos de ejecuci�n, los cuales tienen restricciones asociadas que se chequean continuamente y que, de cumplirse, hacen que el sistema autom�ticamente se considere que est� en otro entorno de ejecuci�n, afectando al algoritmo de decisi�n de estrategias de reparaci�n. & Lo antedicho, agrega una nueva arista a las variables consideradas al momento de elegir la mejor estrategia de reparaci�n, aumentando las probabilidades de que la estrategia seleccionada sea m�s precisa de acuerdo a la situaci�n del sistema en \emph{runtime}.\\
				\end{tabularx}
%				}
		\end{center}
	\end{sidewaystable}