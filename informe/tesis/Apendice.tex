\appendix

\section{Arquitectura de Znn modelada con Acme}
	\label{sec:arquitecturaZNN}

	\begin{Verbatim}
		import TargetEnvType.acme;
	
		Family ZNewsFam extends EnvType with {
	
			Port Type HttpPortT extends ArchPortT with {
	
			}
			Role Type RequestorRoleT extends ArchRoleT with {

			}
			Component Type ProxyT extends ArchElementT with {

				Property deploymentLocation : string <<  default : string = "localhost"; >> ;

				Property load : float <<  default : float = 0.0; >> ;

			}
			Port Type ProxyForwardPortT extends ArchPortT with {

			}
			Component Type ServerT extends ArchElementT with {

				Property deploymentLocation : string <<  default : string = "localhost"; >> ;

				Property load : float <<  default : float = 0.0; >> ;

				Property reqServiceRate : float <<  default : float = 0.0; >> ;

				Property byteServiceRate : float <<  default : float = 0.0; >> ;

				Property fidelity : int <<  HIGH : int = 5; LOW : int = 1; default : int = 5; >> ;

				Property cost : float <<  default : float = 1.0; >> ;

				Property lastPageHit : Record [uri : string; cnt : int; kbytes : float; ];

				Property anotherConstraint : string
						<<  default : string = "heuristic self.load <= MAX_UTIL;"; >> ;
			}
			Role Type ReceiverRoleT extends ArchRoleT with {

			}
			Connector Type ProxyConnT extends ArchConnT with {
				Role req : RequestorRoleT = new RequestorRoleT extended with {

				}
				Role rec : ReceiverRoleT = new ReceiverRoleT extended with {

				}
			}
			Component Type ClientT extends ArchElementT with {

				Property deploymentLocation : string <<  default : string = "localhost"; >> ;

				Property experRespTime : float <<  default : float = 100.0; >> ;

				Property reqRate : float <<  default : float = 0.0; >> ;
				rule primaryConstraint = invariant self.experRespTime <= MAX_RESPTIME;
				rule reverseConstraint = heuristic self.experRespTime >= MIN_RESPTIME;

			}
			Port Type HttpReqPortT extends ArchPortT with {

			}
			Connector Type HttpConnT extends ArchConnT with {

				Property bandwidth : float <<  default : float = 0.0; >> ;

				Property latency : float <<  default : float = 0.0; >> ;

				Property numReqsSuccess : int <<  default : int = 0; >> ;

				Property numReqsRedirect : int <<  default : int = 0; >> ;

				Property numReqsClientError : int <<  default : int = 0; >> ;

				Property numReqsServerError : int <<  default : int = 0; >> ;

				Property latencyRate : float;
				Role req : RequestorRoleT = new RequestorRoleT extended with {

				}
				Role rec : ReceiverRoleT = new ReceiverRoleT extended with {

				}
			}

			Property MIN_RESPTIME : float;

			Property MAX_RESPTIME : float;

			Property TOLERABLE_PERCENT_UNHAPPY : float;

			Property UNHAPPY_GRADIENT_1 : float;

			Property UNHAPPY_GRADIENT_2 : float;

			Property UNHAPPY_GRADIENT_3 : float;

			Property FRACTION_GRADIENT_1 : float;

			Property FRACTION_GRADIENT_2 : float;

			Property FRACTION_GRADIENT_3 : float;

			Property MIN_UTIL : float;

			Property MAX_UTIL : float;

			Property MAX_FIDELITY_LEVEL : int;

			Property THRESHOLD_FIDELITY : int;

			Property THRESHOLD_COST : float;

			Property SUPPORT_FRACTION_GRADIENT : boolean;
		}
	
		System ZNewsSys : ZNewsFam = {
	
			Property MIN_RESPTIME : float = 100.0;
	
			Property MAX_RESPTIME : float = 1000.0;
	
			Property UNHAPPY_GRADIENT_1 : float = 0.1;
	
			Property UNHAPPY_GRADIENT_2 : float = 0.2;
	
			Property UNHAPPY_GRADIENT_3 : float = 0.5;
	
			Property FRACTION_GRADIENT_1 : float = 0.2;
	
			Property FRACTION_GRADIENT_2 : float = 0.4;
	
			Property FRACTION_GRADIENT_3 : float = 1.0;
	
			Property TOLERABLE_PERCENT_UNHAPPY : float = 0.4;
	
			Property MIN_UTIL : float = 0.1;
	
			Property MAX_UTIL : float = 0.75;
	
			Property MAX_FIDELITY_LEVEL : int = 5;
	
			Property THRESHOLD_FIDELITY : int = 2;
	
			Property THRESHOLD_COST : float = 4.0;
	
			Property SUPPORT_FRACTION_GRADIENT : boolean = false;

			Component s1 : ServerT, ArchElementT = new ServerT, ArchElementT extended with {
	
				Property deploymentLocation = "phoenix";

				Property isArchEnabled = true;

				Property cost = 1.0;

				Property fidelity = 3;

				Property load = 0.891;
				Port http0 : HttpPortT, ArchPortT = new HttpPortT, ArchPortT extended with {

				}
	
			}

			Component lbproxy : ProxyT = new ProxyT extended with {
	
				Property deploymentLocation = "127.0.0.1";

				Property isArchEnabled = true;

				Property load = 0.01;
				
				Port fwd0 : ProxyForwardPortT = new ProxyForwardPortT extended with {

						Property isArchEnabled = true;
				}
				Port fwd1 : ProxyForwardPortT = new ProxyForwardPortT extended with {

				}
				Port fwd2 : ProxyForwardPortT = new ProxyForwardPortT extended with {

				}
				Port fwd3 : ProxyForwardPortT = new ProxyForwardPortT extended with {

				}
				Port http0 : HttpPortT = new HttpPortT extended with {

						Property isArchEnabled = true;
				}
				Port http1 : HttpPortT = new HttpPortT extended with {

						Property isArchEnabled = true;
				}
				Port http2 : HttpPortT = new HttpPortT extended with {

						Property isArchEnabled = true;
				}	
			}

			Component s2 : ServerT, ArchElementT = new ServerT, ArchElementT extended with {
	
				Property deploymentLocation = "127.0.0.3";

				Property isArchEnabled = true;

				Property fidelity = 5;

				Property load = 0.99;

				Property cost = 1.0;
				
				Port http0 : HttpPortT, ArchPortT = new HttpPortT, ArchPortT extended with {

						Property isArchEnabled = false;
				}
	
			}

			Component s3 : ServerT, ArchElementT = new ServerT, ArchElementT extended with {

				Property deploymentLocation = "127.0.0.4";

				Property isArchEnabled = true;

				Property cost = 1.0;

				Property fidelity = 3;

				Property load = 0.891;
				
				Port http0 : HttpPortT, ArchPortT = new HttpPortT, ArchPortT extended with {

				}	
			}

			Component s0 : ServerT, ArchElementT = new ServerT, ArchElementT extended with {
	
				Property deploymentLocation = "oracle";

				Property cost = 0.9;

				Property fidelity = 3;

				Property load = 0.594;

				Property isArchEnabled = true;

				Port http0 : HttpPortT = new HttpPortT extended with {

						Property isArchEnabled = true;
				}
			}
			Component c1 : ClientT = new ClientT extended with {
	
				Property deploymentLocation = "127.0.0.1";

				Property isArchEnabled = true;

				Property experRespTime = 433.36273;
				
				Port p0 : HttpReqPortT = new HttpReqPortT extended with {

						Property isArchEnabled = true;
				}
			}

			Component c2 : ClientT = new ClientT extended with {

				Property deploymentLocation = "127.0.0.1";

				Property isArchEnabled = true;

				Property experRespTime = 344.6827;
				
				Port p0 : HttpReqPortT = new HttpReqPortT extended with {

						Property isArchEnabled = true;
				}	
			}

			Component c0 : ClientT = new ClientT extended with {
	
				Property deploymentLocation = "127.0.0.1";

				Property isArchEnabled = true;

				Property experRespTime = 414.76843;

				Port p0 : HttpReqPortT = new HttpReqPortT extended with {

						Property isArchEnabled = true;
				}
			}

			Connector conn0 : HttpConnT = new HttpConnT extended with {
	
				Property latencyRate = 0.0;

				Property isArchEnabled = true;

				Role req  = {

						Property isArchEnabled = true;
				}
				Role rec  = {

						Property isArchEnabled = true;
				}
			}

			Connector proxyconn0 : ProxyConnT = new ProxyConnT extended with {

				Property isArchEnabled = true;

				Role req  = {

						Property isArchEnabled = true;

				}

				Role rec  = {

						Property isArchEnabled = true;
				}
			}

			Connector proxyconn1 : ProxyConnT, ArchConnT = new ProxyConnT, ArchConnT extended with {
	
			}
			Connector proxyconn3 : ProxyConnT, ArchConnT = new ProxyConnT, ArchConnT extended with {
	
			}

			Connector proxyconn2 : ProxyConnT, ArchConnT = new ProxyConnT, ArchConnT extended with {
	
			}

			Connector conn : HttpConnT = new HttpConnT extended with {
	
				Property latencyRate = 0.0;

				Property isArchEnabled = true;

				Role req  = {

						Property isArchEnabled = true;	
				}
				Role rec  = {

						Property isArchEnabled = true;
				}
			}

			Connector conn1 : HttpConnT = new HttpConnT extended with {
	
				Property latencyRate = 0.0;

				Property isArchEnabled = true;

				Role req  = {

						Property isArchEnabled = true;
				}
				Role rec  = {

						Property isArchEnabled = true;
				}
			}

			Attachment lbproxy.fwd0 to proxyconn0.req;
			Attachment s0.http0 to proxyconn0.rec;
			Attachment lbproxy.http0 to conn0.rec;
			Attachment c1.p0 to conn.req;
			Attachment c2.p0 to conn1.req;
			Attachment lbproxy.http2 to conn1.rec;
			Attachment lbproxy.http1 to conn.rec;
			Attachment c0.p0 to conn0.req;
			Attachment s2.http0 to proxyconn2.rec;
			Attachment lbproxy.fwd1 to proxyconn1.req;
			Attachment s1.http0 to proxyconn1.rec;
			Attachment s3.http0 to proxyconn3.rec;
			Attachment lbproxy.fwd3 to proxyconn3.req;
			Attachment lbproxy.fwd2 to proxyconn2.req;
		}
	\end{Verbatim}

\newpage

\section{T�cticas de Znn}
	\label{sec:tacticasZNN}

	\begin{Verbatim}
		module newssite.tactics;
	
		import model "ZNewsSys.acme" { ZNewsSys as M, ZNewsFam as T };
		import op "znews0.operator.EffectOp" { EffectOp as S };
		import op "org.sa.rainbow.stitch.lib.*";
	
	
		/**
		* Enlist n free servers into service pool.
		* Utility: [v] R; [^] C; [<>] F
		*/
		tactic enlistServers (int n) {
			condition {
				// some client should be experiencing high response time
				exists c : T.ClientT in M.components | c.experRespTime > M.MAX_RESPTIME;
				// there should be enough available server resources
				Model.availableServices(T.ServerT) >= n;
			}
			action {
				set servers = Set.randomSubset(Model.findServices(T.ServerT), n);
				for (T.ServerT freeSvr : servers) {
					S.activateServer(freeSvr);
				}
			}
			effect {
				// response time decreasing below threshold should result
				forall c : T.ClientT in M.components | c.experRespTime <= M.MAX_RESPTIME;
			}
		}
	
		/**
		* Deactivate n servers from service pool into free pool.
		* Utility: [^] R; [v] C; [<>] F
		*/
		tactic dischargeServers (int n) {
			condition {
				// there should be NO client with high response time
				forall c : T.ClientT in M.components | c.experRespTime <= M.MAX_RESPTIME;
				// there should be enough servers to discharge
				Set.size({ select s : T.ServerT in M.components | s.load < M.MIN_UTIL }) >= n;
			}
			action {
				set lowUtilSvrs = { select s : T.ServerT in M.components | s.load < M.MIN_UTIL };
				set subLowUtilSvrs = Set.randomSubset(lowUtilSvrs, n);
				for (T.ServerT s : subLowUtilSvrs) {
					S.deactivateServer(s);
				}
			}
			effect {
				// still NO client with high response time
				forall c : T.ClientT in M.components | c.experRespTime <= M.MAX_RESPTIME;
			}
		}
	
		/**
		* Lowers fidelity by integral steps for percent of requests.
		* Utility: [v] R; [v] C; [v] F
		*/
		tactic lowerFidelity (int step, float fracReq) {
			condition {
				// some client should be experiencing high response time
				exists c : T.ClientT in M.components | c.experRespTime > M.MAX_RESPTIME;
				// exists server with fidelity to lower
				exists s : T.ServerT in M.components | s.fidelity > step;
			}
			action {
				// retrieve set of servers who still have enough fidelity grade to lower
				set servers = { select s : T.ServerT in M.components | s.fidelity > step };
				for (T.ServerT s : servers) {
					S.setFidelity(s, s.fidelity - step);
				}
			}
			effect {
				// response time decreasing below threshold should result
				forall c : T.ClientT in M.components | c.experRespTime <= M.MAX_RESPTIME;
			}
		}
	
		/**
		* Raises fidelity by integral steps for percent of requests.
		* Utility: [^] R; [^] C; [^] F
		*/
		tactic raiseFidelity (int step, float fracReq) {
			condition {
				// there should be NO client with high response time
				forall c : T.ClientT in M.components | c.experRespTime <= M.MAX_RESPTIME;
				// there exists some client with below low-threshold response time
				exists c : T.ClientT in M.components | c.experRespTime < M.MIN_RESPTIME;
			}
			action {
				// first find the lowest fidelity set
				set servers = { select s :
						T.ServerT in M.components | s.fidelity <= M.MAX_FIDELITY_LEVEL - step};
			for (T.ServerT s : servers) {
					S.setFidelity(s, java.lang.Math.min(s.fidelity + step, M.MAX_FIDELITY_LEVEL));
				}
			}
			effect {
				// still NO client with high response time
				forall c : T.ClientT in M.components | c.experRespTime <= M.MAX_RESPTIME;
			}
		}
	\end{Verbatim}

\newpage

	\section{Representaci�n en XML de una configuraci�n de \emph{Self Healing}}
	\label{sec:SelfHealingConfigXML}
			\begin{Verbatim}[gobble=4]
				<selfHealingConfiguration description="test">
				  <artifacts>
				    <artifact id="0" name="ClientT" systemName="ZNewsSys"/>
				    <artifact id="1" name="ServerT" systemName="ZNewsSys"/>
				  </artifacts>
				  <environments>
				    <environment id="0" name="NORMAL">
				      <conditions>
				        <numericBinaryRelationalConstraint>
				          <artifact reference="../../../../../artifacts/artifact"/>
				          <property>experRespTime</property>
				          <quantifier>IN_AVERAGE</quantifier>
				          <binaryOperator>LESS_THAN</binaryOperator>
				          <constantToCompareThePropertyWith class="int">800</constantToCompareThePropertyWith>
				        </numericBinaryRelationalConstraint>
				      </conditions>
				      <weights class="tree-map">
				        <no-comparator/>
				        <entry>
				          <concern>RESPONSE_TIME</concern>
				          <double>0.333</double>
				        </entry>
				        <entry>
				          <concern>SERVER_COST</concern>
				          <double>0.333</double>
				        </entry>
				        <entry>
				          <concern>CONTENT_FIDELITY</concern>
				          <double>0.333</double>
				        </entry>
				      </weights>
				    </environment>
				    <environment id="1" name="HIGH LOAD">
				      <conditions>
				        <numericBinaryRelationalConstraint>
				          <artifact reference="../../../../../artifacts/artifact"/>
				          <property>experRespTime</property>
				          <quantifier>IN_AVERAGE</quantifier>
				          <binaryOperator>GREATER_THAN</binaryOperator>
				          <constantToCompareThePropertyWith class="int">800</constantToCompareThePropertyWith>
				        </numericBinaryRelationalConstraint>
				      </conditions>
				      <weights class="tree-map">
				        <no-comparator/>
				        <entry>
				          <concern>RESPONSE_TIME</concern>
				          <double>0.7</double>
				        </entry>
				        <entry>
				          <concern>SERVER_COST</concern>
				          <double>0.2</double>
				        </entry>
				        <entry>
				          <concern>CONTENT_FIDELITY</concern>
				          <double>0.1</double>
				        </entry>
				      </weights>
				    </environment>
				  </environments>
				  <scenarios>
				    <selfHealingScenario id="0" name="Client Experienced Response Time Scenario"
				    	enabled="true" priority="1">
				      <concern>RESPONSE_TIME</concern>
				      <stimulus name="GetNewsContentClientStimulus" source="Any Client requesting news content"
				      	any="false"/>
				      <environments>
				        <anyEnvironment></anyEnvironment>
				      </environments>
				      <artifact reference="../../../artifacts/artifact"/>
				      <response>Requested News Content</response>
				      <responseMeasure>
				        <description>Experienced response time is within threshold</description>
				        <constraint class="numericBinaryRelationalConstraint">
				          <artifact reference="../../../../../artifacts/artifact"/>
				          <property>experRespTime</property>
				          <quantifier>IN_AVERAGE</quantifier>
				          <binaryOperator>LESS_THAN</binaryOperator>
				          <constantToCompareThePropertyWith class="int">500</constantToCompareThePropertyWith>
				        </constraint>
				      </responseMeasure>
				      <repairStrategies class="specificRepairStrategies">
				        <repairStrategy>VariedReduceResponseTime</repairStrategy>
				      </repairStrategies>
				    </selfHealingScenario>
				    <selfHealingScenario id="1" name="Server Cost Scenario" enabled="true" priority="2">
				      <concern>SERVER_COST</concern>
				      <stimulus any="true"/>
				      <environments>
				        <anyEnvironment reference="../../../selfHealingScenario/environments/anyEnvironment"/>
				      </environments>
				      <artifact reference="../../../artifacts/artifact[2]"/>
				      <response>The proper response for the request</response>
				      <responseMeasure>
				        <description>Active servers amount is within threshold</description>
				        <constraint class="numericBinaryRelationalConstraint">
				          <artifact reference="../../../../../artifacts/artifact[2]"/>
				          <property>cost</property>
				          <quantifier>SUM</quantifier>
				          <binaryOperator>LESS_THAN</binaryOperator>
				          <constantToCompareThePropertyWith class="int">4</constantToCompareThePropertyWith>
				        </constraint>
				      </responseMeasure>
				      <repairStrategies class="specificRepairStrategies">
				        <repairStrategy>ReduceOverallCost</repairStrategy>
				      </repairStrategies>
				    </selfHealingScenario>
				  </scenarios>
				</selfHealingConfiguration>
			\end{Verbatim}

\section{Representaci�n en XML del Escenario de Tiempo de Res\-pues\-ta}
	\label{sec:scenarioExpRespTimeXML}

	\begin{Verbatim}
		<selfHealingScenario id="0" name="Client Experienced Response Time Scenario" enabled="true"
			priority="1">
			<concern>RESPONSE_TIME</concern>
			<stimulusSource>Any Client requesting news content</stimulusSource>
			<stimulus>GetNewsContentClientStimulus</stimulus>
			<environments>
				<defaultEnvironment></defaultEnvironment>
			</environments>
			<artifact reference="../../../artifacts/artifact"/>
			<response>Requested News Content</response>
			<responseMeasure>
				<description>Experienced response time is within threshold</description>
				<constraint class="numericBinaryRelationalConstraint" sum="false">
					<artifact reference="../../../../../artifacts/artifact"/>
					<property>experRespTime</property>
					<binaryOperator>LESS_THAN</binaryOperator>
					<constantToCompareThePropertyWith class="int">600
						</constantToCompareThePropertyWith>
				</constraint>
			</responseMeasure>
			<architecturalDecisions/>
			<repairStrategy></repairStrategy>
		</selfHealingScenario>
	\end{Verbatim}

\newpage

\section{Representaci�n en XML del Escenario de Costo}
	\label{sec:scenarioCostXML}

	\begin{Verbatim}
		<selfHealingScenario id="1" name="Server Cost Scenario" enabled="true" priority="2">
		    <concern>SERVER_COST</concern>
		    <stimulusSource>Anyone</stimulusSource>
		    <stimulus>ANY</stimulus>
		    <environments>
		        <defaultEnvironment reference="../../../selfHealingScenario/environments/defaultEnvironment"/>
		    </environments>
		    <artifact reference="../../../artifacts/artifact[2]"/>
		        <response>The proper response for the request</response>
		        <responseMeasure>
		            <description>Active servers amount is within threshold</description>
		                <constraint class="numericBinaryRelationalConstraint" sum="true">
		                    <artifact reference="../../../../../artifacts/artifact[2]"/>
		                    <property>cost</property>
		                    <binaryOperator>LESS_THAN_OR_EQUALS</binaryOperator>
		                     <constantToCompareThePropertyWith class="int">1</constantToCompareThePropertyWith>
		                </constraint>
		        </responseMeasure>
		        <architecturalDecisions/>
		        <repairStrategy></repairStrategy>
		</selfHealingScenario>
	\end{Verbatim}

\newpage

\section{Implementaci�n de Probe y extensi�n de Arco Iris}
	\label{sec:probesCode}

	\begin{Verbatim}[gobble=4]
				public class ClientProxyProbe {
					
					...
					
					public void run() {
						byte[] bytes = new byte[Util.MAX_BYTES];
						URL url = new URL(urlStr);
						HttpURLConnection httpConn = (HttpURLConnection) url.openConnection();
						ByteArrayOutputStream baos = new ByteArrayOutputStream();
						int cnter = 0;
						long startTime = System.currentTimeMillis();
						int length = httpConn.getContentLength();
						BufferedInputStream in = new BufferedInputStream(httpConn.getInputStream());
						while (in.available() > 0 || cnter < length) {
							int cnt = in.read(bytes);
							baos.write(bytes, 0, cnt);
							cnter += cnt;
						}
						long endTime = System.currentTimeMillis();
						String rpt = "[" + Util.probeLogTimestamp() + "]<" + id() + "> " + 
									url.getHost() + ":" + (endTime - startTime) + "ms";
						reportData(rpt);
						try {
							Thread.sleep(sleepTime());
						} catch (InterruptedException e) {
							// intentional ignore
						}
					}
			
					...
				}

	\end{Verbatim}

\newpage

	\begin{Verbatim}[gobble=4]
				public class ClientProxyProbeWithStimulus {

					...
					
					public void run() {
						byte[] bytes = new byte[Util.MAX_BYTES];
						URL url = new URL(urlStr);
						HttpURLConnection httpConn = (HttpURLConnection) url.openConnection();
						ByteArrayOutputStream baos = new ByteArrayOutputStream();
						int cnter = 0;
						long startTime = System.currentTimeMillis();
						int length = httpConn.getContentLength();
						BufferedInputStream in = new BufferedInputStream(httpConn.getInputStream());
						while (in.available() > 0 || cnter < length) {
							int cnt = in.read(bytes);
							baos.write(bytes, 0, cnt);
							cnter += cnt;
						}
						long endTime = System.currentTimeMillis();
						String rpt = "[" + Util.probeLogTimestamp() + "]<" + id() + "> " + 
								url.getHost() + "<stimulus:" + stimulusName + ">:" 
								+ (endTime - startTime) + "ms";
						reportData(rpt);
						httpConn.disconnect();
						try {
							Thread.sleep(sleepTime());
						} catch (InterruptedException e) {
							// intentional ignore
						}
					}

					...

	\end{Verbatim}

\newpage


\section{Implementaci�n de NumericBinaryRelationalConstraint}
	\label{sec:numericBinaryRelationalConstraintCode}

	\begin{Verbatim}[gobble=4]
				@XStreamAlias("numericBinaryRelationalConstraint")
				public class NumericBinaryRelationalConstraint 
						extends BaseSinglePropertyInvolvedConstraint {
				
					private NumericBinaryOperator binaryOperator;
				
					private Number constantToCompareThePropertyWith;
				
					@XStreamOmitField
					private String exponentialPropertyName;
				
					public NumericBinaryRelationalConstraint(Quantifier quantifier, Artifact artifact, 
							String property,NumericBinaryOperator binaryOperator, 
							Number constantToCompareThePropertyWith) {
						super(artifact, property);
						this.quantifier = quantifier;
						this.binaryOperator = binaryOperator;
						this.constantToCompareThePropertyWith = constantToCompareThePropertyWith;
					}			
					
					public boolean holds(Number expValue) {
						boolean holds = this.binaryOperator.performOperation(expValue, 
								this.constantToCompareThePropertyWith);
						return holds;
					}
				
					...
				}	
	
	\end{Verbatim}
	
\section{Instrucciones para ejecutar ``Scenarios UI''}
	\todo{Completar!!!}