\documentclass[11pt, a4paper, spanish]{article}

%%%%%%%%%%%%%%%%%%%%%%%%%%%%%%%%%%% COMIENZO DEL PREAMBULO %%%%%%%%%%%%%%%%%%%%%%%%%%%%%%%%%%%%%%%%

%Info sobre este documento
\author{Jonathan Chiocchio, Gabriel Tursi}
\title{Tesis de Licenciatura Chiocchio - Tursi}

\usepackage{caratula}                     % incluye caratula est�ndar
\usepackage{tabularx}                     % permite usar tablas mas inteligentes que con ``tabular''
\usepackage{lscape}                       % pone un parte del texto de forma apaisada
\usepackage{slashbox}                     % separaci�n en diagonal en las celdas de las esquinas de tablas
\usepackage{rotating}                     % rotar cualquier objeto en cualquier �ngulo
%\usepackage{multirow}
%\usepackage{ltxtable}
\usepackage[ansinew]{inputenc}            % permite que los acentos del estilo ����� salgan joya
\usepackage[spanish, activeacute]{babel}  % idioma espa�ol, acentos f�ciles y deletreo de palabras
\usepackage{indentfirst}                  % permite indentar un parrafo a mano
\usepackage{graphicx}                     % permite insertar gr�ficos
\usepackage{color}                        % permite el uso de colores en el documento
\definecolor{gray97}{gray}{.97}
\usepackage[dvipsnames,table,svgnames]{xcolor}
\usepackage{url}                          % permite el uso de urls
\usepackage[top=2.75cm, bottom=2.75cm, left=2.75cm, right=2.75cm]{geometry} % m�rgenes
\usepackage[pdfcreator={LaTeX2e},
			pdfauthor={Jonathan Chiocchio, Gabriel Tursi},
			pdftitle={Hacia un modelo m'as flexible para la implementaci'on de la auto reparaci'on de sistemas de software basada en Arquitectura},
			pdfsubject={Hacia un modelo m'as flexible para la implementaci'on de la auto reparaci'on de sistemas de software basada en Arquitectura},
			pdfkeywords={arquitecture design, self-healing, atam, rainbow, ACME},
			pdfstartview=FitH,            % Fits the width of the page to the window
			bookmarksnumbered,            % los bookmarks numerados se ven mejor...
			colorlinks,                   % links con bellos colores
			linkcolor=blue]               % permite cambiar el color de los links
			{hyperref}                    % Permite jugar con algunas cosas que aparecer�n en el PDF final
\usepackage{listings}
\lstset{
%	language=XML,                         % choose the language of the code
 	basicstyle=\scriptsize\ttfamily,      % the size of the fonts that are used for the code
	keywordstyle=\scriptsize\ttfamily,    % NOTE that \ttfamily does NOT have a BOLD version     
	commentstyle=\scriptsize\ttfamily,
	stringstyle=\scriptsize\ttfamily,
	identifierstyle=\scriptsize\ttfamily,
%	numbers=none,                         % where to put the line-numbers
%	numberstyle=\scriptsize,              % the size of the fonts that are used for the line-numbers
%	stepnumber=1,                         % the step between two line-numbers. If it is 1 each line will be numbered
%	numbersep=0pt,                        % how far the line-numbers are from the code
	extendedchars=true,                   % allow extended characters
 	backgroundcolor=\color{lightgray!20}, % specify the colour of the background. You must add \usepackage{color}
	rulecolor=\color{lightgray!20},       % specify the colour of the rules
%	fillcolor=\color{white},              % specify the colour of the space between 'text box' and first rule
%	rulesepcolor=\color{gray97},          % specify the colour of the space between two rules (useful for shadows)
 	showspaces=false,                     % show spaces adding particular underscores
 	showstringspaces=false,               % underline spaces within strings
 	showtabs=false,                       % show tabs within strings adding particular underscores
 	tabsize=2,                            % sets default tabsize to 2 spaces
%	linewidth=\linewidth,                 % define the base line width for listings
 	frame=single,                         % adds a frame around the code
%	frameround=tttt,                      % round corners
%	framerule=0.4pt,                      % controls the width of the rules
 	framesep=0pt,                         % control the separation btw the frame both interior and exterior margins
%	rulesep=2pt,                          % control the space between frame and listing and between double rules
	xleftmargin=0pt,                      % extra left margin
%	xrightmargin=0pt,                     % extra right margin
%	framextopmargin=0pt,                  % defines 'how far' we move the top frame margin
%	framexbottommargin=0pt,               % defines 'how far' we move the bottom frame margin
%	framexleftmargin=0pt,                 % defines 'how far' we move the left frame margin to the left
%	framexrightmargin=0pt,                % defines 'how far' we move the right margin to the right
%	captionpos=b,                         % sets the caption-position to bottom
 	breaklines=true,                      % sets automatic line breaking
	breakautoindent=true,                 % activates or deactivates automatic indention of broken lines
%	breakatwhitespace=false,              % sets if automatic breaks should only happen at whitespace
%	morecomment=[l]{//}                   % displays comments in italics (language dependent)
%	escapeinside={\%}{)}                  % if you want to add a comment within your code
}

\selectlanguage{spanish}

\linespread{1.3}                          % interlineado equivalente al 1.5 l�neas de Word...
\headsep = 30pt                           % separaci�n entre encabezado y comienzo del p�rrafo
\setlength{\parskip}{1ex plus 0.5ex minus 0.2ex} % aumenta la separaci�n entre p�rrafos

%\setlength{\parindent}{0pt}              % eliminar el indentado en todo el documento
%\pagestyle{myheadings}                   % encabezado personalizable con \markboth{}{}
% \markboth{}{Encabezado custom}          % permite especificar un encabezado personalizado

%\addtolength{\oddsidemargin}{-2cm}       % configuracion IDEAL!!!
%\addtolength{\textwidth}{4cm}
%\addtolength{\textheight}{2cm}

\def\todo#1{\textcolor{red}{#1}}          % macro 'todo' para To-Do's

%%%%%%%%%%%%%%%%%%%%%%%%%%%%%%%%%%%%%%%% FIN DEL PREAMBULO %%%%%%%%%%%%%%%%%%%%%%%%%%%%%%%%%%%%%%%%

\begin{document}

	\materia{Tesis de Licenciatura en\\[0.3em]Ciencias de la Computaci�n}
	\titulo{Hacia un modelo m�s flexible para la implementaci�n de la auto reparaci�n de sistemas de software basada en Arquitectura}
	\integrante{Chiocchio, Jonathan}{849/02}{jchiocchio@gmail.com}
	\integrante{Tursi, Germ�n Gabriel}{699/02}{gabrieltursi@gmail.com}
	\director{Santiago Ceria}
	
	\maketitle
	
	\newpage
	
	\begin{abstract}
  Los sistemas auto reparables (o aut�nomos) son aquellos que pueden adaptarse din�micamente a las condiciones del
entorno, ayudando as� a asegurar su propia estabilidad y productividad sin depender de la intervenci�n de un usuario
administrador. Existen al momento diversos enfoques en materia de auto reparaci�n de sistemas, sin embargo, en ninguno
se consider� al usuario como un participante crucial al momento de determinar los requerimientos de auto reparaci�n de
un sistema. El objetivo principal del presente trabajo es el de proveer una herramienta que permita superar
esa limitaci�n. Se introduce una extensi�n a \emph{Rainbow}, un \emph{framework} de auto reparaci�n basado
en arquitecturas ya existente. Dicha extensi�n, a la cual llamamos ``Arco Iris'', hace foco en el concepto de
``escenario de atributo de calidad'' como medio para que el usuario junto a otros \emph{stakeholders} del sistema puedan
configurar la auto reparaci�n del sistema de manera colaborativa; estableciendo, entre otras cosas, prioridades entre
distintos escenarios. En este trabajo se establece el marco te�rico necesario para poder estudiar el tema, se explica en
detalle el funcionamiento (de car�cter heur�stico) de ``Arco Iris'', se presenta tambi�n una herramienta visual que
permite al usuario definir los escenarios de manera sencilla e intuitiva, se estudia el funcionamiento de la extensi�n
con algunos casos de uso representativos, se extraen conclusiones sobre los resultados obtenidos y finalmente se
establecen puntos de continuaci�n para el presente trabajo.
\end{abstract}

\def\abstractname{Abstract}
\begin{abstract}

	\todo{SANTI: Traducir Resumen a Ingl�s}

\end{abstract}

	
	\newpage
	
	\tableofcontents
	
	\newpage
	
	\pagestyle{headings}                  % poniendo esto ac� evitamos que se encabece la caratula, el abstract y el
	
	\section{Introducci�n}
	\todo{Breve referencia del trabajo mencionando la parte novedosa del trabajo.}
	
	\newpage
	
	\section{Conceptos Preliminares}

	\subsection{No todo es funcionalidad\ldots}

		\subsubsection{Restricciones}

			En el desarrollo de sistemas cr�ticos, no es suficiente el satisfacer s�lamente \textbf{requerimientos funcionales}
			(i.e. ``aquello que el software deber hacer''). Este tipo de sistemas generalmente deben satisfacer otro tipo de
			\textbf{requerimientos no funcionales}, tambi�n llamados \textbf{Restricciones}, los cuales especifican criterios a
			ser usados para juzgar la operaci�n de un sistema en lugar de describir funcionalidad espec�fica para el mismo.
			
			Algunos ejemplos de restricciones sobre un sistema de software son los siguientes:
	
			\begin{enumerate}
				\item el c�digo de la aplicaci�n debe ser desarrollado en Java.
				\item la base de datos debe ser SQL Server.
				\item s�lo se utilizar�n productos de c�digo abierto (\emph{open-source}).
				\item el sistema debe ser escalable con respecto a la cantidad de usuarios que lo utilizan concurrentemente.
				\item el sistema debe implementar pol�ticas de tolerancia a fallos.
				\item el sistema debe ser dise�ado de manera tal que se minimize el procesamiento y el tiempo de respuesta.
			\end{enumerate}
	
			Observamos que 1, 2 y 3 establecen meras restricciones a tener en cuenta en el an�lisis, dise�o o desarrollo de la
			aplicaci�n; mientras que 4, 5 y 6 se refieren a caracter�sticas espec�ficas de \textbf{calidad} deseadas para el
			sistema. Estos �ltimos se denominan \textbf{Atributos de Calidad} y es imprescindible conocerlos para poder dise�ar
			la arquitectura de un sistema. Notemos que, en el ejemplo anterior, las restricciones correspondientes a atributos de
			calidad (4, 5 y 6), est�n especificados de manera vaga e imprecisa. Esto es lo que suele ocurrir en la mayor�a de
			los casos en la industria de desarrollo de \emph{software}. Notar tambi�n que los requerimientos 4 y 6 pueden
			llegar a afectarse mutuamente: esto tambi�n es muy com�n y para lograr un buen \emph{tradeoff} entre atributos
			de calidad los arquitectos suelen tener que tomar un conjunto de decisiones arquitect�nicas llamadas
			\textbf{estrategias}, sobre las cuales profundizaremos en la secci�n \ref{sec:tacticasEstrategias}.

		\subsubsection{Atributos de Calidad y \emph{Concerns}}
		
			El est�ndar 1061-1998 de la IEEE \footnote{Para m�s informaci�n, visitar
			\url{http://standards.ieee.org/findstds/standard/1061-1998.html}} que establece una metodolog�a para la definici�n de
			m�tricas con respecto a la calidad del \emph{software}, reza:
			
			\begin{quote}
				La calidad del \emph{software} es el nivel que posee de una combinaci�n deseada de atributos (e.g. confiabilidad,
				interoperabilidad, \emph{performance}, etc.)
			\end{quote}
			
			Algunos ejemplos de atributos de calidad definidos en los est�ndares IEEE 1061 / ISO 9126
			\footnote{Para m�s informaci�n, visitar \url{http://es.wikipedia.org/wiki/ISO/IEC_9126}} son:
			\begin{itemize}
				\item Eficiencia
				\item Funcionalidad
				\item Mantenibilidad
				\item Portabilidad
				\item Confiabilidad
				\item Usabilidad
			\end{itemize}
			
			Imaginemos que un \emph{sponsor} de un sistema a ser desarrollado establece que el sistema debe ser ``eficiente''.
			�Qu� significa esto exactamente? La pregunta es dif�cil de contestar si no se dispone de m�s informaci�n.
			Evidentemente, los atributos de calidad son categorizaciones de alto nivel que, si no se dispone de m�s
			informaci�n, no parecen servir de mucho para tomar decisiones arquitect�nicas en pos de alcanzar un
			nivel aceptable de dichos atributos de calidad. Al rescate de tal carencia aparecen las denominadas
			\textbf{Incumbencias} o, del ingl�s y tales como las llamaremos a lo largo de este trabajo, \textbf{Concerns}.
			
			Los \textbf{Concerns} son par�metros mediante los cuales los atributos de un sistema son juzgados, especificados y
			medidos. Usualmente, los requerimientos de atributos de calidad son expresados en t�rminos de \emph{concerns}.
			\cite{QA}.
			
			A continuaci�n, enumeraremos algunos ejemplos de concerns, junto con el atributo de calidad al que pertenecen:
			
			\begin{center}
				\rowcolors*[\hline]{1}{GreenYellow!25}{GreenYellow!10}
				\begin{tabular}{|c|c|}
				\textbf{Atributo de Calidad} & \textbf{\emph{Concerns}}\\
				Eficiencia & Comportamiento Temporal, Utilizaci�n de Recursos\\
				Funcionalidad & Interoperabilidad, Seguridad\\
				Mantenibilidad & Cambiabilidad, Facilidad de Prueba\\
				Portabilidad & Adaptabilidad, Coexistencia\\
				Usabilidad & Compresibilidad, Atractivo			
				\end{tabular}
			\end{center}

			Los \emph{concerns} pueden usualmente relacionarse con propiedades de la arquitectura de un sistema, por ejemplo,
			en una arquitectura cliente-servidor, el atributo de calidad \emph{performance} posee varios \emph{concerns}
			asociados, por ejemplo, el \textbf{tiempo de respuesta}, el cu�l est� relacionado directamente con algunas
			propiedades de la arquitectura como el ancho de banda de los servidores, la carga del sistema, la cantidad de
			servidores, etc.
			
			Tanto los atributos de calidad como los \emph{concerns} son conceptos fundamentales en el presente trabajo, no solo
			desde un punto de vista te�rico sino que tambi�n ser�n utilizados en la pr�ctica.
			
	\subsection{Sistemas Aut�nomos, Auto Adaptaci�n, Auto Reparaci�n\ldots}

 		A medida que va pasando el tiempo, los sistemas de software se vuelven cada vez m�s complejos y m�s exigentes en
		cuanto a disponibilidad se trata. Hoy en d�a los mismos operan en ambientes din�micos, con requerimientos de usuario
		altamente cambiantes y con la necesidad de operar pr�cticamente sin interrupci�n, resultando esto en un aumento
		en la administraci�n operativa del software, lo cual representa un costo importante para que el sistema pueda
		mantenerse operativo. Para reducir este costo, se puede plantear que los sistemas se adapten de manera din�mica para
		poder utilizar los recursos existentes, a fin de poder atender los cambiantes requerimientos de atributos de
		calidad, as� tambi�n como los errores en el sistema. De forma gen�rica, a los sistemas de software que cumplen con
		estas caracter�sticas, se los denomina \textbf{Sistemas Aut�nomos}.

		Hilando m�s fino en la caracterizaci�n de sistemas aut�nomos, encontramos t�rminos en ingl�s como \emph{Self
		Configuring} o \emph{Self Adapting} para referirse a sistemas aut�nomos que tienen la capacidad de auto configurarse
		(o auto adaptarse) a condiciones cambiantes en el entorno de ejecuci�n del sistema.

		Por otro lado, cuando la adaptaci�n din�mica del sistema responde a errores o situaciones excepcionales del
		mismo, el t�rmino m�s utilizado actualmente es \emph{Self Healing} o, en castellano, \textbf{Auto Reparaci�n}.

		Si bien ya existen mecanismos para mitigar los mencionados problemas, ellos normalmente est�n intr�nsecamente ligados
		al lenguaje de programaci�n utilizado para constru�r la aplicaci�n, tales como tratamiento de excepciones, protocolos de
		tolerancia a fallos, etc. Adem�s, estos mecanismos generalmente dependen del c�digo de la aplicaci�n que se intenta
		adaptar y consecuentemente, no son f�cilmente reutilizables entre distintos sistemas. En res�men: hoy en d�a, la
		adaptaci�n de sistemas de software es costosa de construir, dificil de modificar, poco reusable y generalmente solo
		provee soluciones a fallos de manera puntual.

		En cuanto al estado del arte en materia de sistemas aut�nomos existen diversos enfoques tanto en el �mbito acad�mico
		como en la industria del software.

		Dentro del �mbito de la industria, el concepto de sistemas aut�nomos se encuentra ampliamente difundido. Sin dudas, el
		enfoque de IBM, denominado ``Autonomic Computing'' \cite{IBM-AC} es el m�s completo, apalancado por un gran grupo de
		investigaci�n y abarcando el problema desde distintos aspectos. Tambi�n se destaca la iniciativa de Microsoft
		denominada ``Dynamic Systems Initiative'', y no tanto la iniciativa de Sun (``Predictive Self-Healing'' y
		``Conscientious Software''), por estar m�s ligada a adaptar el enfoque al dominio de los sistemas operativos. Para m�s
		informaci�n sobre estos enfoques, remitirse a \cite{Casuscelli}

		Por otro lado, existen distintos investigadores ligados a prestigiosas instituciones acad�micas abocados al estudio de
		distintos aspectos de la autonom�a de los sistemas. Uno de los primeros investigadores en acu�ar el t�rmino ``Self
		Healing'' fue el Dr. David Garlan, de la Universidad de Carnegie Mellon, quien form� un grupo de investigaci�n que
		dedic� a�os a estudiar el tema dentro del marco del proyecto ABLE.\footnote{El proyecto ABLE (\emph{``Architecture
		Based Languages and Environments''}) de la Universidad de Carnegie Mellon lleva a cabo investigaciones que conducen a
		una base de ingenier�a para la arquitectura de software. Para m�s informaci�n, visitar
		\url{http://www.cs.cmu.edu/~able}}

		El presente trabajo toma como base el trabajo generado por el proyecto ABLE de Carnegie Mellon, el cual implementa
		el concepto de ``Auto Reparaci�n de Sistemas Basada en Modelos de Arquitectura'', el cual se describir� en detalle a
		continuaci�n.

	\subsubsection{Auto Reparaci�n de Sistemas Basada en Modelos de Arquitectura}
	\label{sec:ARSBMA}

		En contraste con los mecanismos tradicionales para detecci�n y recuperaci�n de errores que se implementan como parte
		del c�digo espec�fico de la aplicaci�n, con mecanismos localizados y poco escalables entre distintos sistemas; el
		enfoque propuesto por el Dr. David Garlan, usa \textbf{el modelo de la arquitectura} del sistema que se desea adaptar
		como instrumento para razonar sobre sus propiedades (e.g. tiempo de respuesta de un servidor) y sus correlatos
		con la din�mica del sistema en tiempo de ejecuci�n.

		Diversos investigadores han propuesto usar modelos arquitecturales \cite{ArchBasedApproach} que representan al
		sistema como una mera composici�n de componentes, sus interconexiones (conectores) y sus propiedades de
		inter�s. Este modelo es conocido comunmente como \textbf{C\&C} (componentes y conectores)\cite{C&C}. Tal propuesta
		ofrece diversos beneficios, el m�s significativo: un modelo arquitectural abstracto provee una perspectiva global del
		sistema y expone sus propiedades y restricciones de integridad.

		La idea propuesta consiste b�sicamente en un bucle cerrado (\emph{closed-loop} en ingl�s), donde existen dos capas
		(externas al sistema que est� siendo ejecutado) que act�an, una encargada del monitoreo del sistema y la otra
		proveyendo un mecanismo de control y adaptaci�n. Esto ofrece una soluci�n mas efectiva que cualquier mecanismo interno
		porque permite agrupar todo lo concerniente a la detecci�n y soluci�n del problema en m�dulos separados, pudiendo ser
		analizados, modificados, extendidos y reutilizados a trav�s de distintos sistemas.

		\begin{figure}[h]
			\centering
				\includegraphics{images/selfhealing-closed-loop.png}
			\caption{Bucle cerrado}
			\label{fig:selfhealing-closed-loop}
		\end{figure}

		Otro sub proyecto del proyecto ABLE, denominado \textbf{Rainbow} (sobre el cual profundizaremos m�s adelante) utiliza
		esta t�cnica de \emph{closed-loop} para monitorear y reparar sistemas.

	\subsection{Lenguajes de Descripci�n de Arquitecturas}

		Un problema fundamental en el dise�o de arquitecturas de sistemas utilizando el estilo de componentes y
		conectores ha sido encontrar la notaci�n apropiada para definir dichas arquitecturas.

		Un buen lenguaje para describir arquitecturas deber�a permitir generar una documentaci�n clara sobre los componentes
		de la arquitectura, que luego servir� como base a los de\-sa\-rro\-lla\-do\-res, permitiendo a su vez razonar sobre
		las propiedades del sistema y automatizar su an�lisis, hasta pudiendo quiz�s llegar a utilizarse para la generaci�n
		autom�tica de parte del c�digo que implementar� la arquitectura. Tambi�n deber�a ser efectivo para poder validar de
		manera temprana decisiones arquitect�nicas, reduciendo as� el tiempo de implementaci�n y evitando utilizar
		ineficientemente recursos en el desarrollo del sistema.

		Una forma de describir dichas arquitecturas es mediante el modelado de objetos mediante UML, si bien este m�todo ha
		sido ampliamente aceptado y utilizado en la industria, tiene varios inconvenientes, el m�s importante e
		invalidante es que no provee un soporte directo para describir propiedades no funcionales, esto hace dificultoso
		razonar sobre propiedades cr�ticas del sistema, como por ejemplo la \emph{performance} o la confiabilidad. �sta es la
		raz�n principal que ha motivado el avance de los ADLs (\emph{Architecture Description Languages}). Para m�s
		informaci�n sobre la discusi�n ADL's vs. UML, remitirse a \cite{ADLsVsUML}.

		La descripci�n de arquitecturas de sistemas basada en ADLs ha avanzado considerablemente en las �ltimas dos d�cadas,
		al punto de que ya permiten definir una base formal para su descripci�n y an�lisis.

	\subsubsection{Acme}
	\label{sec:acme}

		Acme\cite{ACME} es uno de los ADLs m�s reconocidos y utilizados, ha sido desarrollado en la universidad de Carnegie
		Mellon, m�s precisamente por el proyecto ABLE, liderado por el Dr. David Garlan.

		Acme es un pilar fundamental dentro del proyecto ABLE, ya que todo el proyecto gira en torno a la arquitectura de
		software de los sistemas, y es Acme quien permite describir formalmente dichas arquitecturas, por lo tanto todos los
		restantes sub proyectos utilizan Acme en menor o mayor medida.

		Adem�s de los beneficios de todo ADL, el lenguaje Acme y su kit de herramientas \emph{AcmeLib (Acme Tool Developer's
		Library)} proveen las siguientes capacidades fundamentales:

		\begin{itemize}
			\item Intercambio Arquitectural: al proveer un formato de intercambio gen�rico para dise�ar arquitecturas de
			software, Acme permite a los desarrolladores de herramientas de este tipo \footnote{Ejemplos de herramientas de
			descripci�n de arquitecturas y modelado UML podr�an ser: Enterprise Architect (\url{http://www.sparxsystems.com.au/})
			o Poseid�n (\url{http://www.gentleware.com/}), entre tantas otras.} integrar f�cilmente con otras herramientas
			complementarias. De esta manera, los arquitectos que usan herramientas basadas en Acme tienen un espectro m�s amplio
			de herramientas de	an�lisis y dise�o que quienes dise�an sus arquitecturas usando otros ADLs.
			\item Extensibilidad: Acme provee una base s�lida, gen�rica y extensible, y una infraestructura que evita que los
			desarrolladores vuelvan a construir herramientas de base. M�s a�n, debido a su idea originaria de lenguaje de
			intercambio gen�rico, Acme permite que las herramientas que se han desarrollado utiliz�ndolo sean compatibles con una
			gran variedad ADLs existentes y con herramientas con un m�nimo esfuerzo, y hasta en algunos casos sin esfuerzo alguno.
		\end{itemize}

		Actualmente, el lenguaje Acme y \emph{Acme Tool Developer's Library (AcmeLib)}, proveen una infraestructura gen�rica y
		extensible para describir, representar, analizar y generar descripciones de arquitecturas de software.

		A continuaci�n, observamos un breve ejemplo de una arquitectura modelada en el lenguaje ACME, la cual posee un
		sistema que contiene:
		\begin{itemize}
			\item un servidor HTTP, con algunas propiedades como por ejemplo fidelidad del contenido que provee.
			\item un cliente HTTP, tambi�n con algunas propiedades particulares como el tiempo de respuesta experimentado por el
			usuario.
		\end{itemize}

		\begin{Verbatim}[gobble=3]
			System system : ClientServerType = {
			    Component server : ServerT = new ServerT extended with {
			        Port http0 : HttpPortT;
			        Property cost;
			        Property fidelity;
			        Property load;
			    }
			    Component client : ClientT = new ClientT extended with {
			        Port p0 : HttpReqPortT = new HttpReqPortT extended with {
			            Property isArchEnabled = true;
			        }
			        Property deploymentLocation = "127.0.0.1";
			        Property isArchEnabled = true;
			        Property experRespTime;
			    }
			}
		\end{Verbatim}

		M�s adelante, en la secci�n \ref{sec:casosPracticos}, veremos otro ejemplo (m�s extenso) del lenguaje al mostrar la
		descripci�n completa en Acme de la arquitectura del sistema que utilizaremos para mostrar los resultados de la extensi�n
		implementada en el presente trabajo.

		\subsubsection{T�cticas y Estrategias}
		\label{sec:tacticasEstrategias}

			Hemos mencionado anteriormente que el objetivo de la auto reparaci�n es el alcanzar determinados atributos de
			calidad definidos para un determinado sistema, ajustando su comportamiento, de ser necesario, de acuerdo a
			las condiciones de ejecuci�n del mismo. En el libro \textbf{Software Architecture in Practice} \cite{BassClementz},
			Bass, Clements y Kazman caracterizan y formalizan dos herramientas que vienen siendo ampliamente utilizadas desde
			hace tiempo por los arquitectos de software en la industria, estas son: las \textbf{t�cticas} y las
			\textbf{estrategias}.

			Las \textbf{t�cticas} se definen como decisiones de dise�o tendientes a controlar las respuestas del sistema a
			determinados est�mulos, a fin de satisfacer uno o m�s atributos de calidad requeridos. La figura
			\ref{fig:tactics_control_to_response} muestra gr�ficamente el concepto de t�ctica arquitectural.

			\begin{figure}[h]
				\centering
					\includegraphics{images/tactics_control_to_response.png}
				\caption{Visi�n gr�fica del concepto de t�ctica}
				\label{fig:tactics_control_to_response}
			\end{figure}

			Cada t�ctica es una opci�n de dise�o para el arquitecto, un ejemplo concreto podr�a ser el introducir redundancia en
			determinados componentes de la arquitectura (e.g. base de datos, servidores web replicados, etc.) para incrementar la
			dis\-po\-ni\-bi\-li\-dad del sistema.

			Por otro lado, una \textbf{estrategia} puede ser entendida como un procedimiento delineado por los arquitectos de
			\emph{software} para intentar llevar al sistema a un nivel d�nde los atributos de calidad se cumplan en el nivel
			deseado; haciendo uso de una o m�s t�cticas. Cada t�ctica es ejecutada �nicamente cuando el estado del sistema
			satisface las condiciones impuestas por la estrategia para dicha ejecuci�n. Por ejemplo, en una arquitectura
			cliente-servidor y al verse el tiempo de respuesta comprometido, una estrategia podr�a intentar agregar servidores
			mientras existan disponibles o, hasta que el tiempo de respuesta haya descendido por debajo de un determinado umbral.
			Esta l�gica ser�a descrita en la estrategia, mientras que ser� la t�ctica \verb@levantar-servidor@ la responsable de
			ejecutar la acci�n propiamente dicha. Notar que si bien esta estrategia est� dise�ada para mejorar el tiempo de
			respuesta, tambi�n afecta negativamente al \emph{concern} ``cantidad de servidores'' (correspondiente al atributo de
			calidad ``costo'') puesto que usualmente el utilizar mayor cantidad de servidores suele tener un costo econ�mico no
			despreciable.
			
			Cuando una estrategia se dise�a para mejorar un atributo de calidad en particular, se puede decir que los
			\emph{stakeholders} que definen dichos atributos de calidad obtienen cierto ``beneficio'' de las estrategias. Cada
			estrategia provee un nivel espec�fico de dicho beneficio, pero en contrapartida presenta un costo en tiempo y, sobre
			todo, en dinero. Es por este motivo que los \emph{stakeholders} deben participar en el proceso de decisi�n de c�ales
			estrategias se emplear�n para satisfacer los atributos de calidad definidos para el sistema. Ellos deber�n evaluar el
			retorno de la inversi�n (la relaci�n costo-beneficio) de aplicar cada estrategia para elegir la m�s conveniente.

			La estrategia es la herramienta propuesta por Rainbow para quitar al sistema de un estado no deseado.

	\subsection{Rainbow}

		\subsubsection{Introducci�n a Rainbow}

			La herramienta Rainbow, tambi�n dentro del marco del proyecto ABLE, tiene como finalidad permitir reducir el costo e
			incrementar la confiabilidad al realizar cambios en sistemas complejos de software, para esto Rainbow automatiza la
			adaptaci�n de sistemas de software a traves de sus modelos de arquitectura, tal cual fue descrito en la secci�n
			\ref{sec:ARSBMA}.

			Si bien en principio el enfoque de auto adaptaci�n basado en arquitecturas es atractivo, el mismo supone un n�mero
			significativo de desaf�os en el campo de la investigaci�n as� tambi�n como en el de la ingenier�a:

			\begin{itemize}
				\item En primer lugar, uno de los aspectos claves que Rainbow intenta cubrir es la habilidad de manejar una amplia
				variedad de sistemas con arquitecturas, propiedades de inter�s y mecanismos que soporten modificaciones din�micas
				completamente diferentes.

				\item Por otro lado, Rainbow intenta ser una soluci�n que permita reducir el costo de agregar control externo al
				sistema a reparar, puesto que crear los me\-ca\-nis\-mos de monitoreo y detecci�n de
				problemas desde cero para un sistema nuevo ser�a prohibitivamente costoso. El enfocar la auto reparaci�n de un
				sistema en su arquitectura permite disponer de una infraestructura reutilizable junto con mecanismos para adaptar
				dicha infraestructura a las necesidades espec�ficas de cada sistema.
			\end{itemize}

			Cabe mencionar que el caracter externo y no intrusivo de Rainbow representa una ventaja tambi�n cuando se
			desea implementar auto reparaci�n en sistemas cuyo c�digo fuente no est� disponible o no es plausible de ser
			modificado.

			\subsubsection{Arquitectura de Rainbow}

				La Figura \ref{fig:Rainbow_Framework} muestra la arquitectura de Rainbow. En resumidas palabras, el \emph{framework}
				utiliza un modelo arquitectural abstracto para monitorear las propiedades en \emph{runtime} del sistema que est�
				siendo ejecutado, eval�a el modelo para detectar violaciones a restricciones impuestas sobre el mismo y lleva a cabo
				adaptaciones en el sistema tendientes a eliminar tales violaciones.

				\begin{figure}[H]
					\centering
						\includegraphics{images/Rainbow_Framework.png}
					\caption{Arquitectura de Rainbow}
					\label{fig:Rainbow_Framework}
				\end{figure}

				La infraestructura de adaptaci�n de Rainbow se divide en capas que proveen funcionalidades comunes a distintos
				sistemas auto adaptables logrando por lo tanto el objetivo de disponer de componentes reutilizables, a saber:
				\begin{enumerate}
					\item \textbf{Capa de Sistema}:\\
					En esta capa se define e implementa una interfaz de acceso al sistema que est� siendo ejecutado. Se define un
					mecanismo para medir variables de inter�s, materializado en \emph{Probes}: componentes que observan
					y miden diversos estados del sistema, para luego publicarlos.

					Adicionalmente, existe un mecanismo para descubrir recursos que puede ser utilizado especificando el tipo de
					recurso, entre otros criterios. Finalmente, los denominados \emph{Effectors} llevan a cabo las modificaciones
					propiamente dichas sobre el sistema.
					\item \textbf{Capa de Arquitectura}:\\
					En esta capa, los denominados \emph{Gauges} agregan informaci�n provenientes de los \emph{Probes} y
					mantienen constantemente actualizadas las propiedades correspondientes en el modelo arquitectural del sistema
					(descrito en ACME), el cual es manejado y accedido mediante un componente denominado \emph{Model Manager}. El
					\emph{Constraint Evaluator} chequea el modelo peri�dicamente y dispara la adaptaci�n en el caso que ocurra una
					violaci�n en alguna restricci�n impuesta sobre el modelo. En ese caso, el motor de adaptaci�n (\emph{Adaptation
					Engine}) determina el curso de acci�n y lleva a cabo la adaptaci�n necesaria.
					\item \textbf{Capa de Traducci�n}:\\
					Esta capa es la encargada de cubrir la brecha de abstracci�n existente entre el sistema en ejecuci�n y el modelo
					de su arquitectura (en ambos sentidos). En esta infraestructura, un repositorio de traducci�n mantiene diversos
					mapeos compartidos por distintos componentes dentro de esta capa, por ejemplo, una operaci�n a nivel modelo de la
					arquitectura en su correspondiente operaci�n de \emph{runtime}:

					\begin{Verbatim}[gobble=7]
						       Componente de Log::desactivar       <==>       Logger.disableLog()
					\end{Verbatim}
				\end{enumerate}

				Rainbow es un \emph{framework} desarrollado en el lenguaje de programaci�n Java\textsuperscript{\texttrademark} y
				todos los derechos sobre el c�digo fuente pertenecen al grupo ABLE de la Universidad de Carnegie Mellon. Los autores
				de este trabajo solicitaron permiso a este grupo para poder acceder al c�digo fuente de Rainbow para poder realizar
				la extensi�n objeto de este trabajo. En la wiki oficial de Rainbow pueden encontrarse instrucciones para
				instalar versiones ya compiladas del \emph{framework}. Para m�s informaci�n, visitar
				\url{http://rainbow.self-adapt.org/RainbowInstall}.

		\subsubsection{Conocimiento espec�fico del sistema}
		
			En la secci�n anterior hemos descrito la infraestructura b�sica provista por Rainbow. Es de notar que la misma no es
			suficiente para satisfacer las necesidades puntuales de auto adaptaci�n de un sistema en particular. Para
			lograr esto, es necesario extender dicha infraestructura, agregando conocimiento espec�fico del sistema que se desea
			adaptar. Este conocimiento (t�picamente no reutilizable entre distintos sistemas) incluye el modelo operacional
			del sistema, que define par�metros como tipos de componentes y propiedades, restricciones de comportamiento,
			estrategias de adaptaci�n, interfaz para acceder a la informaci�n de \emph{runtime} del sistema, as� tambi�n como
			para hacer efectivas las estrategias de reparaci�n, etc.

		\subsubsection{Stitch}
			A fin de disponer de una forma suficientemente expresiva de definir t�cticas y estrategias, Rainbow incluye
			un lenguaje de \emph{scripting} de prop�sito espec�fico llamado \textbf{Stitch}, el cual permite plasmar el
			conocimiento rutinario de las personas sobre adaptaci�n de sistemas de software.

			Algunas de las caracter�sticas innovadoras de Stitch:
			\begin{itemize}
				\item \textbf{Control del sistema}: La selecci�n de la pr�xima acci�n a ejecutar en el contexto de una estrategia
				depende de los efectos observados luego de la acci�n previa.
				\item \textbf{Sensibilidad al contexto}: La selecci�n de la mejor estrategia se realiza considerando el estado
				actual del sistema, mediante la inspecci�n de algunas de sus propiedades.
				\item \textbf{Asincronismo}: Stitch permite especificar un tiempo de demora luego de la ejecuci�n de una t�ctica
				para que los efectos de la t�ctica se puedan ver reflejados en el sistema.
			\end{itemize}

		\subsubsection{Ejemplo de una T�ctica en Stitch}

			En la figura \ref{fig:tactics_example} se puede apreciar un ejemplo de una t�ctica definida en Stitch para ser
			utilizada por Rainbow. Primeramente, se importa el modelo de la arquitectura del sistema en cuesti�n junto con la
			implementaci�n de un operador que permite impactar al sistema en ejecuci�n (estos operadores suelen ser provistos por
			el usuario de la aplicaci�n).
			
			\begin{figure}[h]
				\centering
					\includegraphics[width=0.98\textwidth]{images/tactics_example.png}
				\caption{Ejemplo de una t�ctica en Stitch}
				\label{fig:tactics_example}
			\end{figure}
			
			La t�ctica consiste en disminuir la fidelidad (a modo s�lo texto) del contenido provisto por todos los servidores
			cuando se detecta que al menos un cliente experimenta un tiempo de respuesta superior a un determinado umbral. Para lograr
			esto, Rainbow inspecciona las propiedades del modelo de la arquitectura del sistema, definido en ACME, el c�al se
			supone constantemente actualizado por Rainbow con respecto al estado actual del sistema en ejecuci�n.
			
			Por �ltimo, se especifica que el efecto esperado de ejecutar la t�ctica consiste en que todos los clientes
			experimenten un tiempo de respuesta inferior al umbral y que, por otro lado, todos los servidores se encuentren
			prestando servicio en modo s�lo texto.

		\subsubsection{Ejemplo de una Estrategia en Stitch}

			En la figura \ref{fig:strategies_example} podemos ver un ejemplo de una estrategia definida en Stitch para ser
			utilizada por Rainbow.

			\begin{figure}[h]
				\centering
					\includegraphics[width=0.98\textwidth]{images/strategies_example.png}
				\caption{Ejemplo de una estrategia en Stitch}
				\label{fig:strategies_example}
			\end{figure}

			La estrategia representa un algoritmo simple para disminu�r el tiempo de respuesta experimentado por el usuario de un
			sistema cliente-servidor. Se definen algunos predicados de primer orden que predican sobre propiedades del modelo de
			la arquitectura del sistema, el cual se asume constantemente actualizado por Rainbow con respecto al sistema en
			ejecuci�n. En el caso que haya en la arquitectura un conector que muestre alta latencia, se invoca la t�ctica
			definida en la Figura \ref{fig:tactics_example} (la cual cambia la fidelidad de todos los servidores a modo s�lo
			texto) y se espera 1 segundo antes de determinar el �xito o no de la ejecuci�n de la t�ctica.
			
			En el caso en el que no se detecte alta latencia o que la anterior t�ctica haya fallado, y siempre chequeando que se
			detecte alta carga en el sistema; se intenta ejecutar otra t�ctica que consiste en agregar un servidor m�s en pos de
			mejorar el \emph{concern} tiempo de respuesta. Si luego de esperar 2 segundos para que los efectos de la t�ctica
			puedan verse reflejados en el sistema, no hay m�s alta carga, la estrategia se considera exitosa, caso contrario, la
			misma se considera fallida.

		\subsubsection{EMA: Suavizando el Comportamiento}
		\label{sec:exponentialAverage}

			El proceso de adaptar un sistema de manera din�mica puede llegar a ser muy costoso, por ejemplo, en un sistema
			cliente-servidor, el asignar m�s servidores para atender las peticiones de los usuarios implica normalmente un coste
			monetario no despreciable. Otro ejemplo podr�a ser el suspender temporalmente la reproducci�n de videos en un sitio
			de noticias. Si bien esta decisi�n puede servir para disminu�r el tiempo de respuesta experimentado por los usuarios
			en un contexto de alta carga, tambi�n provoca una clara disminuci�n en la calidad del servicio ofrecido. Estos
			ejemplos concretos sirven para inferir que el disparar un mecanismo de auto reparaci�n en un sistema debe ser una
			decisi�n tomada con cierta cautela y en base a datos confiables y sostenidos en el tiempo.
			
			Hemos visto anteriormente, de manera somera, de que manera Rainbow (mediante estrategias y t�cticas definidas en el
			lenguaje Stitch) utiliza datos sobre el sistema en ejecuci�n para decidir el mejor camino a tomar para adaptar el
			sistema a los requerimientos de atributos de calidad correspondientes. Consideremos un escenario d�nde una estrategia
			debe decidir si el �ltimo paso ejecutado fue exitoso o no. Para eso, deber� consultar una o m�s propiedades del
			sistema en ejecuci�n, los cu�les, podr�an llegar a no ser representativos del entorno \emph{real} en que el sistema
			se encuentra ejecutando; debido a la presencia de algunos pocos valores considerablemente distintos del resto de los
			datos recientes \emph{hist�ricos} que se usan para tomar decisiones (\emph{outliers}).
			
			Rainbow implementa un mecanismo que permite evitar accionamientos prematuros de la auto reparaci�n debido a la
			presencia de \emph{outliers}. El mentado mecanismo utiliza \emph{Exponential Moving Average} (EMA)\footnote{Para
			m�s informaci�n, visitar\\
			\url{http://en.wikipedia.org/wiki/Exponential_moving_average\#Exponential_moving_average}}, que permite ponderar los
			valores hist�ricos de la(s) propiedad(es) del sistema que se consultan para tomar decisiones. La funci�n de suavizado
			que implementa esta heur�stica se define inductivamente de la siguiente manera:

			\begin{equation}
				S_0 = Y_0
			\end{equation}
			\begin{equation}
				S_t = \alpha \times Y_t + (1-\alpha) \times S_{t-1}
			\end{equation}

			donde $\alpha$ se denomina \textbf{factor de suavizado} y $0 < \alpha < 1$.
			
			El valor suavizado $S_t$ no es ni m�s ni menos que un promedio ponderado de la �ltima observaci�n $Y_t$ y el valor
			suavizado previo $S_{t-1}$.
			
			Notar que con valores \emph{altos} de $\alpha$, el �ltimo valor observado tendr� m�s preponderancia que el valor
			hist�rico (suavizado) anterior. Por el contrario, con un $\alpha$ tendiendo a cero, la �ltima observaci�n de una
			propiedad de la arquitectura no tendr� mucha relevancia en el valor promedio $S_t$. No existe un procedimiento formal
			para determinar el valor de $\alpha$, en el caso particular de las pruebas realizadas en este trabajo, se ha elegido
			$\alpha = 0.3$, es decir que, se ponderar� con un 30\% al �ltimo valor observado mientras que el valor hist�rico
			tendr� un peso del 70\%; con esto nos aseguramos que los valores de las propiedades del sistema que son relevantes
			para la auto adaptaci�n no fluct�en bruscamente debido a unos pocos \emph{outliers}.

	\subsubsection{Znn: Evaluando la Efectividad de Rainbow}

		Znn es un sistema que simula un sitio web de noticias, el cual naci� en el contexto de la tesis de doctorado
		\cite{TesisOwen} de un investigador de la universidad de Carnegie Mellon, Shang-Wen Cheng. En dicha tesis, se
		eval�a a Rainbow en los siguientes aspectos:
		\begin{itemize}
			\item su efectividad para mantener los atributos de calidad ante condiciones cambiantes.
			\item la sobrecarga de procesamiento que implica la auto reparaci�n.
			\item el esfuerzo que implica agregar auto reparaci�n mediante Rainbow a Znn.
		\end{itemize}

		Si bien Znn y sus herramientas asociadas (\emph{probes}, \emph{gauges}, t�cticas, estrategias, etc.) nacieron con
		el �nico objetivo de evaluar la efectividad de Rainbow, las mismas han sido abiertas a la comunidad para que puedan
		ser utilizadas para tomar m�tricas y poder comparar distintos sistemas de auto reparaci�n.

		Znn provee un entorno de simulaci�n de una arquitectura cliente servidor ampliamente configurable que permite
		representar situaciones y controlar las variables de simulaci�n permitiendo modificarlas en cualquier punto de �sta
		�ltima.
		
		Por ejemplo, es posible configurar a Znn para que inicie con solamente 2 clientes, mostrando all� un desempe�o
		aceptable y que luego se agreguen 10 clientes, comprometiendo as� la \emph{performance} del mismo. De esta manera, Znn
		permite simular como responder�an las estrategias implementadas ante dicha situaci�n.

		Rainbow soporta dos modos de ejecuci�n: el normal, con un sistema real conectado al mismo � un modo de simulaci�n, el
		cu�l permite utilizar, por ejemplo, las herramientas provistas por Znn. En el presente trabajo se utilizar� el modo
		simulaci�n para los casos de prueba, tomando las estrategias y t�cticas provistas por Znn. �stas debieron ser
		adaptadas para que puedan seguir funcionando con los cambios propuestos en el presente trabajo.

	\subsection{Escenarios de Atributos de Calidad}
		\label{sec:QAS}

		\todo{TASK: Extender definici�n de Escenarios de Atributos de Calidad (poco, est� casi OK as�)}

		\todo{tomar info de ac� que est� muy bueno y conciso:
		\url{http://etutorials.org/Programming/Software+architecture+in+practice,+second+edition/Part+Two+Creating+an+Architecture/Chapter+4.+Understanding+Quality+Attributes/4.3+System+Quality+Attributes/}}

		Los atributos de calidad pueden ser representados mediante escenarios. Los escenarios est�n compuestos por:

		\begin{itemize}
			\item \textbf{Fuente del est�mulo}: Interna o externa \todo{mejorar esta explicaci�n}
			\item \textbf{Est�mulo}: condici�n que debe ser tenida en cuenta al llegar al sistema
			\item \textbf{Entorno}: condiciones en las cuales ocurre el est�mulo
			\item \textbf{Artefacto}: el sistema o partes de �l afectadas por el est�mulo
			\item \textbf{Respuesta}: qu� hace el sistema ante la llegada del est�mulo
			\item \textbf{Medici�n de la respuesta}: cuantificaci�n de un atributo de la respuesta
		\end{itemize}

		\begin{figure}[h]
			\centering
				\includegraphics{images/scenario.png}
			\caption{Visi�n gr�fica de un escenario}
			\label{fig:scenario}
		\end{figure}

		Los escenarios son peque�as historias que describen una interacci�n con el sistema, que impacta sobre un atributo de
		calidad en particular. Por ejemplo, un escenario sobre disponibilidad podr�a ser:
			\begin{quote}
			``Un proceso del sistema recibe un mensaje externo no anticipado durante un modo de operaci�n normal. El proceso
			informa al operador y contin�a su operaci�n sin ca�das.''
			\end{quote}

		Este escenario se descompone de la siguiente manera:
		\begin{itemize}
			\item \textbf{Fuente del est�mulo}: Sistema externo
			\item \textbf{Est�mulo}: Mensaje no anticipado
			\item \textbf{Entorno}: Operaci�n normal
			\item \textbf{Artefacto}: Proceso interno
			\item \textbf{Respuesta}: Informar al operador y seguir operando
			\item \textbf{Medici�n de la respuesta}: sin ca�das (downtime)
		\end{itemize}

		Los escenarios permiten obtener el punto de vista de un grupo diverso de \emph{stakeholders} (arquitectos,
		desarrolladores, usuarios, sponsors, etc). Estos escenarios pueden luego ser utilizados para analizar la arquitectura
		del sistema e identificar concerns y posibles estrategias para atacar problemas.


		\subsubsection{QAW}
		\todo{fuente de lo que sigue: http://www.sei.cmu.edu/architecture/tools/qaw/index.cfm}

		Existe una metodolog�a definida por el Software Engineering Institute (SEI) conocida como \textbf{Quality Attribute
		Workshop (QAW)}, cuya principal herramienta son los escenarios. Esta metodolog�a provee un m�todo para identificar los
		atributos de calidad cr�ticos de la arquitectura de un sistema, tales como disponibilidad, \emph{performance},
		seguridad, etc, que son derivados de objetivos del negocio del sistema. QAW no asume la existencia de una arquitectura
		del software, sino que fue desarrollado como consecuencia de la necesidad de \emph{stakeholders} y arquitectos de un
		m�todo que permita identificar los atributos de calidad importantes para el correcto funcionamiento del sistema antes de
		que su arquitectura estuviese definida.

		QAW consiste en reuniones en las que participan todos los \emph{stakeholders} del sistema, en las que se definen los
		escenarios que en definitiva representar�n los requerimientos de atributos de calidad que el sistema idealmente deber�
		satisfacer. Una vez definidos todos los escenarios, el siguiente paso del QAW consiste en priorizar y refinar los
		escenarios, por ejemplo agregando detalles adicionales tales como los participantes involucrados, la secuencia de
		actividades, y preguntas sobre el requerimiento que representa cada escenario. El proceso de refinar los escenarios
		permite a los \emph{stakeholders} comunicarse entre ellos, exponiendo supuestos que pueden no ser tan claros para el
		resto de los participantes. Dicho refinamiento tambi�n proporciona una visi�n de c�mo interact�an los atributos
		de calidad entre s�, sirviendo de base para definir \emph{tradeoffs} entre estos atributos.

		El proceso QAW termina con la lista de escenarios refinados y priorizados. Los mismos pueden servir para definir casos
		de pruebas, o como semillas para el proceso ATAM.
		
		Si bien no es requisito excluyente utilizar la metodolog�a QAW para poder emplear Arco Iris, su uso es recomendado ya
		que har� uso intensivo de los escenarios de atributos de calidad, y de su correcta definici�n depender� el nivel de 
		optimizaci�n y flexibilizaci�n que Arco Iris puede alcanzar al auto reparar un sistema.

	\subsection{ATAM}
	\label{sec:atam}
		\todo{fuente: svn/doc/atam/ATAM Method for Architecture Evaluation.pdf}

		\todo{TASK: Extender definici�n de ATAM, poner grafiquito, poner foco en la parte de sensitivity point y tradeoff
		point que son los que nos sirven. Poner un ejemplo de c/u.}

		ATAM (Architecture Tradeoff Analysis Method) ha sido desarrollado por el SEI, al igual que QAW, y es una t�cnica que
		permite analizar arquitecturas de software. Las arquitecturas son complejas, e involucran muchos tradeoffs de dise�o.
		Sin un proceso de an�lisis formal, no se puede estar seguro de que las decisiones de arquitectura tomadas - en
		particular aquellas que afectan el cumplimiento de requerimientos de calidad como \emph{performance}, disponibilidad,
		seguridad y modificabilidad - son adecuadas para mitigar los riesgos.

		El nombre ATAM proviene no solo del hecho de que refleja cu�n bien una arquitectura satisface objetivos de calidad
		espec�ficos, sino tambi�n del hecho de que provee una visi�n de c�mo esos objetivos de calidad interact�an entre s�,
		esto es lo que conocemos como tradeoffs.

		La meta de evaluar una arquitectura con ATAM es entender las consecuencias de las decisiones arquitect�nicas con
		respecto a los requerimientos de atributos de calidad del sistema. Otro objetivo fundamental de ATAM es determinar si
		dichos requerimientos pueden ser alcanzados con la arquitectura concebida, antes de destinar grandes cantidades de
		recursos a la construcci�n del software.

		ATAM es un m�todo estructurado y repetible, ayudando as� a que las preguntas correctas sobre la arquitectura sean
		planteadas de manera temprana en el proyecto, durante las etapas de an�lisis de requerimentos y de dise�o, en las
		cuales los problemas detectados pueden ser corregidos sin mayores costos. ATAM gu�a a los usuarios del m�todo
		(\emph{stakeholders}) para que busquen conflictos en la arquitectura y soluciones a dichos conflictos.

		Cabe aclarar que los QAW han surgido como consecuencia del uso de ATAM, ya que usuarios de �ste �ltimo solicitaban una
		herramienta o m�todo que les permitiera identificar los requerimientos y los atributos de calidad m�s importantes del
		sistema, pero antes de que existiese la arquitectura sobre la cual ATAM trabajar�a.

		\todo{Sensitivity points, tradeoffs, etc.}
	
	\newpage
	
	\section{Extensi�n a Rainbow: Arco Iris}

	\subsection{Introducci�n}

		Como ya se ha comentado anteriormente, la idea de este trabajo es extender el \emph{framework} ``Rainbow'' para poder
		lograr un mecanismo de auto reparaci�n m�s flexible y con mayor capacidad expresiva, con el objetivo de proveer mayor
		visibilidad a los \emph{stakeholders} de la aplicaci�n sobre dicho proceso, permiti�ndoles as� involucrarse en la
		definici�n de escenarios de uso real del sistema, incluyendo la relaci�n de estos �ltimos con los atributos de
		calidad requeridos. La soluci�n intenta involucrarlos tambi�n en la definici�n de prioridades de los escenarios
		y/o estrategias a considerar en la auto reparaci�n del sistema cuando uno de los escenarios no se cumple.

		A fin de lograr lo antedicho, se consider� necesario extender el \emph{framework} Rainbow, cuyo c�digo se encuentra
		escrito mayormente en el lenguaje de programaci�n Java.

		Los integrantes del proyecto ABLE, que son quienes poseen propiedad intelectual sobre el \emph{framework}, no
		s�lo decidieron proveer el c�digo fuente completo para poder realizar esta extensi�n, sino que estuvieron dispuestos
		en todo momento a colaborar, responder dudas e inquietudes, hacer sugerencias, etc.

	\subsection{Rainbow ``out of the box''}

		El primer paso fue inspeccionar el c�digo del framework y leer su documentaci�n (la cual no es abundante, por
		cierto). Luego de ese primer paso, se tuvo una buena noci�n sobre el funcionamiento de Rainbow y sus componentes principales
		de arquitectura, las cuales se pueden observar en el siguiente diagrama de colaboraci�n de objetos:

		\begin{center}
			\includegraphics[width=1.00\textwidth]{images/Rainbow_Architecture.png}
		\end{center}

		Observamos que, en Rainbow, una persona con rol de arquitecto (o similar) es el encargado de configurar el framework
		utilizando, b�sicamente, dos v�as:
		\begin{enumerate}
			\item la creaci�n de un modelo de la arquitectura del sistema al cual Rainbow va a adaptar. Dicho modelo se
			especifica utilizando el estilo de componentes y conectores (C\&C) y el lenguaje de descripci�n de arquitecturas
			ACME, ya descripto en el presente trabajo. En dicho lenguaje, los componentes y conectores poseen propiedades (con
			valores asociados), \emph{constraints} y tambi�n existe la posibilidad de especificar invariantes a nivel de sistema
			o sub sistema. Dichos invariantes ser�n evaluados peri�dicamente por Rainbow para verificar que el sistema funcione
			dentro de los l�mites determinados como normales.
			\item la generaci�n de un conjunto de archivos de configuraci�n que especifican diversos aspectos relacionados con
			la definici�n del sistema a auto-reparar, como por ejemplo: ubicaci�n f�sica del archivo ACME que describe la
			arquitectura de este �ltimo, archivos de t�cticas y estrategias de reparaci�n (estos escritos en un lenguaje de
			\emph{scripting} de prop�sito espec�fico llamado ``stitch''), datos para la configuraci�n de la conexi�n de Rainbow
			con el sistema en \emph{runtime}, etc.
		\end{enumerate}

		Podemos ver tambi�n en el gr�fico anterior, c�mo los denominados \emph{Probes} son los componentes designados en
		Rainbow para interactuar directamente con el sistema a adaptar (el \emph{Target System}), obteniendo as� informaci�n
		relevante a los fines de su auto-reparaci�n. Dicha informaci�n no es interpretada dentro de los probes sino que ellos
		delegan dicha tarea a los denominados \emph{Gauges}. Estos componentes son los encargados de entender la informaci�n
		provista por los gauges y traducirla a pares $<$Propiedad,Valor$>$ d�nde la \emph{Propiedad} refiere al nombre de una
		propiedad correspondiente a un componente (o conector o subsistema, etc\ldots) de la arquitectura del sistema a
		adaptar; y \emph{Valor} es ni m�s ni menos que el nuevo valor que posee dicha propiedad.

		Normalmente, existe un par $<$Gauge, Probe$>$ distinto (que a nivel implementativo son clases Java) por cada tipo
		relevante de \emph{concern} que interese ser monitoreado en tiempo de ejecuci�n. Un ejemplo de \emph{concern} podr�a
		ser: ``tiempo de respuesta experimentado por el usuario'', el cual es un \emph{concern} relacionado al atributo de
		calidad \emph{performance}.

		Asociado al tipo de componente \emph{Gauge}, se encuentra el \emph{Gauge Coordinator} el cual, tal como su nombre lo
		sugiere, coordina la informaci�n provista por todos los gauges e invoca a la operaci�n \verb@update property@ del
		componente \emph{Rainbow Model}.

		\emph{Rainbow Model}, componente clave en la arquitectura del \emph{framework}, tiene, entre otras cosas, la
		responsabilidad de hacer efectiva la actualizaci�n de los valores de la propiedades de los componentes, conectores,
		sub-sistemas o sistemas del modelo de arquitectura de la aplicaci�n a adaptar. Tambi�n verifica peri�dicamente si se
		ha violado alguna de las \emph{constraints} presentes en el modelo de arquitectura con el cual el \emph{framework} ha
		sido inicializado.

		Por otra parte, existe otro componente llamado \emph{Architecture Evaluator}, el cual consulta peri�dicamente a
		\emph{Rainbow Model} sobre si se ha violado alguna constraint. En ese caso, el \emph{Architecture Evaluator} dispara
		el mecanismo de adaptaci�n invocando al \emph{Adaptation Manager} (este �ltimo, junto con el \emph{Rainbow Model}, el
		componente m�s importante de todos).

		El \emph{Adaptation Manager} sigue una l�gica un tanto compleja (que luego explicaremos en detalle) para determinar la
		mejor estrategia de reparaci�n a realizar y luego, mediante \emph{Effectors} (no mostrados en el gr�fico anterior por
		claridad), la l�gica embebida en las estrategias de reparaci�n impactan en el sistema a ser adaptado, siempre
		intentando acercar el estado del sistema a sus invariantes pre-establecidos.

	\subsection{Rainbow + Escenarios = ``Arco Iris''}

			En el presente apartado repasaremos someramente las caracter�sticas principales de la arquitectura de Rainbow
			luego de incorporar las extensiones planteadas en el presente trabajo, a las cuales denominamos ``Arco Iris''.

		\subsubsection{Compatibilidad hacia atr�s}

			Uno de los objetivos de dise�o m�s importantes planteados al momento de pensar ``Arco Iris'' fue sin
			duda el intentar que la presente extensi�n fuera ``compatible hacia atr�s'' con \emph{Rainbow}, es decir, es
			altamente deseable que la extensi�n se inserte en el \emph{framework} (conceptualmente hablando) de manera similar a
			c�mo un \emph{plug-in} trabaja en cualquier otro sistema de software; pudiendo en este caso el usuario de manera
			relativamente simple poder elegir entre utilizar los mecanismos de auto reparaci�n provistos por ``Arco Iris'' solos,
			conjuntamente con los de Rainbow, o usar unicamente Rainbow original sin extensi�n alguna.  La implementaci�n de
			``Arco Iris'' cumple con dicho requerimiento ya que en la misma se extienden (en el sentido estricto de la palabra)
			componentes, sin cambiar su l�gica original. Es importante notar que para que la extensi�n sea f�cilmente adicionable
			y/o removible se necesitar�a que los creadores de Rainbow hagan m�nimos cambios a la visibilidad de algunos
			componentes del \emph{framework} (i.e. clases Java), ya que su implementaci�n est� hecha de tal manera que impide la
			extensi�n, ya sea v�a herencia o composici�n de clases.

		\subsubsection{Arquitectura de Arco Iris}

			A continuaci�n, veremos c�mo luce la arquitectura del framework con la incorporaci�n de las extensiones realizadas:

			\begin{center}
				\includegraphics[width=1.00\textwidth]{images/Rainbow_Architecture_With_Scenarios.png}
			\end{center}

			El arquitecto sigue realizando las mismas tareas que realizaba en Rainbow (i.e. ``alimentar'' a Rainbow con un
			modelo de la arquitectura del sistema a adaptar y archivos de configuraci�n) pero ahora se le suma una tarea m�s: el
			configurar los escenarios de atributos de calidad; tarea que realiza en conjunto con una o m�s personas que asumen
			distintos roles que normalmente se engloban en la palabra \emph{stakeholders} (e.g. analistas funcionales, usuarios
			del sistema, l�deres y sponsors del proyecto, clientes, etc.)

			En Rainbow original (i.e. sin extensiones) el framework deja al usuario final la responsabilidad por codificar los
			\emph{probes} y \emph{gauges} que recolectar�n informaci�n del sistema a adaptar y traducir�n esa informaci�n a
			cambios en los valores de las propiedades del sistema. Esto sigue siendo igual en Rainbow con las extensiones
			provistas por Arco Iris, es decir, el usuario final sigue siendo el encargado de crear los componentes
			que en tiempo de ejecuci�n encuestan al sistema peri�dicamente para obtener informaci�n relevante.

			As�, en el anterior diagrama podemos observar que tanto el \emph{probe} como el \emph{gauge} normalmente tendr�n que
			tener conocimiento del concepto de ``est�mulo'' (tal cual est� descrito por ATAM en sus escenarios de atributos de
			calidad).

			Puesto que en la presente extensi�n el componente \emph{Gauge} ahora provee informaci�n sobre los est�mulos que
			desencadenaron un cambio de propiedades en el sistema ejecut�ndose, el componente \emph{Gauge Coordinator} ha sido
			extendido (ahora llamado \emph{Gauge Coordinator With Scenarios}) para soportar dicho agregado de informaci�n, sin
			perder el comportamiento original provisto por Rainbow. Esto hace que este componente siga pudiendo coordinar informaci�n
			proveniente de \emph{gauges} de Rainbow y \emph{gauges} con conocimiento sobre el est�mulo invocado en el sistema en
			ejecuci�n.

			De la misma manera anterior, el componente \emph{Rainbow Model With Scenarios} ha sido extendido para manejar
			informaci�n de est�mulos proveniente del \emph{Gauge Coordinator With Scenarios} sin perder el soporte original
			provisto por \emph{Rainbow Model}. Tambi�n, \emph{Rainbow Model With Scenarios} es el componente que carga los
			escenarios de atributos de calidad creados por los \emph{stakeholders} y arquitecto(s).

			En el caso de que el \emph{Rainbow Model} detecte que un est�mulo ha sido invocado en el sistema en ejecuci�n,
			proceder� a recabar el subconjunto de escenarios habilitados que poseen dicho est�mulo y que a su vez se encuentran
			no cumpli�ndose\footnote{De ahora en m�s, diremos que el escenario en esta situaci�n se encuentra ``roto''.}. Luego
			de obtener este subconjunto de escenarios, proceder� a invocar al \emph{Adaptation Manager With Scenarios}, extensi�n
			del componente original que soporta tanto invocaciones desde el \emph{Architecture Evaluator} original de Rainbow,
			como as� tambi�n invocaciones con un conjunto de escenarios detectados como ``rotos'' a los cuales debe intentar
			reparar considerando diversas variables como por ejemplo sus prioridades relativas.

		\subsubsection{Actualizaci�n din�mica de cambios de configuraci�n}

			Una de las principales limitaciones de Rainbow es la imposibilidad de actualizar la configuraci�n del
			\emph{framework} sin tener que reiniciar su ejecuci�n. A fin de superar (al menos parcialmente) tal limitaci�n,
			proponemos un simple mecanismo de detecci�n de cambios en el archivo de configuraci�n de Arco Iris \todo{poner una
			referencia a la parte del doc donde se explica en detalle este archivo}, el cual, al detectar cualquier cambio en
			dicho archivo, \textbf{reemplazar� din�micamente la configuraci�n de Arco Iris cargada en memoria por la nueva
			configuraci�n}; todo esto sin necesidad de reiniciar el \emph{framework}.
			
			El mecanismo se ejecuta peri�dicamente cada $X$ milisegundos d�nde $X$ es configurable utilizando el archivo de
			configuraci�n est�ndar de Rainbow (inicialmente, $X = 5000$, es decir, el mecanismo se ejecuta cada 5 segundos).
			
			Se utiliza el patr�n \emph{Observer}\footnote{Para m�s informaci�n acerca de Eclipse, visitar
			\url{http://en.wikipedia.org/wiki/Observer_pattern}} como modo de notificar a todos aquellos objetos interesados en
			llevar alguna acci�n a cabo como consecuencia de un cambio en la configuraci�n. El componente m�s interesado en
			conocer cuando un cambio en la configuraci�n tiene lugar es el denominado \textbf{SelfHealingConfigurationManager},
			el cual, en ese caso, recalcula una variada cantidad de informaci�n guardada en memoria para que pueda ser usada
			instant�neamente por el componente \textbf{RainbowModelWithScenarios}.
			
			El mecanismo descripto en este apartado se encuentra implementado en la clase\\
			\mbox{\textbf{FileSelfHealingConfiguracionDao}} y podemos ver su c�digo a continuaci�n:
			
			\begin{Verbatim}[gobble=4]
				private static final long CONFIG_RELOAD_INTERVAL_MS =
					Long.valueOf(Rainbow.property("customize.scenarios.reloadInterval"));
			
				private static final String SELF_HEALING_CONFIG_FILE_NAME =
					Rainbow.property("customize.scenarios.path");
			
				private static final File SELF_HEALING_CONFIG_FILE =
					Util.getRelativeToPath(Rainbow.instance().getTargetPath(), SELF_HEALING_CONFIG_FILE_NAME);
				(...)
				public FileSelfHealingConfigurationDao() {
					super();
					this.listeners = new HashSet<SelfHealingConfigurationChangeListener>();
					this.loadSelfHealingConfigurationFromFile();
			
					TimerTask task = new FileChangeDetector(SELF_HEALING_CONFIG_FILE) {
						@Override
						protected void onChange(File file) {
							logger.info(SELF_HEALING_CONFIG_FILE_NAME +
								" has just changed, reloading Self Healing Configuration!");
							loadSelfHealingConfigurationFromFile();
							notifyListeners();
						}
					};
			
					Timer timer = new Timer();
					timer.schedule(task, new Date(), CONFIG_RELOAD_INTERVAL_MS);
				}
				(...)
			\end{Verbatim}

	\subsection{ATAM para Flexibilizar la Auto Reparaci�n}

		Actualmente Rainbow posee conocimiento sobre la arquitectura del sistema a adaptar mediante un modelo de su arquitectura
		expresado en el lenguaje de descripci�n de arquitecturas Acme. Uno de los objetivos de Arco Iris es extender el
		conocimiento que el \emph{framework} tiene sobre el sistema en general. Este nuevo conocimiento ser� utilizado por Arco
		Iris en tiempo de ejecuci�n para refinar y optimizar la auto reparaci�n.

		Esencialmente, se incluye informaci�n sobre los atributos de calidad del sistema que son relevantes para los
		\emph{stakeholders} del sistema en tiempo de ejecuci�n. Se permite, por ejemplo, poder describir la importancia relativa
		de la \emph{performance}, la usabilidad, la disponibilidad, etc.; definiendo as� una serie de \emph{tradeoffs} entre
		distintos atributos de calidad requeridos por el sistema. El enfoque propuesto para lograr esto consiste en especificar
		\textbf{Escenarios de Atributos de Calidad} \cite{Scenarios} (de ahora en m�s, simplemente ``Escenarios''), tal cual
		fueron descriptos anteriormente, aunque con algunos agregados de informaci�n orientados a la auto reparaci�n.

		Como ya vimos anteriormente, un Escenario modela una situaci�n concreta y real de uso del sistema ante la cual debe
		comportarse de una manera esperada. Los escenarios contienen la siguiente informaci�n:
		\begin{itemize}
			\item Fuente del Est�mulo
			\item Est�mulo
			\item Artefacto
			\item Entorno
			\item Respuesta
			\item Cuantificaci�n de la Respuesta
		\end{itemize}

		Supongamos que tenemos el siguiente escenario:

		\begin{quote}
			Un proceso del sistema recibe un mensaje externo no anticipado durante un modo de operaci�n normal. El
			proceso informa al operador y contin�a su operaci�n sin ca�das.
		\end{quote}

		Veamos c�mo se desgloza la informaci�n de este escenario seg�n la descomposici�n antes mencionada:

		\begin{itemize}
			\item Fuente: Sistema externo
			\item Est�mulo: Mensaje no anticipado
			\item Entorno: Operaci�n normal
			\item Artefacto: Proceso interno
			\item Respuesta: Informar al operador y seguir operando
			\item Medici�n de la respuesta: sin ca�das (downtime)
		\end{itemize}

		\indent De todos estos atributos, el \textbf{Est�mulo}, el \textbf{Artefacto}, el \textbf{Entorno} y la
		\textbf{Cuantificaci�n de la Respuesta} son particularmente relevantes a los fines de establecer informaci�n �til para
		el mecanismo de auto reparaci�n. A lo largo de las pr�ximas subsecciones mostraremos c�mo Arco Iris hace uso de esta
		informaci�n para modificar, optimizar y refinar la auto reparaci�n realizada por Rainbow.


		\subsubsection{El Est�mulo en la Auto Reparaci�n}

			El \textbf{Est�mulo} de un escenario normalmente se asocia a un evento desencadenado en el sistema por la acci�n del
			alguno de sus usuarios. Dicho evento es el punto de entrada del escenario, el disparador (interno o externo) que
			inicia la interacci�n con el sistema, y m�s particularmente, con el artefacto del escenario en cuesti�n. Por ejemplo,
			supongamos que en un sistema de administraci�n de cuentas bancarias un cliente intenta hacer una transferencia, en
			este caso la fuente del est�mulo ser�a el cliente y el est�mulo en s� mismo ser�a realizar transferencia.

			Saber cual es el est�mulo de cada escenario permite optimizar la auto reparaci�n, ya que habiendo ocurrido determinado
			est�mulo, Arco Iris podr� detectar cuales son los potenciales escenarios que pueden verse afectados y trabajar�
			verificando dicho subconjunto, evitando as� chequeos innecesarios.

			Cabe destacar que tambi�n se ofrece la posibilidad de que el est�mulo sea ``cualquiera'', es decir, que el escenario
			aplique siempre, independientemente del est�mulo que haya impactado al sistema. Esto puede ser �til para casos m�s
			gen�ricos, por ejemplo, si se requiere que el tiempo de respuesta experimentado por el usuario nunca sobrepase
			determinado umbral sin importar la funcionalidad del sistema que el usuario est� utilizando. De no configurar ning�n
			est�mulo en ning�n escenario, Arco Iris verificar� que se satisfagan todos los escenarios en cada iteraci�n. Esta es
			la manera en que trabaja Rainbow.

			Entonces, a diferencia de Rainbow, Arco Iris permite refinar el conjunto de escenarios que pudieron verse afectados por
			medio del est�mulo que ocasion� el problema, reduciendo as� la cantidad de escenarios a verificar y sobre todo la
			cantidad de estrategias a evaluar para intentar resolverlo, mientras que en Rainbow (en Arco Iris se mantiene esta
			posibilidad), la �nica manera de limitar el uso de las estrategias es agregar restricciones en la misma estrategia,
			siendo esto algo bastante t�cnico y quedando fuera del alcance de la mayor parte de los \emph{stakeholders} del
			sistema, por ejemplo, para que una estrategia que soluciona problemas de performance no sea tenida en cuenta al
			reparar un escenario relacionado a un atributo de calidad distinto, como por ejemplo el costo, debe agregarse una
			precondici�n como la siguiente en la estrategia:

			\begin{Verbatim}[gobble=4]
				strategy SimpleReduceResponseTime
				  [ styleApplies && responseTimeConstraintViolation ] {
				  ...
				}
			\end{Verbatim}

			Donde \emph{responseTimeConstraintViolation} es un predicado que permite determinar si existe alg�n cliente cuyo tiempo
			de respuesta haya sobrepasado el m�ximo tiempo de respuesta permitido:

			\begin{Verbatim}[gobble=4]
				define boolean responseTimeConstraintViolation =
					exists c : T.ClientT in M.components | c.experRespTime > M.MAX_RESPTIME;
			\end{Verbatim}

			Es importante remarcar que, en caso de que el problema que se est� intentando reparar no est� relacionado con la
			performance, Rainbow deber� de todos modos chequear el tiempo de respuesta de todos los clientes, mientras que en
			Arco Iris existe la posibilidad de evitarlo al configurar correctamente los est�mulos de cada escenario.

			Adentr�ndonos ya en la implementaci�n, tengamos en cuenta que el est�mulo era un concepto inexistente en Rainbow, por lo
			cual en Arco Iris debimos modificar algunos de los componentes existentes para poder tomar conocimiento de qu� est�mulo
			desencadenaba una determinada acci�n, as� como tambi�n para interpretar luego dicha informaci�n.
			Dado a que Rainbow permite trabajar en modo simulaci�n, fue necesario implementar dos soluciones para poder utilizar el
			concepto de est�mulo en ambos escenarios.

			Al trabajar en el modo de simulaci�n, dado que no participan ni los \emph{probes} ni los \emph{gauges}, fue necesario
			implementar un mecanismo ad hoc, que consisti� en barrer todos los escenarios al iniciar Arco Iris, recogiendo en un
			mapa la informaci�n de qu� est�mulos eran los posibles responsables de las modificaciones causadas sobre cada una de las
			propiedades de los artefactos configurados en los escenarios. A continuaci�n mostramos el concepto de dicha
			implementaci�n en pseudoc�digo:

			\begin{Verbatim}[gobble=4]
				por cada escenario habilitado
				  por cada constraint del escenario
				    por cada propiedad involucrada en el constraint
				      escenariosPorEstimulo.agregar(escenario.estimulo, escenario)
				      estimulosPorPropiedades.agregar(propiedad, escenario.estimulo)
			\end{Verbatim}

			El modo simulaci�n funciona notificando los cambios en las propiedades de inter�s, las cuales son
			determinadas a la hora de implementar la simulaci�n. Entonces, al contar los est�mulos relacionados con cada
			propiedad y los escenarios por cada est�mulo, podemos acotar la cantidad de escenarios a verificar en cada
			ejecuci�n de la auto reparaci�n.

			En modo real, o sea sin simulaci�n, para notificar el est�mulo desencadente fue necesario extender los
			\emph{probes}, que son los componentes encargados de testear una determinada funcionalidad del sistema,
			siendo el responsable tambi�n de medir la respuesta y de reportar dicha informaci�n a un bus, del cual
			luego la consumir�n los \emph{gauges}, quienes interpretar�n los datos transform�ndolos en cambios en el
			modelo de la arquitectura. Obviamente tambi�n fue necesario extender la funcionalidad de los \emph{gauges}
			para poder intrepretar la informaci�n de est�mulo agregada en cada uno de los \emph{probes}. En ambos
			casos, y gracias al mecanismo empleado por Rainbow, no fue necesario modificar las interfaces ni de los
			\emph{probes} ni de los \emph{gauges}, ya que �stos se comunican mediante cadenas de texto, justamente
			para ofrecer una amplia libertad a la hora de implementarlos ya que pueden existir maneras muy diversas de
			testear y tomar m�tricas de las distintas funcionalidades de los sistemas. Luego, una vez obtenido el est�mulo
			correspondiente, no resta m�s que recurrir al mapa de escenarios por est�mulos (utilizado tambi�n en el modo
			simulaci�n) para acotar la cantidad de escenarios a comprobar.

			\todo{agregar comparaci�n de info reportada por los Probes con estimulos y sin estimulos}

		\subsubsection{El Artefacto en la Auto Reparaci�n}

			El \textbf{Artefacto} se refiere al componente, subsistema o parte del sistema afectada por el escenario. Dado que
			Rainbow trabaja con el modelo de la arquitectura del sistema descripto en el lenguaje Acme, contaremos con la
			especificaci�n de las propiedades de los componentes y los conectores del sistema, entonces en el escenario de Arco Iris
			podremos contar con una vinculaci�n directa con los componentes afectados. Al seleccionar un artefacto, el usuario de
			Arco Iris no estar� m�s que eligiendo un componente o conector de la arquitectura del sistema, esto lo aprovecharemos
			para acotar las propiedades sobre las cuales podr� predicar la \textbf{Cuantificaci�n de la Respuesta}.

			Para ejemplificar el concepto de artefacto utilizaremos la arquitectura presentada en Znn, recordemos que es una
			arquitectura cliente-servidor, de la cual en este caso tomaremos la especificaci�n del componente cliente definido como
			\emph{ClientT}, a saber:

			\begin{Verbatim}[gobble=4]
				Component Type ClientT extends ArchElementT with {

				  Property deploymentLocation : string <<  default : string = "localhost"; >> ;

				  Property experRespTime : float <<  default : float = 100.0; >> ;

				  Property reqRate : float <<  default : float = 0.0; >> ;
				}
			\end{Verbatim}

			Y supongamos que se define el siguiente escenario:

			\label{escenarioPerformance}
			\begin{quote}
				Znn news debe servir el contenido de las noticias a los clientes en un tiempo de respuesta menor a 3 segundos en un
				entorno de operaci�n normal.
			\end{quote}

			Claramente en este ejemplo \emph{ClientT} ser� el artefacto sobre el cual predicar� la verificaci�n de validez de
			este escenario, m�s adelante veremos c�mo se configurar� dicha restricci�n.

			En la secci�n \ref{sec:scenariosUI} se explicar� en m�s detalle c�mo se podr� emplear esta informaci�n para mejorar
			la experiencia de los \emph{stakeholders} a la hora de especificar escenarios mediante la herramienta visual
			presentada junto a Arco Iris: \textbf{scenarios UI}


		\subsubsection{El Entorno en la Auto Reparaci�n}

			Dentro de la definici�n de escenario, el \textbf{Entorno} se refiere al estado en el que el sistema se encuentra cuando
			se recibe el est�mulo que desencadena el escenario, por ejemplo, al recibir una solicitud de creaci�n de una cuenta
			bancaria el sistema puede encontrarse en ``operatoria normal'' o en ``alta carga''. El entorno condiciona la validez del
			escenario en cuesti�n a que el sistema se encuentre en un determinado estado. Una respuesta aceptable o f�cil de cumplir
			bajo un entorno puede ser inaceptable o muy costosa de cumplir en un entorno diferente. En el escenario planteado
			anteriormente en la secci�n \ref{escenarioPerformance}, si el sistema se encuentra en un entorno de ``alta carga'' el
			escenario autom�ticamente se cumple, ya que para considerar dicho escenario el sistema deber�a encontrarse en un entorno
			de ``operaci�n normal''.

			Por otro lado, el entorno del escenario permite a Arco Iris optimizar la b�squeda de escenarios que no se satisfagan ya
			que se ignorar�n los escenarios cuyo entorno posea condiciones que no se cumplan en el estado del sistema en el instante
			en que se recibe el est�mulo. Es importante mencionar que si bien seg�n la definici�n de ATAM, cada escenario posee
			un �nico entorno, en Arco Iris hemos decidido, por razones de flexibilidad, permitir configurar varios escenarios,
			siendo obligatorio indicar al menos uno.

			Arco Iris tambi�n provee la posibilidad de especificar que un determinado escenario aplica bajo cualquier entorno,
			esto ser�a equivalente a que el entorno del escenario tenga como �nica condici�n subyacente \verb@true@. Esto puede
			ser �til para simplificar la configuraci�n del escenario, pero se debe tener en cuenta que esto acota la posibilidad
			de obtener un rendimiento m�s �ptimo del sistema, ya que al configurar un entorno se puede ser m�s espec�fico sobre
			las prioridades relativas de los \emph{concerns} seg�n el escenario y el estado del sistema.

			Supongamos que el sistema se encuentra bajo una carga excesiva, y los \emph{stakeholders} consideran que bajo tales
			circunstancias lo m�s prioritario es optimizar la \emph{performance} de un servicio en particular, en tal caso, de no
			especificar el entorno no podremos darle mayor peso a la \emph{performance} por sobre otros \emph{concerns}, quedando
			�nicamente la opci�n de aumentar la prioridad del escenario tergiverzando as� dicha informaci�n, ya que en realidad lo
			�ptimo era darle m�s peso a la \emph{performance} en el entorno de ``Alta Carga''.

			En definitiva, al establecer que el escenario aplica para cualquier entorno, Arco Iris estar� interpretando que el
			entorno carece de importancia, dejando as� de lado los pesos de los \emph{concerns} y distribuyendo equitativamente
			su importancia relativa, lo cual es equivalente a que no exista el concepto de \emph{concern}.


			La estructura del entorno estar� formada de la siguiente manera:
			\begin{itemize}
				\item Un conjunto de condiciones, y
				\item Un mapa $<$Concern, Peso$>$
			\end{itemize}

			Estas condiciones fueron implementadas en base a un concepto de restricci�n presente en Rainbow, que permite definir
			precondiciones de aplicabilidad de las estrategias. En Arco Iris, en cambio, utilizaremos el mismo concepto con dos
			fines, ninguno de los cuales se corresponde exactamente con la concepci�n original. Por una lado lo utilizaremos para
			conocer el entorno en que se encuentra el sistema, y por el otro ser� utilizado en la cuantificaci�n de la respuesta
			(ver \ref{sec:responseMeasure}) para decidir si el escenario se satisface o no. En dicha secci�n se explicar� la
			diferencia vital del concepto de restricci�n utilizado en Rainbow con el de Arco Iris.

			El concepto de condici�n (o restricci�n) presentado en Arco Iris permite agregar nuevos tipos de restricciones
			simplemente implementando la interface \verb@Constraint@ \footnote{Dicha implementaci�n deber� ser agregada al
			enumerado ConstraintType en Scenarios UI para que la UI reconozca el nuevo tipo}.

			En cuanto al mapa presente en la estructura del entorno, deber� contener todos los \emph{concerns} definidos para el
			sistema, y obviamente la suma de los pesos debe ser igual a uno, ya que la utilidad del mapa es determinar la
			preponderencia relativa entre los \emph{concerns}. Se detallar� c�mo se utilizan estos pesos para calcular el valor
			de un escenaio en la secci�n \ref{sec:scenarioWeight}.

			A la hora de decidir en que entorno se encuentra el sistema, no ser�a suficiente chequear las condiciones definidas para
			el entorno tomando los datos actuales del sistema, ya que deber�amos eliminar el ruido provocados por los
			\emph{outlier}, pues para que el sistema se encuentre en un determinado estado es l�gico esperar que dicho estado se
			mantenga por un tiempo m�nimo. Entonces, para definir el valor m�s real para las propiedades del sistema tomamos los
			valores nuevos con un determinado peso, configurable, que nos permite ir adecuando los valores del sistema de manera
			paulatina evitando el impacto de los valores que puedan darse por alguna ocasi�n excepcional, como por ejemplo alg�n
			cliente que experimenta un tiempo de respuesta excesivo por alg�n problema es su conexi�n o en estaci�n de trabajo.
			Para mitigar estos casos, utilizaremos el mismo mecanismo utilizado por Rainbow explicado ya en la secci�n
			\ref{sec:exponentialAverage}.

			Un valor recomendable para el factor de suavizado utilizado tanto en las pruebas de Rainbow como en las de Arco Iris
			es 0.3. Si se configura un valor muy peque�o Arco Iris tardar� en adaptarse a los cambios en el entorno del sistema,
			aunque �sto podr�a ser �til sistemas no muy din�micos o en los cuales la adaptaci�n pueda resultar muy costosa. En el
			caso contrario, o sea, al configurar un alpha muy elevado, es probable que el sistema se vea afectado por unos pocos
			valores que est�n muy por fuera del rango considerado como normal, aunque de no ser costoso lanzar una adaptaci�n o
			volverla atr�s, puede llegar a lograrse un rendimiento muy �ptimo y din�mico del sistema.


		\subsubsection{La Cuantificaci�n de la Respuesta en la Auto Reparaci�n}
			\label{sec:responseMeasure}

			La \textbf{Cuantificaci�n de la Respuesta} es quiz�s la propiedad m�s importante de un escenario: de ella surgen las
			restricciones que deben ser evaluadas para que, en caso de no cumplirse, se lance la auto reparaci�n. De hecho, para
			que un escenario se considere bien formado debe quedar claro cual es la m�trica o manifestaci�n observable de su
			respuesta que se debe satisfacer. Latencia y \emph{throughput} son ejemplos de las manifestaciones sobre las cuales
			puede predicar la cuantificaci�n de la respuesta. En pocas palabras, la cuantificaci�n ser� la m�trica seg�n la cual
			se decida la aceptaci�n de una respuesta del sistema ante un determinado est�mulo.

			Para graficar la importancia de contar con una cuantificaci�n de la respuesta precisa, supongamos que contamos con la
			siguiente definici�n:

			\begin{quote}
				``Modificar el sistema para incorporar un nuevo generador de eventos discretos''
			\end{quote}

			Esta premisa no es suficiente para medir el �xito de la incorporaci�n de la nueva funcionalidad, ya que con
			suficiente tiempo y dinero, cualquier modificaci�n es posible. En este escenario se necesitar�a una m�trica del
			siguiente tipo por ejemplo: ``Utilizando 160 horas hombre''. Esto forzar�a al arquitecto a asegurar que el sistema
			sea modificable bas�ndose en un criterio bien particular y con una m�trica aplicable.

			Sumergi�ndonos ya en la implementaci�n, cabe mencionar que Arco Iris reutiliza las restricciones gen�ricas brindadas
			por Rainbow, poni�ndolas en el contexto de un Escenario de Atributo de Calidad, proveyendo as� mayor visibilidad a los
			\emph{stakeholders} sobre estas restricciones, anteriormente s�lo conocidas por los arquitectos o encargados de
			configurar Rainbow, ya que dichas restricciones en Rainbow terminaban siendo implementadas en el modelo
			de la arquitectura descripto en el lenguaje Acme, perdiendo toda correlaci�n con su est�mulo desencadenador. Veamos
			un ejemplo extra�do de Znn de c�mo se implementan dichas restricciones al utilizar Rainbow:

			\begin{Verbatim}[gobble=4]
                Component Type ClientT {
                    Property experRespTime : float <<  default : float = 100.0; >> ;
                    rule primaryConstraint = invariant self.experRespTime <= MAX_RESPTIME;
                }
			\end{Verbatim}

			Es importante recordar que en Rainbow, una vez que se detect� que la auto reparaci�n debe ser ejecutada, se volver�n
			a ejecutar todas las restricciones dictadas en la precondici�n de las estrategias para saber si las mismas aplican
			en el estado actual del sistema, volvemos a tomar un ejemplo de Znn:

			\begin{Verbatim}[gobble=4]
                define boolean cViolation =
                    exists c : T.ClientT in M.components | c.experRespTime > M.MAX_RESPTIME;

                strategy BruteReduceResponseTime
                [ cViolation ] {
  					...
  					do some tactic
  					...
                }
			\end{Verbatim}

			En Arco Iris no ser� necesario contar con �stas �ltimas restricciones ya que el conocimiento de cuales estrategias
			son capaces de reparar una determinada condici�n ya est�n plasmado en cada escenario. En definitiva, al utilizar Arco
			Iris el usuario no tendr� m�s que configurar la Cuantificaci�n de la respuesta de cada escenario, tarea que se ve
			sumamente facilitada al utilizar UI Scenarios (ver secci�n~\ref{sec:scenariosUI} en la
			p�gina~\pageref{sec:scenariosUI}).

			Otra diferencia vital ante el uso de las mencionados restricciones entre la visi�n de Rainbow y de Arco Iris consiste
			en el momento en que se aplican. Rainbow posee un componente llamado \emph{``Architecture Evaluator''}, cuya
			responsabilidad consiste en chequear la arquitectura del modelo de a intervalos y de manera asincr�nica, siempre y
			cuando este haya sido modificado y la auto reparaci�n no se encuentre ya en ejecuci�n, esto implica verificar
			absolutamente todas las restricciones. Mientras que Arco Iris, al descubrir un cambio en el modelo de la
			arquitectura, la acci�n que tomar� consistir� en verificar los escenarios, pero no cualquier escenario, solamente los
			que coincidan en su est�mulo con el est�mulo que desencaden� el cambio en el modelo de la arquitectura, optimizando
			as� de manera determinante la cantidad de restricciones a verificar, lo que hace mucho m�s escalable la auto reparaci�n.

			Dado al cambio en el manejo de restricciones, en Arco Iris se implement� el concepto de Constraint como una
			interface, dando as� la posibilidad de extender el \emph{framework} con el tipo de Constraint que el usuario
			considere necesario. La interface planteada presentada es la siguiente:

			\begin{Verbatim}[gobble=4]
                public interface Constraint {

                    boolean holds(Number value);

                    String getFullyQualifiedPropertyName();
                }
			\end{Verbatim}

			Donde el m�todo \emph{holds} ser� el encargado de determinar si la Constraint se cumple o no, recibiendo como
			par�metro el valor de la propiedad a evaluar, mientras que el m�todo \emph{getFullyQualifiedPropertyName} retornar�
			el nombre completo cualificado de la propiedad sobre la que predica, incluyendo el sistema y el componente al que
			pertenece, un ejemplo tomado de Znn ser�a: \emph{ZNewsSys.ClientT.experRespTime}. 
			
			Para el presente trabajo se utiliz�	una �nica implementaci�n, que consiste en una relaci�n binaria de orden (igual,
			mayor, mayor o igual, menor, menor o igual) entre una propiedad de la arquitectura (e.g. \verb@server1.responseTime@)
			y un valor fijo (puede ser entero o flotante) preestablecido al configurar la cuantificaci�n de la respuesta del
			escenario. Esta restricci�n ser� configurada mediante un cuantificador, el cual provee dos opciones: sumatoria y
			promedio, los que especifican si la condici�n debe cumplirse para la suma de todas las instancias en \emph{runtime}
			de dicha propiedad o si la condici�n debe darse ``en promedio'' para todas las instancias; respectivamente.
			La implementaci�n mencionada recibe el nombre de \emph{NumericBinaryRelationalConstraint} y puede verse su c�digo
			en la secci�n \ref{sec:numericBinaryRelationalConstraintCode}.
			

	\subsection{Prioridades entre Escenarios}

		Rainbow tambi�n maneja el concepto de escenario, aunque no es equivalente al manejado por Arco Iris, el cual toma el
		concepto de Escenario de QAW de la definici�n de ATAM. Desde la perspectiva de Rainbow, el escenario es algo configurable pero
		est�tico, es decir, no ofrece la posibilidad de adaptarse a los cambios del entorno, por lo que el arquitecto deber�
		configurar el escenario\footnote{Recordar que aqu�``escenario'' no se refiere a los Escenarios de QAW utilizados por
		Arco Iris, tan s�lo se reutiliza con la finalidad de explicar el funcionamiento de Rainbow.} previo a iniciar Rainbow,
		intentando vislumbrar cuales ser�n las condiciones en las que la aplicaci�n deber� responder, para lo cual configurar�
		la importancia de cada uno de los concerns del sistema. A continuaci�n se muestra c�mo se realiza dicha configuraci�n
		en Rainbow: \begin{Verbatim}[gobble=4] 
				weights: 
					scenario 1:
						uR: 0.35
						uF: 0.4
						uC: 0.25
					scenario 2:
						uR: 0.5
						uF: 0.3
						uC: 0.2
					scenario 2b:
						uR: 0.5
						uF: 0.2
						uC: 0.3
		\end{Verbatim}
		
		Cada una de las l�neas representa el valor que se le asigna a cada \emph{concern}, en donde uR representa al Tiempo de
		Respuesta, uF a la Fidelidad de la informaci�n y uC al Costo. Notar que la sumatoria de los pesos de los distintos
		concern debe ser igual a 1 dentro de cada escenario, al igual que sucede en el \emph{Entorno} al configurar Arco Iris.
		
		Esto es lo que veremos, entre otras cosas, en el archivo de configuraci�n \emph{utilities.yml} utilizado por Rainbow.
		Aqu�, como vemos, existe m�s de una configuracion del escenario, luego se deber� seleccionar una de estas opciones en
		el archivo de configuraci�n \emph{rainbow.properties}. Estos valores ser�n luego utilizados por Rainbow a la hora de
		seleccionar una estrategia, veremos en detalle su utilizaci�n en la secci�n \ref{sec:strategySelection}.
		
		Presentado ya el concepto de escenario utilizado por Rainbow, pasaremos estas ideas a su contrapartida en Arco Iris.
		Como hemos visto anteriormente, en Arco Iris existen los conceptos de \emph{Entorno} y \emph{Escenario}, los cuales
		respetan al pie de la letra los conceptos introducidos por ATAM (ver secci�n \ref{sec:atam}). El concepto de escenario
		utilizado por Rainbow se asemeja mucho m�s al concepto de \emph{Entorno} de ATAM que al de \emph{Escenario}. La
		diferencia fundamental radica en que en Arco Iris no se requiere que el usuario configure en qu� entorno se encuentra
		el sistema, sino que el usuario debe detallar cuales son las condiciones que definen al entorno, y, en caso de
		cumplirse dichas condiciones, cuales ser�n los pesos que tomar�n cada uno de los \emph{concerns} del sistema.
		
		Dicho esto podemos introducirnos de lleno en la relaci�n de prioriades entre escenarios, sin confundir los
		conceptos manejados en ambos frameworks.

		En resumen, Rainbow permite priorizar b�sicamente por \emph{concerns}, y de manera est�tica, mientras que Arco Iris
		permite priorizar escenarios asignandoles prioridades relativas, de modo tal que al momento de escoger una estrategia
		de autoreparaci�n la estrategia seleccionada no comprometa a alguna otra funcionalidad de la aplicaci�n considerada
		m�s importante seg�n la visi�n de los \emph{stakeholders}.

		Para lograr esto, cada escenario tendr� asignada una prioridad, la cual ser� un entero mayor a cero, uno ser� la
		prioridad m�xima asignable, mientras que su valor m�ximo no estar� acotado, permitiendo  as� asignar valores muy grandes para
		escenarios de muy baja prioridad. Si bien se ofrece esta posibilidad al usuario, la misma debe manejarse con mucho
		cuidado, ya que es probable que, al configurar escenarios con prioridades tan grandes, �stos no tengan pr�cticamente
		ning�n peso a la hora de seleccionar la estrategia de autoreparaci�n a aplicar, por lo cual dichos escenarios nunca
		ser�n reparados, careciendo de sentido directamente su existencia.

		La prioridad del escenario es una propiedad fundamental para el correcto funcionamiento de Arco Iris, veremos su
		importancia al detallar c�mo Arco Iris selecciona la estrategia que aplicar� para reparar el sistema en la secci�n
		\ref{sec:strategySelection}. Por esta raz�n ser� muy importante realizar una configuraci�n a conciencia de cuales
		ser�n los escenarios prioritarios: �stos deber�n reflejar la importancia fundamental de los servicios ofrecidos por el
		sistema  seg�n las expectativas de los usuarios finales. Se recomienda realizar la asignaci�n de la prioridad de cada
		escenario como un paso m�s del Quality Attribute Workshop (QAW), para m�s detalle ver la secci�n \ref{sec:QAS}.

	\subsection{Estrategias y su Relaci�n con los Escenarios}

		Antes de sumergirnos en c�mo Arco iris utiliza las estrategias implementadas para reparar escenarios, recordemos
		brevemente su utilizaci�n en Rainbow.
		
		Las estrategias en Rainbow tienen un conjunto de precondiciones que nos indican si dicha estrategia puede ser
		utilizada para reparar el sistema en un determinado momento. Esto implica que, en cuanto Rainbow detecta un problema
		en el sistema, deba recorrer todas las estrategias existentes para ver si son aplicables para las condiciones
		actuales, o sea que Rainbow no tiene manera de determinar a priori la utilidad de una estrategia en un momento dado.
		Por otro lado, la soluci�n est� atada a la detecci�n del problema, entonces, al no poder modularizar en detecci�n del
		problema y soluci�n, se deben repetir las precondiciones por cada soluci�n (estrategia) implementada.
		
		En Arco Iris se propone desacoplar la detecci�n del problema de la soluci�n, y para lograr esto utilizaremos los
		escenarios. All� se definir�n las condiciones de detecci�n del problema, y se referenciar�n las posibles estrategias
		de reparaci�n a ser consideradas para su ejecuci�n, en el caso de que el escenario en cuesti�n se vea comprometido,
		quedando as� las estrategias exentas de conocer cuales son las circunstancias en las que su ejecuci�n tiene sentido.
		Esto permite al arquitecto tener un mayor control sobre las estrategias a ejecutar en determinadas condiciones, ya que
		al situarse en un escenario concreto, �l sabr� cuales ser�n las soluciones m�s adecuadas bas�ndose en el entorno del
		sistema y en la prioridad del escenario actual. Obviamente, Arco Iris tambi�n ofrece la posibilidad de utilizar el
		comportamiento brindado por Rainbow, esto se logra simplemente indicando la opci�n que representa a todas las
		estrategias existentes (para m�s detalle ver \ref{sec:strategySelectionUI}) cuando se configura el escenario.

		Una ventaja derivada del agregado del concepto de \emph{Escenario} es que ahora los problemas y sus posibles
		soluciones (i.e. estrategias de reparaci�n), pueden ser visibles a los usuarios y stakeholders de la aplicaci�n. Como
		ya suced�a en Rainbow, las estrategias poseen la informaci�n necesaria para permitir simular su aplicaci�n y estimar
		en que condiciones quedar�a el sistema luego de haber sido aplicadas (esta informaci�n ser� utilizada para la
		estimaci�n de la nueva ``utilidad del sistema\footnote{Para m�s detalle acerca del concepto Utilidad del Sistema ver
		secci�n \ref{sec:systemUtility}}'').
		
		En Arco Iris, para poder escribir estrategias que involucren varias t�cticas, el usuario contar� con un mecanismo que
		le permitir� verificar si los escenarios de un determinado \emph{concern} y que han sido marcados para reparar, a�n
		siguen sin cumplirse. En base a esta informaci�n la estrategia podr� decidir c�mo continuar su ejecuci�n. A
		continuaci�n se muestra un ejemplo de su utilizaci�n:
		
		\begin{Verbatim}[gobble=4] 
				define boolean RESP_TIME_STILL_BROKEN = 
						AdaptationManagerWithScenarios.isConcernStillBroken("RESPONSE_TIME");
				
				/*
				 * This Strategy will drop fidelity once, observe, then drop again if necessary.
				 */
				strategy BruteReduceResponseTime
				[ styleApplies ] {
				  t0: (true) -> lowerFidelity(2, 100) @[5000 /*ms*/] {
				    t1: (!RESP_TIME_STILL_BROKEN) -> done;
				    t2: (RESP_TIME_STILL_BROKEN) -> lowerFidelity(2, 100) @[8000 /*ms*/] {
				      t2a: (!RESP_TIME_STILL_BROKEN) -> done;
				      t2b: (default) -> TNULL;  // in this case, we have no more steps to take
				    }
				  }
				}	
			\end{Verbatim}

		C�mo se puede observar en el c�digo, para que esto funcione es necesario tener definida la funci�n
		\emph{isConcernStillBroken} en la clase \emph{AdaptationManagerWithScenarios}, a continuaci�n vemos la implementaci�n,
		notar que el usuario solamente deber� indicar cual es el \emph{concern} de su inter�s:

		\begin{Verbatim}[gobble=4] 
				public static boolean isConcernStillBroken(String concernString) {
					Concern concern = Concern.valueOf(concernString);
					doLog(Level.INFO, "Is Concern " + concern + " Still Broken?");
		
					boolean result = false;
					for (SelfHealingScenario scenario : currentBrokenScenarios) {
						if (scenario.getConcern().equals(concern) &&
								scenarioBrokenDetector4CurrentSystemState.isBroken(scenario)) {
							result = true;
							break;
						}
					}
					doLog(Level.INFO, "Concern " + concern + (result == true ? 
							" Still Broken!" : " Not Broken Anymore!!!"));
					return result;
				}		
		\end{Verbatim}

	\subsection{Detecci�n: cuando activar el mecanismo de auto reparaci�n}

		Con el nuevo enfoque presentado en este trabajo, d�nde el \emph{Escenario} es el concepto central, es necesario
		establecer cambios en la l�gica aplicada por el \emph{framework} a la hora de decidir en qu� momento es necesario
		intentar auto reparar el sistema (i.e. evaluar sus restricciones o invariantes).

		En Rainbow, las restricciones del sistema se encuentran embebidas en la descripci�n arquitect�nica de sus componentes,
		m�s precisamente en el modelo de la arquitectura, el cual se describe utilizando el lenguaje de descripci�n de
		arquitectura ACME (\ref{sec:acme}). Por ejemplo, para determinar que el tiempo de respuesta no debe exceder un
		umbral determinado es necesario definir un invariante en el componente que posee dicha propiedad, como se puede
		observar en el siguiente ejemplo tomado de la arquitectura de Znn:
		
		\begin{Verbatim}[gobble=4] 
				Component Type ClientT extends ArchElementT with {
				
					Property experRespTime : float <<  default : float = 100.0; >> ;
				
					rule primaryConstraint = invariant self.experRespTime <= MAX_RESPTIME;
				}
				
				Property MAX_RESPTIME : float = 1000.0;
		\end{Verbatim}

		En Arco Iris ya no ser� necesario definir estas \emph{constraints} en la arquitectura, desacoplando as� la arquitectura
		del sistema de la definici�n de condiciones a evaluar para lanzar la auto reparaci�n. Ahora las restricciones estar�n
		presentes en la \emph{Cuantificaci�n de la Respuesta}(\ref{sec:responseMeasure}) de cada escenario. Al cambiar el modo
		de difinir las constraints, se logra dar mayor visibilidad a los \emph{stakeholders} del sistema. Arco Iris tambi�n
		a�ade la posibilidad de definir las restricciones de manera visual utilizando Scenarios UI (\ref{sec:scenariosUI}), para
		m�s detalle ver la secci�n \ref{sec:scenariosUI_constraints}. Todo esto hace que los \emph{stakeholders} puedan  
		acceder de manera sencilla a las restricciones a las cuales debe acatarse el sistema, pudiendo editarlas sin la  
		necesidad de poseer conocientos sobre las tecnolog�as utilizadas por el sistema o por el \emph{framework} de auto  
		reparaci�n.

		
		Para determinar si el sistema necesita auto repararse, Rainbow utilizar� una funcionalidad ofrecida por ACME, que
		permite evaluar las restricciones descriptas en el modelado de la arquitectura, para esto ACME provee los \emph{Types
		Checkers}. A continuaci�n podemos observar c�mo Rainbow se sirve de esta utilidad para saber si debe lanzar la auto
		reparaci�n:
		
		\begin{Verbatim}[gobble=4]
				public void evaluateConstraints () {
					IAcmeTypeChecker typechecker = m_acmeEnv.getTypeChecker();
					if (typechecker instanceof SynchronousTypeChecker) {
						SynchronousTypeChecker synchChecker = (SynchronousTypeChecker) typechecker;
						synchChecker.typecheckAllModelsNow();
						m_constraintViolated = !synchChecker.typechecks(m_acmeSys);
						if (m_constraintViolated) {
							Set<?> errors = m_acmeEnv.getAllRegisteredErrors();
							Oracle.instance().writeEvaluatorPanel(m_logger, errors.toString());
						}
					}
				}
		\end{Verbatim}

		Recordemos que en Arco Iris hemos desacoplado el modelado de la arquitectura de las restricciones que deben
		satisfacerse, por lo cual no podremos utilizar la funcionalidad provista por ACME. Para suplir esto, y tambi�n para
		poder permitir incluir toda la informaci�n necesaria en los escenarios y que la misma pueda ser visualizada y editada
		por todos los \emph{stakeholders} involucrados, hemos desarrollado nuestras propias implementaci�n de las
		restricciones, las cuales permiten expresar las mismas \emph{constraints} definidas en ACME, y en caso de ser
		necesario agregar un nuevo tipo de restricci�n permite f�cilmente agregar una nueva implementaci�n. Estas
		\emph{constraints} ser�n vitales a la hora de activar el mecanismo de auto reparaci�n.
		
		Al iniciar, Arco Iris leer� todos los escenarios definidos, y armar� un mapa que le permitir� optimizar las
		verificaciones recurrentes de los escenarios ante la llegada de cada est�mulo. El mapa almacena todos los escenarios
		correspondientes a cada est�mulo, as�, ante la invocaci�n de un est�mulo en el sistema, Arco Iris solo deber�
		verificar que se cumplan los escenarios relacionados con dicho est�mulo, acotando as� de manera sustancial la cantidad
		de chequeos a realizar. Recordar que en Rainbow, ante cualquier modificaci�n en el estado de la arquitectura, es
		necesario verificar todas las restricciones definidas para la auto reparaci�n.
		
		\todo{Ampliar gigantescamente la siguiente explicaci�n, con gr�ficos, mas detalle, etc. y dejando en claro como
		funcionaba en Rainbow y que agregamos nosotros}
	
		\todo{continuar desde aca!!!}


		En el caso de Arco Iris, ante la invocaci�n de un est�mulo en el sistema en ejecuci�n, el \textbf{Monitor} del sistema
		informa de esta situaci�n al \textbf{Int�rprete}, que a su vez actualiza las propiedades del modelo de la arquitectura (recordemos: en
		ACME) e invoca finalmente al \textbf{Evaluador de Restricciones} para que busque aquellos escenarios que posean al
		est�mulo ejecutado y que est�n definidos para el \textbf{Entorno} de ejecuci�n actual y que las restricciones
		asociadas a sus \textbf{Cuantificaciones de Respuesta} no se cumplan. Aquellos escenarios que cumplan dichas
		condiciones ser�n aquellos a los cuales el Manejador de Reparaciones intentar� reparar.

	\subsection{Selecci�n de estrategia a aplicar}
	\label{sec:strategySelection}

		\todo{Esta subsecci�n probablemente necesite ser reescrita en su totalidad, explicando en una subsubsection el
		concepto de utilidad del sistema, explicar primero como se calcula el peso de un escenario}

		\subsubsection{Peso de un escenario}
			\label{sec:scenarioWeight}

		\subsubsection{Utilidad del Sistema}
		\label{sec:systemUtility}

			\todo{Explicar Utilidad del sistema}

		El m�dulo denominado como ``Adaptation Manager'' es uno de los m�dulos que han sufrido m�s modificaciones como parte
		de esta extensi�n a Rainbow. El mismo ha sido extendido \todo{�c�mo?} para que utilice el conocimiento plasmado en los
		escenarios para que determine la mejor (en un sentido heur�stico) estrategia de reparaci�n a aplicar, teniendo en
		cuenta:
		\begin{itemize}
			\item el entorno de ejecuci�n en que se encuentra la aplicaci�n,
			\item el atributo de calidad asociado al escenario, \todo{Hablamos por todos lados de QA pero en realidad es
			Concern!} y
			\item las prioridades relativas de los escenarios.
		\end{itemize}
		El objetivo final es el de intentar reparar el inconveniente hallado pero siempre restricto a evitar perjudicar
		alg�n otro escenario de mayor prioridad \todo{Esto es un efecto colateral de la utility function pero...podemos
		garantizar que esto es asi siempre? Creo que esto necesita ser reescrito} y mediante el uso de heur�sticas, poder
		aproximar la mejor estrategia de reparaci�n a llevar a cabo de modo que la \emph{utilidad del sistema} se maximize.

	
		\todo{REVISAR}Cabe aclarar que la ``utilidad del sistema'' es una funci�n que se calcula a partir del cumplimiento o
		no de los escenarios requeridos por los stakeholders, teniendo en cuenta sus prioridades relativas y la prioridad de
		la estrategia simulada.

	\subsection{Modelo de Arco Iris}

		\todo{La idea de esta secci�n es la de mostrar el modelo de la extensi�n de una manera consolidada. No tiene sentido
		ponerlo al principio de la secci�n cuando muchos de los conceptos que se ven todav�a no fueron explicados.}

		\begin{center}
			\includegraphics[width=1.00\textwidth]{images/Arco_Iris_Model.png}
		\end{center}

		\todo{Explicar un poco sobre el diagrama anterior.}
	
	\newpage
	
	\section{Interfaz Gr�fica: Scenarios UI}
	\label{sec:scenariosUI}

	\subsection{Motivaciones para una ``GUI''}
		A�n considerando las mejoras provistas por la extensi�n a Rainbow propuesta en el presente trabajo, uno de los
		escollos m�s notorios para poder utilizar de manera amena, �gil y productiva a ``Arco Iris'' es, sin dudas, la
		ausencia de una interfaz visual para que los \emph{stakeholders} y arquitectos de la aplicaci�n a adaptar puedan
		crear, editar y eliminar escenarios, entornos, artifacts y otros conceptos introducidos en ``Arco Iris''; as� tambi�n
		como otros conceptos ya existentes en ``Rainbow''.
		
		% TODO CONTINUAR! A fin de supEn el presente apartado nos concentraremos en repasar 
	
	
		\todo{Chamuyo de que nos llev� a hacer esto, incluir pq es en ingles}
		
		Se propone el desarrollo de una interfaz de usuario gr�fica (GUI, de sus siglas en ingl�s: Graphical User Interface)
		para que los distintos \emph{stakeholders}, incluyendo usuarios y arquitectos, puedan colaborar creando y editando
		escenarios que luego ser�n importados y utilizados por Rainbow.

	\subsection{Conceptos b�sicos de uso de la herramienta}
		\todo{Explicar que siempre hay por debajo un SHC, que se va actualizando siempre, que la estructura es
		siempre ver todo actualizado en las consultas, etc\ldots}
		
		\subsubsection{Constraints}
			\todo{Mencionar que es extensible el composite pero que por ahora solo implementamos la numeric relational binary
			constraint}

	\subsection{Administraci�n de Artifacts}

	\subsection{Administraci�n de Entornos}
	
		\subsubsection{Pesos relativos de Concerns}
		
		\subsubsection{El entorno ``ANY''}

	\subsection{Administraci�n de Escenarios}

		\subsubsection{Selecci�n de Entornos}
		
		\subsubsection{Selecci�n de Estrategias de Reparaci�n}

	\subsection{Puntos de extensi�n}
	
	\newpage
	
	\section{Casos Pr�cticos}
\label{sec:casosPracticos}

	En esta secci�n se presentan y analizan algunos casos de prueba concretos de uso de Arco Iris a fin de evaluar su
	comportamiento. Se utilizar� el modo simulaci�n provisto por Rainbow (ver secci�n \ref{sec:modosEjecucion}) para
	adaptar utilizando Arco Iris al sistema ficticio \textbf{Znn}, reutilizando los componentes de simulaci�n creados para
	la tesis de doctorado d�nde dicho sistema es presentado. (ver secci�n \ref{sec:znn})

	La simulaci�n permite configurar la variaci�n de los valores de ciertas propiedades de los componentes de la
	arquitectura, a fin de simular diversas situaciones de carga en el sistema ficticio. Por ejemplo, se podr�a especificar
	que a los 10 segundos de haber comenzado la simulaci�n, el ancho de banda de un servidor en particular disminuya y por
	otro lado que en ese mismo instante, la frecuencia de arribo de \emph{requests} del usuario suba en una determinada
	proporci�n. Este cambio en el entorno de ejecuci�n del sistema simulado, por ejemplo, permitir�a evaluar c�mo se
	comporta Arco Iris en un contexto de \textbf{alta carga}.
	
	Para todos los casos de prueba presentados en esta secci�n, se utilizar� una simulaci�n que dura 60 segundos.
	
	\subsection{Arquitectura del Sistema Simulado}
	
		Los componentes principales de la arquitectura de Znn son clientes y servidores. �stos no se conectan directamente
		entre s�, sino que lo hacen por medio de un \emph{proxy}, al cual arriban todas las peticiones de los clientes, y es
		�l quien conoce todos los servidores disponibles y distribuye el trabajo entre ellos. A continuaci�n se muestra un diagrama
		de la arquitectura del sistema, extra�da utilizando la herramienta AcmeStudio. (En el ap�ndice
		\ref{sec:arquitecturaZNN} se puede observar el c�digo fuente Acme de dicha arquitectura)

		\begin{figure}[ht]
			\centering
				\includegraphics[width=\textwidth]{images/znnArchitecture.png}
			\caption{Arquitectura de Znn vista en Acme Studio}
			\label{fig:znnArchitecture}
		\end{figure}

		En la arquitectura de Znn est�n definidos tres \emph{concerns} aunque en el presente trabajo, a saber:

		\begin{itemize}
			\item \textbf{Tiempo de Respuesta}: tiempo de respuesta promedio experimentado por el usuario de Znn, y
			\item \textbf{Costo}: el cual, refleja la cantidad de servidores prestando servicio con los que Znn cuenta en un
			determinado momento.
			\item \textbf{Fidelidad del Contenido}: trata sobre la \textbf{calidad} del contenido ofrecido por un servidor de
			Znn. A fin de mejorar el \emph{throughput}, un servidor podr�a servir contenido en un modo \emph{full} (texto,
			videos, audio, animaciones, etc.), en un modo s�lo texto o bien en una combinaci�n de estos dos.
		\end{itemize}

		Para las pruebas realizadas en este trabajo, s�lo se utilizar�n las siguientes t�cticas, con el fin de modificar el
		comportamiento del sistema en ejecuci�n:
		\begin{itemize}
			\item Dar de alta un servidor
			\item Dar de baja un servidor
			\item Disminu�r la fidelidad del contenido provisto por un servidor.
		\end{itemize}

		Para que estas t�cticas puedan ejecutarse, Znn implementa los correspondientes \emph{effectors}, quienes ser�n
		los responsables de efectuar los cambios propiamente dichos sobre el sistema en \emph{runtime}. Los \emph{effectors}
		ser�n invocados desde las t�cticas, cuyas implementaciones en Znn pueden verse en el ap�ndice \ref{sec:tacticasZNN}.
	
	\subsection{Configuraci�n B�sica para Casos de Prueba}
	\label{sec:configBasicaCasosPrueba}
	
		Como parte de la configuraci�n utilizada para las pruebas que ser�n presentadas en esta secci�n, es necesario definir
		los siguientes conceptos:

		\begin{itemize}
		  \item Entorno de carga normal.
		  \item Entorno de alta carga.
		  \item Escenario de tiempo de respuesta experimentado por el usuario.
		  \item Escenario de costo de servidores del sistema.
		  \item Estrategias asociadas a cada escenario, las cuales ser�n presentadas a medida que su utilizaci�n sea
		  requerida.
		\end{itemize}
		
		\begin{figure}[H]
			\centering
				\includegraphics[scale=0.6]{images/Environment_Normal.png}
			\caption{Entorno de ejecuci�n de carga normal}
			\label{fig:Environment_Normal}
		\end{figure}

		Cabe destacar que para el entorno de carga normal, la configuraci�n por defecto de los pesos para cada uno de los
		\emph{concerns} del sistema se encuentra equidistribuida. Esta decisi�n convierte a las prioridades entre
		escenarios, cuando todos ellos pertenecen al entorno de carga normal, en el �nico factor influyente en la selecci�n de
		una estrategia candidata para reparar el sistema (ver algoritmo de selecci�n de estrategias en la secci�n
		\ref{sec:arcoIrisStrategyScoring}), simplificando as� los c�lculos, as� como tambi�n la comprensibilidad de los
		resultados presentados.

		\begin{figure}[ht]
			\centering
				\includegraphics[scale=0.6]{images/Environment_HighLoad.png}
			\caption{Entorno de ejecuci�n de alta carga}
			\label{fig:Environment_HighLoad}
		\end{figure}

		En la figura \ref{fig:Environment_HighLoad} se puede observar la configuraci�n del entorno de alta carga, cuyos pesos
		no se encuentran equidistribu�dos, enfatizando as� la importancia del tiempo de respuesta frente a los restantes
		\emph{concerns} definidos en el sistema.
		
		\begin{figure}[H]
			\centering
				\includegraphics[width=0.9\textwidth]{images/scenario_expRespTime.png}
			\caption{Escenario de tiempo de respuesta experimentado por el usuario}
			\label{fig:scenario_expRespTime}
		\end{figure}

		\begin{figure}[H]
			\centering
				\includegraphics[width=0.9\textwidth]{images/scenario_cost.png}
			\caption{Escenario de costo de servidores del sistema}
			\label{fig:scenario_cost}
		\end{figure}

		En el ap�ndice \ref{sec:scenarioExpRespTimeXML} se muestra a modo de ejemplo la representaci�n en XML del escenario de
		tiempo de respuesta definido anteriormente.

		Con respecto a la configuraci�n por defecto de los escenarios aqu� definidos, notar lo siguiente:

		\begin{itemize}
			\item el escenario de tiempo de respuesta posee mayor prioridad que el de costo.
			\item ambos escenarios aplican para cualquier entorno en que se encuentre el sistema en ejecuci�n (ver definici�n del
			pseudo entorno ``ANY'' en la secci�n \ref{sec:environment});
			\item ambos escenarios no poseen estrategias de reparaci�n configuradas.
		\end{itemize}		
		
		La configuraci�n b�sica expuesta hasta aqu� ser� la utilizada por todos los casos de prueba a desarrollar en el
		presente trabajo. Al avanzar con las pruebas, y de acuerdo a las necesidades particulares de configuraci�n de cada
		una, ser� necesario efectuar algunos ajustes menores que impactar�n sobre los valores de los siguientes atributos:

		\begin{itemize}
		  \item Prioridad de cada escenario.
		  \item Estrategias asociadas a cada escenario.
		  \item Pesos de los \emph{concerns} para cada entorno.
		\end{itemize}

	\subsection{Caso 0: Comportamiento del Sistema sin Auto Reparaci�n}
		
		Para comenzar, se presenta el comportamiento del sistema de no existir escenarios ni estrategias, esto
		sentar� las bases para luego poder comparar y evaluar el comportamiento de Arco Iris a medida que se
		vayan agregando escenarios y/o estrategias en los siguientes casos de prueba a considerar.

		En la figura \ref{fig:Caso0} se puede observar el comportamiento del sistema sin escenarios, es decir, sin mecanismo
		de auto reparaci�n alguno.

		\begin{figure}[ht]
			\begin{center}
				\subfigure[Tiempo de Respuesta]{\includegraphics[width=\textwidth]{images/testcase0_expRespTime.png}}
				\subfigure[Costo de
				Servidores]{\label{fig:testcase0_cost}\includegraphics[width=\textwidth]{images/testcase0_cost.png}}
			\end{center}
			\caption{Comportamiento del sistema sin escenarios}
			\label{fig:Caso0}
		\end{figure}

		Como se puede observar, la simulaci�n ha sido configurada expl�citamente para que Znn se comporte de la siguiente
		manera: el tiempo de respuesta crece hasta superar los 600 ms., manteni�ndose all� hasta 30 segundos despu�s de haber
		comenzado la simulaci�n, para luego ir bajando paulatinamente hasta estacionarse cerca de los 400 ms. Notar que el
		costo de los servidores se mantiene inmutable frente a los cambios en el tiempo de respuesta, es decir que el sistema,
		de no mediar un usuario administrador o un \emph{framework} de auto reparaci�n como Arco Iris, trabaja siempre con un
		�nico servidor. Es importante tener en cuenta que la merma en el tiempo de respuesta no se debe a ninguna acci�n
		propia de la auto reparaci�n, sino a cambios en el ambiente, externos al sistema, como por ejemplo el ancho de banda
		de la conexi�n de cada uno de sus clientes.

	\subsection{Caso 1: Comportamiento con un Escenario, Sin Estrategias}

		Para el presente caso de prueba se utiliza el escenario de tiempo de respuesta definido anteriormente (ver figura
		\ref{fig:scenario_expRespTime}), el cual determina un umbral m�ximo aceptado de 600 ms. para el tiempo de respuesta
		experimentado por el usuario. 
		
		El objetivo de esta prueba es visualizar c�mo, al no haberse definido a�n ninguna estrategia, Arco Iris detectar� que
		existe un escenario que no se satisface aunque no efectuar� reparaci�n alguna sobre el sistema.

		Dado que Arco Iris no ejecuta estrategia alguna, el costo de servidores no se ver� modificado, manteni�ndose constante
		en 1, tal como se ha visto en la figura \ref{fig:testcase0_cost}.
		
		Por otro lado, en la figura \ref{fig:testcase1_expRespTime} se puede observar que Arco Iris detecta que, a partir de
		cierto instante, el tiempo de respuesta alcanza y supera el umbral predefinido en la cuantificaci�n de la respuesta
		del �nico escenario del sistema.
		
		\begin{figure}[ht]
			\centering
				\includegraphics[width=1.00\textwidth]{images/testcase1_expRespTime.png}
			\caption{El umbral definido para el tiempo de respuesta es superado}
			\label{fig:testcase1_expRespTime}
		\end{figure}

	\subsection{Caso 2: Comportamiento con un Escenario y una Estrategia}
	\label{sec:testCase2}

		En el presente caso se intenta reflejar c�mo Arco Iris repara el sistema al encontrar una estrategia candidata adecuada
		para el escenario de tiempo de respuesta anteriormente presentado. Para tal fin, se define una estrategia que consiste
		simplemente en agregar un servidor, siempre y cuando existan servidores disponibles. La estrategia, definida en Stitch,
		posee la siguiente l�gica:

		\begin{figure}[ht]
			\centering
			\begin{Verbatim}[gobble=4]
				strategy EnlistServerResponseTime {
				  t0: (true) -> enlistServers(1) @[5000 /*ms*/] {
				    t1: (!RESP_TIME_STILL_BROKEN) -> done;
				    t2: (default) -> TNULL;
				  }
				}
			\end{Verbatim}
			\caption{Estrategia que agrega un servidor m�s al sistema}
			\label{fig:EnlistServerResponseTime}
		\end{figure}

		Al agregar esta estrategia al escenario, se observa en la figura \ref{fig:testcase2_expRespTime} que el tiempo de
		respuesta experimentado por el usuario mejora (i.e. desciende) r�pidamente. De manera simult�nea a esta mejora, el
		costo de servidores aumenta a 2, producto de la ejecuci�n de la estrategia. Esto puede observarse en la figura
		\ref{fig:testcase2_cost}.

		\begin{figure}[ht]
			\begin{center}
				\subfigure[Mejora en el tiempo de respuesta debido a la ejecuci�n de una
				estrategia]{\label{fig:testcase2_expRespTime}\includegraphics[width=1.00\textwidth]{images/testcase2_expRespTime.png}}
				\subfigure[Impacto de la estrategia sobre el costo de
				servidores]{\label{fig:testcase2_cost}\includegraphics[width=1.00\textwidth]{images/testcase2_cost.png}}
			\end{center}
			\caption{Impacto del agregado de una estrategia}
			\label{fig:Caso2}
		\end{figure}

		En resumen, se ha visto hasta aqu� el comportamiento del sistema en las siguientes circunstancias:
		\begin{enumerate}
			\item no existe informaci�n alguna sobre auto reparaci�n.
			\item se ha definido un escenario pero sin estrategias que lo puedan reparar.
			\item se ha definido un escenario con una estrategia asociada.
		\end{enumerate}

		Antes de proseguir con casos de prueba m�s complejos, cabe mencionar que los \emph{logs} generados por Arco Iris
		ofrecen la informaci�n necesaria para analizar en detalle los casos de pruebas presentados en este informe. Dada la
		extensi�n de dichos archivos, es inviable mostrarlos todos para cada caso de prueba, por lo cual, a modo de ejemplo,
		en el ap�ndice \ref{sec:logCasoPruebaArcoIris} se presenta un extracto del \emph{log} generado por Arco Iris para el
		caso que se acaba de desarrollar en esta secci�n.

	\subsection{Caso 3: \emph{Tradeoff} entre Estrategias}
	
		El presente caso intenta mostrar c�mo Arco Iris escoge, entre varias estrategias candidatas para un mismo escenario,
		aquella que maximiza la utilidad del sistema.
		
		En particular, este caso presenta dos escenarios: uno relacionado con el costo de servidores, sin estrategias de
		reparaci�n definidas; y otro cuyo \emph{concern} es el tiempo de respuesta, configurado con la estrategia
		\verb@EnlistServerResponseTime@ antes definida y la introducci�n de una nueva estrategia:
		
		\begin{Verbatim}[gobble=3]
			strategy LowerFidelityReduceResponseTime {
			  t0: (true) -> lowerFidelity(2, 100) @[5000 /*ms*/] {
			    t1: (!RESP_TIME_STILL_BROKEN) -> done;
			    t2: (RESP_TIME_STILL_BROKEN) -> lowerFidelity(2, 100) @[8000 /*ms*/] {
			      t2a: (!RESP_TIME_STILL_BROKEN) -> done;
			      t2b: (default) -> TNULL;  // in this case, we have no more steps to take
			    }
			  }
			}
		\end{Verbatim}

		En concreto, en este caso de prueba se puede observar de qu� manera (mediante el uso del concepto de
		\textbf{Utilidad del Sistema}), Arco Iris - dentro de las estrategias que reparan al escenario en cuesti�n - otorga
		m�s valor a aquellas estrategias que no ``rompen'' otro escenario, es decir, las que dejan al sistema en una situaci�n
		m�s estable.
		
		Es de notar que, al ``competir'' estrategias relacionadas con el mismo \emph{concern} y reparando ellas al mismo
		escenario, las prioridades configuradas para cada escenario, en este caso, carecen de importancia.
		
		Cabe mencionar que, para que el escenario de costo deje de cumplirse, debe ser un escenario que aplique al entorno
		actual del sistema. Esta condici�n se satisface trivialmente, considerando que el escenario de costo fue definido para
		aplicar en cualquier entorno.
		
		En el siguiente extracto de \emph{log} se puede observar c�mo Arco Iris considera las estrategias mencionadas,
		escogiendo a \verb@LowerFidelityReduceResponseTime@ por sobre\\
		\verb@EnlistServerResponseTime@, ya que si bien ambas reparan al escenario de tiempo de respuesta, la �ltima rompe al
		escenario de costo de servidores:
		
		\begin{Verbatim}[gobble=3]
			...
			Evaluating strategy EnlistServerResponseTime...
			Scoring EnlistServerResponseTime...
			  Server Cost Scenario broken after simulation for Server Cost ([ESum] 2.0)? true
			  Experienced Response Time Scenario broken after simulation for Response time ([EAvg] 457.81)? false
			  Score for strategy EnlistServerResponseTime: 0.333
			  Current best strategy EnlistServerResponseTime
			  Evaluating strategy LowerFidelityReduceResponseTime...
			Scoring LowerFidelityReduceResponseTime...
			  Server Cost Scenario broken after simulation for Server Cost ([ESum] 1.0)? false
			  Experienced Response Time Scenario broken after simulation for Response time ([EAvg] 481.81)? false
			  Score for strategy LowerFidelityReduceResponseTime: 0.49
			  Current best strategy: LowerFidelityReduceResponseTime
			Selected strategy!: LowerFidelityReduceResponseTime
			EXECUTING STRATEGY LowerFidelityReduceResponseTime...
			...
		\end{Verbatim}
		
		Por �ltimo, en la figura \ref{fig:Caso3} se pueden observar las variaciones de los \emph{concerns} tiempo de
		respuesta, costo de servidores y fidelidad, para este caso de prueba. 
		
		\begin{figure}[H]
			\begin{center}
				\subfigure[Reparaci�n del tiempo de respuesta usando la mejor estrategia]
						  {\label{fig:testcase3_expRespTime}
						  \includegraphics[width=0.99\textwidth]{images/testcase3_expRespTime.png}}
				\subfigure[El costo de servidores se mantiene intacto]
						  {\label{fig:testcase3_cost}
						  \includegraphics[width=0.99\textwidth]{images/testcase3_cost.png}}
				\subfigure[Desciende la fidelidad de la informaci�n]
						  {\label{fig:testcase3_fidelity}
						  \includegraphics[width=0.99\textwidth]{images/testcase3_fidelity.png}}
			\end{center}
			\caption{Variaciones de los tres \emph{concerns} involucrados}
			\label{fig:Caso3}
		\end{figure}
		
	\subsection{Caso 4: \emph{Tradeoff} entre Escenarios seg�n Prioridades}

		El objetivo de esta prueba consiste en evaluar el comportamiento de Arco Iris ante la existencia de escenarios con
		distintas prioridades.

		Para el presente caso de prueba, se toma como base la configuraci�n del caso anterior con las siguientes
		modificaciones:
		\begin{itemize}
			\item El escenario de tiempo de respuesta contar� ahora solamente con la estrategia\\
			\verb@EnlistServerResponseTime@,
			\item Al escenario de costo se le agrega una estrategia de reparaci�n, cuya l�gica puede verse a continuaci�n:

			\begin{Verbatim}[gobble=4]
				strategy ReduceOverallCost {
				  t0: (true) -> dischargeServers(1) @[2000 /*ms*/] {
				    t1: (!COST_STILL_BROKEN) -> done;
				    t3: (default) -> TNULL;
				  }
				}
			\end{Verbatim}
		\end{itemize}

		Como ya se ha mencionado en la introducci�n de la presente secci�n (ver secci�n \ref{sec:configBasicaCasosPrueba}), el
		escenario de tiempo de respuesta es m�s prioritario que el de costo. Esta configuraci�n es crucial en este caso de
		prueba, ya que determinar� el rumbo de la auto reparaci�n llevada a cabo por Arco Iris.

		En las figuras \ref{fig:testcase4_expRespTime} y \ref{fig:testcase4_cost} se puede observar el comportamiento del
		tiempo de respuesta y del costo, respectivamente, para la configuraci�n actual.

		\begin{figure}[ht]
			\begin{center}
				\subfigure[Uso de prioridades favoreciendo al escenario de eficiencia por sobre el de costo]
						  {\label{fig:testcase4_expRespTime}\includegraphics[width=0.95\textwidth]{images/testcase4_expRespTime.png}}
				\subfigure[Reducci�n del costo como consecuencia de un cambio en el entorno]
						  {\label{fig:testcase4_cost}\includegraphics[width=0.95\textwidth]{images/testcase4_cost.png}}
			\end{center}
			\caption{Comportamiento del sistema respetando prioridades entre escenarios}
			\label{fig:Caso4}
		\end{figure}

		Tal cual se ha visto en los casos b�sicos anteriores, el escenario de tiempo de respuesta es el primero en dejar de
		cumplirse. Considerando que s�lo se ha configurado una �nica estrategia de reparaci�n para dicho escenario, la misma
		es ejecutada exitosamente, ya que cerca de los 9 segundos, el tiempo de respuesta vuelve a ubicarse en valores aceptables.
		
		Ahora bien, la estrategia ejecutada consiste ni m�s ni menos que en agregar un servidor m�s al sistema, con lo cual se
		observa en la figura \ref{fig:testcase4_cost} de qu� manera, a partir de los 5 segundos, el escenario relacionado con
		el costo de servidores deja de cumplirse, puesto que la cantidad m�xima de servidores all� especificados es 1.
		
		Arco Iris debe decidir si repara o no a este nuevo escenario que se ha ``roto''. Es aqu� d�nde las prioridades entre
		escenarios juegan un papel determinante: dado que el escenario relaciondo con el tiempo de respuesta es m�s
		prioritario que aqu�l relacionado con el costo de servidores, Arco Iris decide no efectuar reparaci�n alguna sobre
		este �ltimo, ya que detecta (mediante la heur�stica explicada en la secci�n \ref{sec:arcoIrisStrategyScoring}) que el
		arreglar el escenario de costo, potencialmente llevar�a a ``romper'' el escenario de tiempo de respuesta, que es m�s
		prioritario. En otras palabras, la auto reparaci�n provista por Arco Iris no intentar� reparar el escenario de costo
		de servidores mientras que la utilidad del sistema en el estado actual sea mayor a la prevista en caso de repararlo.

		En la figura \ref{fig:testcase4_cost}, aproximadamente a partir de los 33 segundos, se puede observar c�mo Arco Iris
		decide desactivar un servidor en el preciso momento en que el entorno de ejecuci�n de la aplicaci�n cambia y el
		tiempo de respuesta mejora por razones ajenas a la auto reparaci�n. Aprovechando esto, Arco Iris logra satisfacer as�
		al escenario que estaba sin cumplirse, sin perjudicar al otro escenario (m�s prioritario) que se estaba cumpliendo
		hasta ese momento.

		Finalmente, se arriba a un estado de estabilidad d�nde ambos escenarios se satisfacen simult�neamente. Si bien el tiempo
		de respuesta sufre un peque�o detrimento al trabajar el sistema con un servidor menos, los valores de las
		propiedades relacionadas con los \emph{concerns} de inter�s, siguen siendo lo suficientemente aceptables como para no
		violar ninguna de las restricciones definidas en la cuantificaci�n de la respuesta de los escenarios aqu�
		configurados.
		
		En resumen, se ha mostrado por un lado la potencia y conveniencia de agregar el concepto de prioridad entre escenarios
		como un elemento de relevancia para configurar al \emph{framework} y por otro, c�mo distintos escenarios con
		distinta prioridad pueden convivir en Arco Iris, tomando �ste decisiones inteligentemente sobre qu� escenario(s)
		reparar, considerando siempre como factor crucial la utilidad que el sistema exhibir�a de ejecutar, o no, determinada
		estrategia de reparaci�n.

	\subsection{Caso 5: \emph{Tradeoff} entre Escenarios seg�n \emph{Concerns}}

		El objetivo de este caso de prueba es analizar c�mo Arco Iris, al tener que escoger entre favorecer dos escenarios
		con igual prioridad, elige favorecer a aquel cuyo \emph{concern} posee un peso mayor para el entorno de ejecuci�n
		actual.

		Los escenarios utilizados en este caso de prueba son id�nticos a los del caso anterior, excepto que ahora pasan a
		tener ambos igual prioridad y el escenario de costo no posee estrategias de reparaci�n asociadas. Por otro lado, el
		entorno de carga normal pasa a asignar mayor peso al \emph{concern} costo de servidores:
		
		\begin{figure}[H]
			\centering
				\includegraphics[scale=0.57]{images/testcase5_normal_environment_modified.png}
			\caption{Nueva distribuci�n de pesos para el entorno ``normal''}
			\label{fig:testcase5_normal_environment_modified}
		\end{figure}

		En el extracto del \emph{log} generado por Arco Iris se observa c�mo, al igual que en la mayor�a de los casos de
		pruebas ya presentados aqu�, el escenario de tiempo de respuesta deja de cumplirse y Arco Iris debe decidir qu�
		acci�n llevar a cabo ante dicha situaci�n. En la figura \ref{fig:Caso5} puede apreciarse que Arco Iris opta por no
		reparar el escenario de tiempo de respuesta ya que esto perjudicar�a al escenario de costo, puesto que si bien ambos
		poseen id�ntica prioridad, el �ltimo est� relacionado al \emph{concern} \textbf{costo de servidores}, el cu�l en el
		entorno de ejecuci�n actual (normal) tiene m�s peso que el \emph{concern} \textbf{tiempo de respuesta}.
		
		\begin{figure}[ht]
			\centering
			\begin{Verbatim}[gobble=4]
				Current environment: NORMAL
				Computing Current System Utility...
				Server Cost Scenario broken for [ESum] 1.0? false
				Experienced Response Time Scenario broken for [EAvg] 602.25? true
				Current System Utility (Score to improve): 0.555
				Evaluating strategy EnlistServerResponseTime...
				  Scoring EnlistServerResponseTime...
				  Server Cost Scenario broken after simulation for Server Cost ([ESum] 2.0)? true
				  Experienced Response Time Scenario broken after simulation for Response time ([EAvg] 458.25)? false
				  Score for strategy EnlistServerResponseTime: 0.111
				NO applicable strategy, adaptation cycle ended.
			\end{Verbatim}
			\caption{\emph{log} de Arco Iris para el caso de prueba 5}
			\label{fig:Caso5}
		\end{figure}

		En otras palabras, al igual que en el caso anterior, Arco Iris vuelve a utilizar la Utilidad del sistema como una
		medida para estimar el estado en que quedar�a el sistema, de ejecutar o no una determinada estrategia; concluyendo
		que, en el caso de intentar reparar el escenario de tiempo de respuesta ``roto'', el sistema brindar�a menos utilidad
		que en el estado actual. Vale reiterar que el factor determinante en el c�lculo de dicha utilidad simulada del sistema
		es, ni m�s ni menos, que el peso del \emph{concern} costo de servidores posee en el entorno de ejecuci�n actual.
		
		Cabe mencionar que, avanzada ya la ejecuci�n del sistema, el tiempo de respuesta vuelve a estar por debajo del umbral
		m�ximo definido en el escenario correspondiente. Esto, nuevamente, se debe a condiciones cambiantes en el entorno de
		ejecuci�n y no a una acci�n activa llevada a cabo por el \emph{framework} para lograr tal efecto. Esta situaci�n,
		junto con las variaciones del tiempo de respuesta durante toda la ejecuci�n de este caso de prueba, pueden
		visualizarse en la figura siguiente:
		
		\begin{figure}[H]
			\centering
				\includegraphics[width=1.00\textwidth]{images/testcase5_expRespTime.png}
			\caption{El tiempo de respuesta no es reparado por Arco Iris, arreglandose s�lo luego.}
			\label{fig:testcase5_expRespTime}
		\end{figure}

		Cabe destacar que el costo de servidores, por su parte, se mantiene constante en 1 durante toda la ejecuci�n.

	\subsection{Caso 6: Comportamiento Ante Los Cambios en el Entorno de Ejecuci�n}

		Este caso de prueba pretende demostrar c�mo Arco Iris var�a su comportamiento dependiendo del entorno en el cual se
		encuentra ejecutando el sistema en un momento dado.
	
		Para este caso es necesario contar con tres escenarios similares a los utilizados hasta aqu� pero con las siguientes
		variaciones:
		\begin{itemize}
		  \item Un escenario de \textbf{prioridad 3} cuya cuantificaci�n de la respuesta acota el tiempo de respuesta a 600
		  ms. como m�ximo, cuando el sistema se encuentra en \textbf{carga normal}.
		  \item Un escenario de \textbf{prioridad 1} que define que el tiempo de respuesta no debe superar los 900 ms. cuando
		  el sistema se encuentra bajo \textbf{alta carga}.
		  \item Un escenario de \textbf{prioridad 2} que predica sobre el costo de servidores, \textbf{limitando al sistema a
		  utilizar como m�ximo un servidor}, cuando se encuentre en un entorno de \textbf{carga normal}.
		\end{itemize}
		
		En otras palabras, mientras el sistema se encuentre dentro de par�metros normales de carga (i.e. carga normal), se
		prefiere que el tiempo de respuesta suba por encima de su umbral m�ximo (600 ms.) antes que agregar un servidor m�s
		al sistema. Ahora, cuando el entorno de ejecuci�n es de alta carga, el tiempo de respuesta pasa a tener m�s
		importancia que el costo de servidores. Esto, en teor�a, habilitar�a a Arco Iris a agregar uno o m�s servidores en pos
		de que el tiempo de respuesta no supere su umbral m�ximo (900 ms.) para un entorno de alta carga. Este es el objetivo
		primordial de este caso de prueba.
		
		Para lograr que el sistema en un momento determinado de la ejecuci�n pase a estar en alta carga, ha sido necesario
		modificar los par�metros de la simulaci�n, generando as� condiciones que emulan un incremento en la carga de pedidos
		al �nico servidor disponible en el sistema. Recordar que la condici�n que especifica cu�ndo el sistema se encuentra en
		alta carga, est� configurada en el entorno definido en la secci�n \ref{sec:configBasicaCasosPrueba} (ver figura
		\ref{fig:Environment_HighLoad}).
		
		Los escenarios de tiempo de respuesta tendr�n asociada la estrategia de auto reparaci�n
		\verb@EnlistServerResponseTime@, utilizada en pruebas anteriores, mientras que el escenario de costo no poseer�
		estrategias asociadas.
		
		Tal como se esperaba, Arco Iris no toma acci�n alguna mientras el sistema se encuentra en carga normal, puesto que el
		escenario de costo posee m�s prioridad que el escenario de tiempo de respuesta para dicho entorno. Esto se puede
		apreciar en la figura \ref{fig:testcase6_expRespTime}.
		
		Luego de superados los 800 ms. el sistema pasa a estar en alta carga, por lo cual dos de los tres escenarios pasan a
		satisfacerse trivialmente. Consecuentemente, Arco Iris deber� procurar s�lamente que se cumpla el escenario de tiempo
		de respuesta que impide que se superen los 900 ms. Puede observarse en la figura \ref{fig:testcase6_cost} que la auto
		reparaci�n decide agregar un servidor aproximadamente a partir de los 23 segundos de iniciada la simulaci�n, reparando
		efectivamente el escenario en cuesti�n.

		\clearpage
		\begin{figure}[ht]
			\begin{center}
				\subfigure[Arco Iris s�lo arregla el tiempo de respuesta con el sistema en alta carga.]
						  {\label{fig:testcase6_expRespTime}\includegraphics[width=\textwidth]{images/testcase6_expRespTime.png}}
				\subfigure[En carga normal, el costo es priorizado y se mantiene en 1.]
						  {\label{fig:testcase6_cost}\includegraphics[width=\textwidth]{images/testcase6_cost.png}}
			\end{center}
			\caption{Comportamiento del sistema en distintos entornos de ejecuci�n}
			\label{fig:Caso6}
		\end{figure}
	
	\newpage
	
	\section{Trabajo Futuro}
\label{sec:trabajoFuturo}

	Si bien con la introducci�n de las extensiones presentes en Arco Iris se han presentado considerables mejoras con
	respecto al uso b�sico de Rainbow, tambi�n se han suscitado numerosos puntos de extensi�n posibles que si bien no
	pudieron formar parte del presente trabajo, es importante que sean detallados en pos de incentivar la investigaci�n
	futura. A continuaci�n, citaremos los puntos de extensi�n posibles a Arco Iris y/o Rainbow.

	\subsection{Arco Iris: un \emph{plug-in} de Rainbow}

			Una caracter�stica deseable para Arco Iris es, sin duda, que la extensi�n se inserte en el \emph{framework}
			(conceptualmente hablando) de manera similar a c�mo un \emph{plug-in} trabaja en cualquier otro sistema de
			\emph{software}, pudiendo el usuario, de manera relativamente simple, elegir entre utilizar los mecanismos de auto
			reparaci�n provistos por Arco Iris o los de Rainbow.

			Para que esto sea posible, es necesario que Rainbow provea una arquitectura abierta a la extensi�n de sus componentes
			claves, como ser el \verb@AdaptationManager@, el \verb@RainbowModel@ y \verb@GaugeCoordinator@, entre otros.

			A continuaci�n, se enumeran algunos de los motivos por los cuales Rainbow, en la actualidad, no permite extensiones
			de tipo \emph{plug-in}:
			\begin{enumerate}
				\item Rainbow no trabaja orientado a interfaces, tal como Erich Gamma explica en \cite{DesignPatterns}, observando
				as� un alto acoplamiento entre las clases que conforman su dise�o y la consecuente imposibilidad de cambiar las
				implementaciones por defecto por otras nuevas.
				\item El mecanismo utilizado por los componentes de Rainbow para obtener referencias a otros componentes con los que
				desean interactuar, implementado en una clase Java denominada \verb@Oracle@), es insuficiente para permitir la
				inyecci�n de diferentes implementaciones del mismo componente. Esto es un error de dise�o �ntimamente relacionado
				con el punto anterior.
				\item Muchos de los componentes b�sicos de Rainbow, los cuales se modelan como clases Java, impiden expl�citamente
				su extensi�n (i.e. herencia) mediante modificadores de acceso restrictivos.
			\end{enumerate}

			A modo de conclusi�n, ser�a muy positiva una reestructuraci�n de Rainbow para permitir extensiones o agregados al
			mismo de una manera m�s prolija y ordenada que la que tuvo que utilizarse para la creaci�n de Arco Iris.

	\subsection{An�lisis y Aprendizaje de la Auto Reparaci�n}

		\subsubsection{Herramientas de an�lisis y visualizaci�n}

			Supongamos que los \emph{stakeholders} ya han definido todos los escenarios que utilizar� Arco Iris para gestionar la
			auto adaptaci�n de un sistema dado. Ahora bien, imaginemos que existe un escenario de muy alta prioridad y a su vez
			las condiciones necesarias para que no se cumpla se dan con mucha frecuencia. Esto posiblemente resulte en la auto
			reparaci�n repetida del mencionado escenario. Pero, �qu� suceder�a si la �nica estrategia que repara dicho escenario
			``rompe'' constantemente otros, digamos tres, escenarios de menor prioridad? A priori pareciera que este es el
			comportamiento esperado, pero\ldots �podr�an los \emph{stakeholders} convalidar este comportamiento? Claramente la
			implementaci�n actual de Arco Iris carece de una serie de \textbf{estad�sticas y vistas} que permitan analizar lo que
			est� siendo reparado, as� tambi�n como el impacto sobre los escenarios perjudicados y sobre cada uno de los
			\emph{concerns} del sistema.

			Tambi�n ser�a �til proveer una herramienta que permita analizar qu� estrategias se fueron ejecutando hist�ricamente,
			y de qu� manera se lleg� a decidir que cada una de ellas era la adecuada para reparar el sistema en cada momento.

		\subsubsection{M�s Visibilidad Sobre Las Estrategias Fallidas}

			Tal como se coment� en la secci�n \ref{sec:failureRate}, Rainbow provee un mecanismo que considera la historia de
			ejecuci�n de las estrategias con el objetivo de evitar elegir aquellas con un alto porcentaje hist�rico de fallas,
			esto es, estrategias que no resolvieron anteriormente las anomal�as que se supon�a deb�an resolver.
			
			Si bien este mecanismo es bastante aceptable, el mismo podr�a ser mejorado en el futuro en dos aspectos:
			\begin{itemize}
				\item El porcentaje de fallas m�ximo permitido para que una estrategia pueda ser considerada debe ser configurado
				por el usuario final. La forma m�s simple de hacer esto es sencillamente externalizando este valor al archivo de
				configuraci�n de la aplicaci�n ya existente (\verb@rainbow.properties@). Este trabajo no se ha encarado en el
				presente trabajo, puesto que implica la modificaci�n (en contraposici�n a una extensi�n) de una clase
				espec�fica de Rainbow. Es conveniente que las modificaciones al n�cleo de Rainbow sean realizadas por el grupo de
				gente a cargo de su mantenimiento evolutivo.
				\item Este mecanismo provee poca visibilidad al usuario final. Una posible idea para analizar la
				configuraci�n de la auto reparaci�n del sistema podr�a ser un \textbf{\emph{ranking} de estrategias}, en el que se
				presente el porcentaje de �xito y fallas de cada una as� como la cantidad de veces que se ejecut�, cu�l es la
				utilidad del sistema promedio luego de ejecutar la estrategia, etc.
				\item Otra posible idea podr�a ser el proveer una suerte de alarma al usuario administrador (v�a e-mail por ejemplo)
				que, previamente a desactivar una estrategia o luego de haber desactivado reiteradas veces la misma estrategia,
				avise sobre este tipo de situaciones, haciendo expl�cito este tipo de decisiones cruciales y permitiendo as� la
				rectificaci�n o ratificaci�n de la configuraci�n de \emph{self healing} que est� siendo utilizada.
			\end{itemize}

	\subsection{Ampliaci�n de la Recarga Din�mica de Configuraci�n}

		En la secci�n \ref{sec:actualizacionDinamicaConfig} se explic� el trabajo hecho en Arco Iris en materia de recarga
		autom�tica de la configuraci�n. Recordemos que se implement� un mecanismo que refresca la configuraci�n relativa a
		Escenarios, Entornos, \emph{Artifacts} y las referencias a las estrategias asociadas a cada escenario. Si bien tal
		mecanismo representa una mejora sustancial con respecto a Rainbow (el cual recordemos que no provee ning�n tipo de
		recarga ``en caliente'' de la configuraci�n) entendemos que se puede dar un paso m�s en la materia, a fin de que el
		\emph{framework} resulte m�s �til para su uso en un ambiente tan din�mico como el de la industria del \emph{software};
		d�nde el reiniciar la aplicaci�n para aplicar un cambio en la configuraci�n resulta muchas veces sencillamente
		inaceptable. En consecuencia, es altamente deseable tender a \textbf{recargar din�micamente el 100\% de la
		configuraci�n relacionada con auto reparaci�n}. Los cambios necesarios para lograr �sto deben llevarse a cabo
		principalmente en Rainbow, d�nde reside el grueso de la configuraci�n est�ticamente cargada.

		Rainbow actualmente maneja (a grandes rasgos) los siguientes puntos de configuraci�n:
		\begin{itemize}
			\item el archivo \verb@rainbow.properties@, un t�pico archivo \verb@.properties@ de Java, el cual es le�do una �nica
			vez al inicializarse el \emph{framework} y que sirve para externalizar propiedades tales como el nivel de \emph{log}
			deseado para la aplicaci�n, el \emph{path} en d�nde buscar estrategias y t�cticas, el archivo \verb@.acme@ con el
			modelo de la arquitectura del sistema a adaptar, etc. En este caso, ser�a relativamente sencillo implementar un
			mecanismo de recarga din�mica id�ntico al explicado en detalle en la secci�n \ref{sec:actualizacionDinamicaConfig}
			para recargar la configuraci�n de Arco Iris. El mismo tendr�a que ser realizado en el constructor de la clase
			\verb@Rainbow)@.
			\item \verb@utilities.yml@, d�nde se configuran las curvas de utilidad para el sistema. Al igual que en el caso
			anterior, estas propiedades se leen en el constructor de la clase \verb@Rainbow@ utilizando una clase \emph{helper}
			llamada \verb@YamlUtil@. Utilizando el mismo mecanismo descrito anteriormente se podr�a dinamizar tambi�n la recarga
			din�mica de este archivo.
			\item archivos Stitch (con extensi�n \verb@.s@) de t�cticas y estrategias. La l�gica de lectura de t�cticas y
			estrategias est� ubicada en el \verb@AdaptationManager@. Puesto que en Arco Iris se necesit� extender esta clase, se
			aprovech� la oportunidad para abstraer esta l�gica de lectura desde un archivo, a una clase espec�fica llamada
			\verb@StitchLoader@, la cual tiene la responsabilidad no s�lo de cargar los datos desde un archivo sino de proveer
			tambi�n acceso a los mismos. Ser�a una buena idea el incorporar esta clase al c�digo base de Rainbow puesto que dicha
			abstracci�n provee la posibilidad de incorporar r�pidamente el mecanismo de refresco mencionado ya reiteradas veces.
			Actualmente dicho comportamiento tampoco est� incorporado en Arco Iris, no porque sea un trabajo dif�cil sino debido
			a que dicha tarea se encuentra afuera del alcance del presente trabajo.
		\end{itemize}

		Como podemos observar, el mecanismo explicado en detalle en la secci�n \ref{sec:actualizacionDinamicaConfig} es
		f�cilmente reutilizable y tambi�n observamos posibilidades de mejora del mismo para simplificar a�n m�s su uso
		extendido en todo el \emph{framework}.

	\subsection{Ampliaci�n de la Configuraci�n Existente}
	\label{sec:ampliacionConfiguracionExistente}

		A lo largo del desarrollo del presente trabajo, hemos notado que existen algunas posibilidades de ampliar la
		configuraci�n existente en Arco Iris. A continuaci�n, mencionaremos algunos de ellos.

		\subsubsection{Toda la configuraci�n en un solo archivo}
		\label{sec:todaLaConfigEnUnSoloArchivo}

			A fin de simplificar el uso del \emph{framework} para el usuario final, es altamente deseable poder reducir la cantidad
			de archivos de configuraci�n que deben ser creados, modificados y/o mantenidos para que el t�ndem Rainbow / Arco Iris
			funcione. Ser�a deseable realizar modificaciones en ambos \emph{frameworks} para que la configuraci�n se encuentre
			centralizada en uno (idealmente) o dos archivos de configuraci�n; conglomerando all� informaci�n referente a t�cticas,
			estrategias, el modelo de la arquitectura y todo el modelo existente de Arco Iris, actualmente configurable via XML.

		\subsubsection{Mas tipos de restricciones por defecto}

			En Arco Iris los tipos de restricciones - debido a su inherentemente compleja l�gica - est�n codificados en clases Java
			que, en conjunto, configuran el abanico de restricciones soportadas por el \emph{framework}. Actualmente se provee
			soporte para un tipo solo de restricci�n: aquella que modela una restricci�n sobre el valor de una propiedad de un
			artefacto, con respecto a una funci�n binaria num�rica como por ejemplo los operadores $<$, $\le$, $>$, $\ge$ y
			$=$.\\ Si bien este tipo de restricciones son altamente representativas del tipo de cosas sobre las que un usuario
			promedio normalmente desea predicar, es de notar que el poder expresivo est� claramente acotado. Se propone para un
			trabajo futuro extender el esquema de restricciones modelado en Arco Iris, mediante nuevas implementaciones de la
			interfaz \verb@Constraint@.

	\subsection{Atributos de Calidad y \emph{Concerns} configurables por el usuario}

		Actualmente Arco Iris maneja un conjuntos fijo de Atributos de Calidad y \emph{Concerns}, los cuales se encuentran
		embebidos en clases Java a las cuales el usuario final no puede acceder. En otras palabras, el usuario no puede
		agregar o eliminar libremente Atributos de Calidad o \emph{Concerns}. Claramente, esta rigidez restringe la usabilidad
		del \emph{framework} y debe ser superada.

		Puesto que en Arco Iris los \emph{concerns} juegan un papel m�s preponderante que los Atributos de Calidad, en este
		apartado nos referiremos �nicamente al problema de permitir que el usuario final de Arco Iris pueda agregar o eliminar
		\emph{Concerns} a una configuraci�n de \emph{Self Healing}.

		Existen b�sicamente dos formas de compensar la falencia antes mencionada: una es sencilla de implementar pero
		incompleta y la otra es dif�cil de implementar aunque definitiva.

		\subsubsection{Implementaci�n sencilla e incompleta}

			El agregado de una amplia y variada gama de \emph{Concerns} en el c�digo fuente de Arco Iris representa una buena
			soluci�n de compromiso entre m�xima flexibilidad para el usuario y facilidad de implementaci�n, ya que s�lo se trata
			de recabar una lista de los \emph{Concerns} m�s conocidos y usualmente utilizados y agregarlos a la clase enumerada
			que modela los tipos posibles de \emph{Concerns} reconocidos por Arco Iris.

		\subsubsection{Implementaci�n dif�cil y definitiva}

			El problema de permitir el libre agregado y eliminaci�n de \emph{Concerns} por parte del usuario posee varios
			aristas, algunas de las cu�les no son f�ciles de resolver, a saber:

			\begin{itemize}
				\item \textbf{Actualizaci�n del mapa $<$Concern,Weight$>$ en todos los Entornos}

					Al agregar un concern al sistema, se suscita la duda sobre qu� hacer con el mapa\\
					\verb@<Concern,Weight>@ de cada uno de los Entornos ya existentes en el sistema.

					Una posible opci�n ser�a cambiar la sem�ntica actual de dicho mapa y que el mismo pase a almacenar �nicamente
					aquellos pesos que son distintos de cero, en otras palabras, al no encontrarse en el mapa un valor para un
					determinado \emph{Concern}, se sobreentiende que su peso relativo es 0 (cero).

					Existe otra opci�n que no cambia la sem�ntica de los mapas (i.e. todos los \emph{Concerns} seguir�an siendo
					enumerados por extensi�n) y para eso se vale de la herramienta Arco Iris UI. B�sicamente, al agregar un nuevo
					\emph{Concern}, Arco Iris UI autom�ticamente lo agregar�a tambi�n en los mapas de todos los Entornos existentes en
					el sistema, pero\ldots �con qu� peso? �con peso cero? �con un peso por defecto configurable al crear el concern? �se
					obligar�a al usuario a configurar el peso en todos los Entornos? En el caso de eliminaci�n, �qu� ocurre con el peso
					del \emph{Concern} eliminado? �se reparte entre los restantes \emph{concerns}? Pensemos adem�s en lo engorroso que
					ser�an todo este tipo de reajustes en la configuraci�n para un usuario que no utiliza Arco Iris UI\ldots

					�stas y otras preguntas son las que alguien encargado de flexibilizar la configuraci�n de \emph{Concerns} deber�a
					responder y, por ese motivo, preferimos s�lamente plantearlas y dejar su resoluci�n abierta.

				\item \textbf{Actualizaci�n del \emph{Concern} de Escenarios}

					En el caso de dar de baja un \emph{Concern} lo m�s razonable ser�a que Arco Iris UI lo elimine autom�ticamente de
					todos los Entornos, aunque cabe preguntarse: �qu� ocurre con los escenarios que estaban relacionados con este
					\emph{Concern}? �deben darse de baja? �debe forzarse su actualizaci�n? Nuevamente, si el usuario no usara Scenarios
					UI, el mantenimiento de la configuraci�n podr�a tornarse dificultoso.

				\item \label{asignacionEquidistribuidaPesos} \textbf{Asignaci�n equidistribu�da de pesos para un Entorno nuevo}

					En Arco Iris, al construir una instancia de \verb@Environment@, se invoca a una funci�n llamada
					\verb@createMapWithEquallyDistributedWeights@ que inicializa el mapa de pesos del Entorno de manera equidistribu�da
					entre todos los entornos (e.g. si hay 3 \emph{Concerns} en el sistema, asigna $0.\wideparen{33}$ a cada uno).

					Esta decisi�n de dise�o si bien es razonable, es tambi�n mejorable: esto podr�a ser configurable por el usuario del
					\emph{framework} para poder adaptar mejor la configuraci�n por defecto a los intereses particulares de cada
					sistema. Por ejemplo: es probable que un sistema financiero le otorgue m�s prioridad a los \emph{concerns}
					relacionados con la eficiencia, mientras que una aplicaci�n de \emph{e-commerce} es probable que le de m�s
					importancia a la seguridad.

					En el supuesto de que los \emph{Concerns} sean configurables por el usuario final, la funci�n
					\verb@createMapWithEquallyDistributedWeights@ deber�a reubicarse en una clase del tipo \emph{helper} que tenga
					acceso de alguna manera al \verb@SelfHealingConfigurationManager@ que es el objeto que tendr�a acceso a la
					informaci�n presente en el archivo de configuraci�n de \emph{self healing}.
					
				\item \textbf{Acceso del \emph{ArcoIrisAdaptationManager} a los \emph{Concerns}}

					Una situaci�n similar a la anterior (el acceso al universo de \emph{Concerns} definidos en el sistema) ocurre
					tambi�n en el \verb@ArcoIrisAdaptationManager@:

					\begin{Verbatim}[gobble=6]
						public static boolean isConcernStillBroken(String concernString) {
							try {
								Concern concern = Concern.valueOf(concernString);
								...
					\end{Verbatim}

					El \verb@ArcoIrisAdaptationManager@ actualmente posee como colaborador interno a\\
					\verb@RainbowModelWithScenarios@, el cu�l accede al \verb@SelfHealingConfigurationManager@, qui�n es el que
					puede proveer el acceso deseado a todos los \emph{Concerns} definidos en el sistema en un momento dado.
			\end{itemize}

	\subsection{Flexibilizaci�n del Entorno}
	\label{sec:flexibilizacionEntorno}

		Actualmente, el algoritmo que decide en qu� entorno de ejecuci�n se encuentra el sistema a adaptar en un
		determinado instante, presupone que las restricciones de todos los Entornos del sistema son mutualmente excluyentes
		entre s�. Esto tiene como principal consecuencia el hecho de que, al momento, no se pueden configurar Entornos que
		posean alg�n tipo de intersecci�n en sus condiciones de aplicabilidad. En particular y a modo de ejemplo, el
		usuario querr�a poder especificar un escenario de ``Alta Carga'' y otro de ``Extrema Alta Carga'', cuyas condiciones
		de aplicabilidad podr�an tener una intersecci�n no nula pues probablemente las condiciones necesarias para
		que el entorno de ``Alta Carga'' se cumpla est�n inclu�das en las condiciones el entorno de ``Extrema Alta Carga''.

		Es deseable que en el caso de entornos con intersecci�n no nula en sus condiciones de aplicabilidad, Arco Iris
		seleccione a aqu�l que sea m�s adecuado, de acuerdo al caso en particular. En el ejemplo anterior, ``Extrema Alta
		Carga'' ser�a el elegido para representar m�s fielmente el estado actual del sistema. Esto sin duda es altamente
		relevante ya que los pesos configurados en cada uno de los escenarios pueden determinar el curso de la auto reparaci�n
		puesto que afectan directamente al algoritmo de selecci�n de estrategias.

		Una posible soluci�n para este problema podr�a ser el explicitar relaciones de intersecci�n (o de inclusi�n) entre
		restricciones y reconfigurar el algoritmo de detecci�n del Entorno actual de ejecuci�n existente en el
		\verb@ArcoIrisAdaptationManager@ para que el mismo elija aquel Entorno con condiciones m�s generales. A
		continuaci�n, un pseudo-c�digo del posible algoritmo que contempla \textbf{inclusi�n entre Entornos}:

		\begin{Verbatim}[gobble=3]
			funci�n detectCurrentSystemEnvironment
				candidate = null;
				para cada ent en todos los entornos
					si ent aplica en las actuales condiciones de ejecuci�n entonces
						si candidate == null entonces
							candidate = ent;
							siguiente iteraci�n;
						fin si
				        si ent est� incluido en candidate entonces
				        	candidate = ent;
				        fin si
				    is
				done
		\end{Verbatim}

	\subsection{Configuraci�n de Escenarios en AcmeStudio}

		Actualmente existe una herramienta de creaci�n y edici�n de arquitecturas modeladas en Acme, llamada
		\textbf{AcmeStudio} \footnote{Para m�s informaci�n acerca de AcmeStudio, visitar
		\url{http://www.cs.cmu.edu/~acme/AcmeStudio}} la cual se encuentra integrada en la popular herramienta de desarrollo
		Eclipse\footnote{Para m�s informaci�n acerca de Eclipse, visitar \url{http://www.eclipse.org}} como un \emph{plug-in}.

		Se propone como trabajo a futuro extender la herramienta AcmeStudio para que d� soporte a las extensiones propuestas
		en el presente trabajo, permitiendo as� la integraci�n del modelado de la arquitectura con el modelado de los
		escenarios y dem�s conceptos introducidos en Arco Iris, utilizando la misma herramienta.

		Dependiendo del grado de profundidad en el que se avance en la integraci�n de Arco Iris con Acme Studio, se podr�a
		incluso especular con un reemplazo completo de la herramienta Arco Iris UI. Esta tarea no fue realizada en el
		contexto del presente trabajo debido a que el encarar el desarrollo de un \emph{plug-in} de Eclipse con estas
		caracter�sticas era una tarea de alta complejidad, que se encuentra afuera del alcance de este trabajo.

	\subsection{Optimizaci�n en la Selecci�n de la Estrategia}

		En la secci�n \ref{sec:strategySelection} se describe en detalle el mecanismo de selecci�n de la mejor estrategia para
		resolver una situaci�n an�mala en el sistema que se est� adaptando. Este problema posee muchas aristas complejas,
		por lo tanto, una soluci�n sofisticada requiere un trabajo de investigaci�n que excede lo que se puede realizar en
		este trabajo. Esta idea fue adem�s compartida por los investigadores de Carnegie Mellon que fueron consultados al
		inicio del presente trabajo.
		
		La soluci�n propuesta aqu� al problema de selecci�n de la mejor estrategia de reparaci�n, es una heur�stica simple
		basada en el uso del concepto de \textbf{utilidad del sistema}, el cu�l, si bien es �til para reducir la dimensi�n del
		problema y poder avanzar en otros aspectos, es claramente una soluci�n limitada, debido a que para el c�lculo de esta
		utilidad se realizan muchos supuestos y simplificaciones sobre el contexto de ejecuci�n del sistema y su correlato con
		el modelo de su arquitectura. Las oportunidades de mejora avisoradas por los autores de este trabajo pasan
		principalmente por implementar mecanismos de aprendizaje sobre qu� es lo que funcion� y qu� no en reparaciones
		anteriores, por mecanismos din�micos de modificaci�n de los pesos de los \emph{concerns} y prioridades de los
		escenarios en funci�n del estado del sistema en un momento dado y finalmente por ajustes a las t�cticas y estrategias
		que se aplican.

	\subsection{Ausencia u Obsolescencia del Modelo de la Arquitectura}
	\label{sec:ausenciaObsolescenciaModeloArquitectura}

		Tanto Rainbow como Arco Iris hacen foco en disponer de un modelo (actualizado) de la arquitectura del sistema a
		adaptar pero en realidad, no siempre es posible disponer del mismo: puede no haber existido nunca o bien haber
		quedado desactualizado. Existen l�neas de investigaci�n que intentan derivar el modelo de la arquitectura a partir de
		un sistema, obteniendo informaci�n del mismo en \emph{runtime}. Uno de los proyectos tiene lugar en el contexto del
		grupo de investigaci�n ABLE, el responsable de Rainbow y se llama \textbf{DiscoTect}. Para m�s informaci�n sobre
		dicho proyecto, visitar \url{http://www.cs.cmu.edu/~able/research/discotect.html}.

	\subsection{Adaptaci�n de un Sistema Sobre el Cual se Tiene Poco Control}

		Otro de los t�picos sobre los cuales es necesario tambi�n avanzar, es en desarrollar t�cnicas y herramientas que
		permitan poder alterar el comportamiento de un sistema sobre el cu�l se posee muy poco control. Tal es el caso, por
		ejemplo, de sistemas desarrollados por terceras partes, de los cuales no se posee ni siquiera el c�digo fuente. Otro
		ejemplo podr�a ser el de sistemas pobremente modularizados o codificados con un lenguaje de programaci�n antiguo sobre
		los cu�les parece poco probable que se puedan establecer puntos de comunicaci�n con un \emph{framework} de auto
		adaptaci�n como Arco Iris o Rainbow.
		
		Este tema es cr�tico para el desarrollo de la auto adaptaci�n de sistemas de \emph{software} considerando que existen
		en el mundo numerosas aplicaciones que se encuentran en las condiciones descritas anteriormente. 

	\subsection{Mecanismo de Reparaci�n ``declarativo''}

		Hasta el momento, en Rainbow y en Arco Iris, las estrategias de reparaci�n se encuentran configuradas de una manera
		imperativa, es decir, el arquitecto o una persona con ese rol, configura el \emph{framework} especificando una serie
		de estrategias de reparaci�n, que no son m�s que algoritmos que intentan resolver o paliar una anomal�a en el sistema
		basandose en observaciones sobre las propiedades del mismo en \emph{runtime} y en un conjunto de t�cticas (e.g.
		levantar un servidor extra, disminu�r el nivel de \emph{logging}, apagar la encripci�n, etc\ldots) tambi�n provistas
		por el usuario del \emph{framework}.
		
		Otra posibilidad menos program�tica y m�s declarativa podr�a consistir en que el usuario �nicamente especifique, para
		un escenario, las t�cticas que podr�an ejecutarse en el caso en que el escenario no se cumpla; dejando al
		\emph{framework} la tarea de combinar ``inteligentemente'' dichas t�cticas de la mejor manera posible. Es decir, el
		\emph{framework} deber�a poder ser capaz de manejar heur�sticamente los recursos que posee para interactuar con el
		sistema en \emph{runtime}, de una manera similar a c�mo se opera actualmente con estrategias est�ticamente
		configuradas por el usuario.
		
		Lo antedicho agregar�a flexibilidad puesto que el \emph{framework} no estar�a acotado a un conjunto fijo de
		estrategias pre configuradas sino que podr�a intentar otras opciones din�micamente, considerando la historia de sus
		ejecuciones pasadas, entre otros tantos aspectos posibles. Por otro lado, siempre que se agrega flexibilidad,
		normalmente se pierde en predictibilidad, ya que las acciones del \emph{framework} ser�an m�s dif�ciles de
		seguir y predecir.

	
	\newpage
	
	\section{Conclusiones}

	\subsection{Resumen del trabajo realizado}

		En resumidas palabras, el presente trabajo a�ade una serie de caracter�sticas a Rainbow, todas ellas tendientes a
		disponer de una herramienta gen�rica de auto reparaci�n m�s poderosa y flexible y que, al mismo tiempo, los actores
		funcionales del sistema se vean involucrados en el proceso de definici�n de requerimientos de atributos de calidad,
		puesto que este tipo de actores son los encargados de determinar (junto a arquitectos, dise�adores, etc.) c�mo se debe
		comportar el sistema ante determinadas situaciones de operaci�n. Las caracter�sticas a�adidas en este trabajo son
		b�sicamente las siguientes:
		\begin{itemize}
			\item Posibilidad de modelar escenarios de atributos de calidad siguiendo los principios de ATAM y QAW.
			\item Posibilidad de relacionar escenarios con componentes de la arquitectura.
			\item Posibilidad de especificar prioridades relativas entre escenarios, a ser utilizadas en la elecci�n de la
			estrategia de reparaci�n a ejecutar.
			\item Definici�n de entornos de ejecuci�n para el sistema, con ponderaciones particulares para cada uno
			de los \emph{concerns} existentes, los cuales juegan un papel determinante en el algoritmo de elecci�n de
			estrategias de reparaci�n.
			\item Posibilidad de asociar estrategias de reparaci�n a escenarios.
			\item Modelado de un nuevo algoritmo de elecci�n de la mejor estrategia, utilizando todos los conceptos
			introducidos en Arco Iris.
			\item Implementaci�n de cambios en diversos m�dulos de Rainbow (como por ejemplo, el \emph{RainbowModel} y el
			\emph{AdaptationManager} e implementaci�n de algunos casos pr�cticos que permitan mostrar c�mo esta estrategia
			puede funcionar y llevar a un \emph{framework} de adaptaci�n m�s flexible y poderoso.
			\item Implementaci�n de un mecanismo de recarga din�mica de la configuraci�n de Arco Iris, la cu�l en un futuro
			podr�a ser f�cilmente integrada a Rainbow en pos de dinamizar el uso del \emph{framework}, minimizando su
			\emph{downtime} debido a cambios en la configuraci�n.
			\item Implementaci�n de una herramienta visual de escritorio que permita al usuario de Arco Iris, configurar el
			\emph{framework} de una manera sencilla, amena y eficiente.
		\end{itemize}

		Todo lo antedicho tiene como consecuencia que los usuarios tendr�n m�s herramientas para influir sobre lo que el
		sistema decida hacer para auto repararse: s�lo con modificar la informaci�n de los escenarios el sistema modificar� su
		comportamiento. Como contrapartida, las extensiones propuestas en este trabajo hacen que el sistema se comporte de
		manera m�s aut�noma a medida que se agregan escenarios; esto parad�jicamente le quita control al usuario, ya que el
		procedimiento de auto reparaci�n se vuelve m�s complejo, dificultando el seguimiento de las decisiones tomadas por el
		\emph{framework}.

	\subsection{Arco Iris comparado con Rainbow}

		\noindent En la presente secci�n repasaremos las ventajas que posee Arco Iris por sobre Rainbow. Profundizaremos en
		los siguientes puntos b�sicos:

		\noindent \todo{sacar grafico y poner tabla cuando imprimamos la versi�n definitiva (es para evitar falsos bad boxes)}
%  		\begin{center}
%  			\scriptsize{
%  			\rowcolors*[\hline]{1}{GreenYellow!25}{GreenYellow!10}
%  			\begin{tabularx}{\textwidth}{|X|X|X|X|}
%   			\textbf{Problema} & \textbf{Implementaci�n en Rainbow} & \textbf{Implementaci�n en \mbox{Arco Iris}} &
%   			\textbf{Ventajas}\\
%   			Rapidez en el cambio de configuraci�n & La configuraci�n se lee una vez al principio y no puede ser cambiada
%   			din�micamente mientras Rainbow est� funcionando. & Los escenarios se actualizan din�micamente en \emph{runtime} al
%   			ser modificados. & Una gran parte de los cambios de configuraci�n pasan a impactarse ins\-tan\-t�\-nea\-men\-te,
%   			evitando reiniciar el sistema para cambiar la configuraci�n\\
%   			Informaci�n sobre restricciones & Guardadas en el modelo de la arquitectura (en \mbox{ACME}) & Mediante escenarios
%   			de QAW editados por \emph{stakeholders} & F�cil y transparente agregado y edici�n de las mismas.\\
%   			Decisiones sobre que reparaciones realizar & Tambi�n se encuentran en archivos de configuraci�n & Se evalu�n los
%   			escenarios cargados y como se afectan entre ellos. Luego se invoca a un m�dulo que decide qu� reparaciones pueden
%   			llevarse a cabo en ese contexto. & Posibilidad de extensi�n de dicho m�dulo para inclu�r complejas heur�sticas que
%   			aprendan de los efectos de reparaciones pasadas, etc.\\
%   			Entorno de ejecuci�n (e.g. Alta Carga)& Configurado est�ticamente. Para modificarlo es necesario el reinicio del
%   			\emph{framework}. & Se permite especificar distintos entornos de ejecuci�n. El sistema cambia
%   			di\-n�\-mi\-ca\-men\-te de en\-tor\-no sin necesidad de in\-ter\-ven\-ci�n humana ni reinicio del \emph{framework}.
%   			& Se agrega una nueva variable a considerar para la elecci�n de una estrategia, aumentando la probabilidad de que
%   			la misma se adeque mejor a la situaci�n actual del sistema.
%  			\end{tabularx}
%  			}
%  		\end{center}
			
		\begin{center}
			\includegraphics[width=1.00\textwidth]{images/ArcoIrisVsRainbow.png}
		\end{center}

		\subsubsection{Rapidez en el cambio de configuraci�n}
			Como se ha mencionado anteriormente, Arco Iris (es decir, Rainbow con las extensiones realizadas en este trabajo)
			trabaja utilizando:
			\begin{itemize}
				\item por un lado, una configuraci�n de \emph{Self Healing} en formato XML, producida ya sea por la herramientas
				Arco Iris UI o bien manualmente; la cual incluye: escenarios de QAW, entornos de ejecuci�n y \emph{artifacts}.
				\item por otro, una bater�a de archivos de configuraci�n heredados de Rainbow, en distintos formatos: stitch, ACME,
				\emph{properties files}, etc.
			\end{itemize}
			Una de las mejoras que agrega este trabajo a Rainbow es el de proveer un mecanismo din�mico de actualizaci�n de la
			configuraci�n de \emph{Self Healing} mediante el cual cualquier modificaci�n en el archivo XML de configuraci�n
			cargada al iniciar Arco Iris, conlleva un ``refresco'' de la configuraci�n que est� siendo usada en tiempo de
			ejecuci�n por el \emph{framework}, sin necesidad de intervenci�n alguna de un operador.
			
			Este dinamismo agregado en lo que respecta a los cambios de configuraci�n, trae aparejado una considerable serie de
			ventajas, a saber:
			\begin{itemize}
				\item ya no es necesario reiniciar el \emph{framework} de auto reparaci�n para modificar cualquier caracter�stica
				relacionada con los escenarios, entornos de ejecuci�n y \emph{artifacts} utilizados en el sistema. Esto evita
				tiempos de no-servicio, recursos involucrados en la reconfiguraci�n y reinicio del sistema, etc. 
				\item se agrega flexibilidad al uso de la herramienta: se podr�an agregar nuevos escenarios, nuevos entornos de
				ejecuci�n que (re)definan situaciones de ejecuci�n concretas, cambiar pesos de \emph{concerns} y/o prioridades relativas
				entre escenarios de acuerdo a necesidades puntuales del negocio e infinidad de otros cambios de configuraci�n
				relevantes a la auto reparaci�n quedar�an impactados en el sistema autom�ticamente.
			\end{itemize}
			
		\subsubsection{Informaci�n sobre restricciones}
			Como se ha visto anteriormente, en Rainbow las restricciones impuestas sobre el funcionamiento del sistema (e.g.
			``el tiempo de respuesta para cualquier tipo de \emph{request} no debe superar los 5 m�lisegundos'') son guardadas en
			el modelo de la arquitectura, el cual est� expresado en el lenguaje de descripci�n de arquitecturas \mbox{ACME}.
			Esto claramente representa un impedimento para que los \emph{stakeholders} no t�cnicos puedan participar activamente
			en la definici�n de requerimientos de auto reparaci�n del sistema; ya que de intentarlo, deber�an poder comprender y
			modificar correctamente un diagrama de arquitectura, con sus restricciones escritas en un lenguaje no coloquial sino
			especialmente t�cnico.
			
			En Arco Iris, los \emph{stakeholders} usan una interfaz visual\ref{sec:arcoIrisUI} intuitiva y f�cil de utilizar
			para expresar restricciones del sistema en formato de escenarios de QAW; los cuales a su vez fueron pensados
			originalmente para facilitar la participaci�n de personas con roles funcionales.
			
			La informaci�n incorporada de esta manera, es analizada por Arco Iris para tomar decisiones en tiempo de ejecuci�n
			sobre las auto reparaciones a realizar, considerando una variedad de aspectos, todos ellos configurados en un �nico
			lugar: los escenarios de QAW.
			
			Como vemos, Arco Iris provee una manera simplificada de agregar y/o editar restricciones sobre el sistema, sin
			necesidad de modificar su modelo de arquitectura, sino por medio de los escenarios de QAW. Esto representa una avance
			sobre como funciona Rainbow hoy d�a en ese aspecto.

		\subsubsection{Decisiones sobre que reparaciones realizar}

			Con respecto a este punto, la diferencia principal entre Rainbow y Arco Iris es el enfoque: al momento de
			decidir qu� estrategia es la m�s adecuada para ser ejecutada en el contexto de una o m�s \emph{constraints} no
			cumpliendose, en Rainbow se calcula un \emph{score} para cada una de las estrategias existentes en el sistema,
			mientras Arco Iris busca aquellos escenarios de QAW ``rotos'' y s�lo asigna un \emph{score} a las estrategias
			definidas en dicho subconjunto de escenarios. Este enfoque, adem�s de ser m�s eficiente, es considerablemente mas
			sencillo de configurar y tiene m�s sentido a nivel funcional, ya que en el caso de Rainbow no parece haber una manera
			sencilla de intu�r de antemano qu� estrategia eligir�a en cada caso, hecho que complica en an�lisis de las
			reparaciones efectuadas por Rainbow durante el transcurso de una ejecuci�n.

		\subsubsection{Entorno de ejecuci�n}

			En Rainbow, el concepto de entorno de ejecuci�n es est�ticamente configurado en un archivo de configuraci�n y no se
			modifica de acuerdo al estado din�mico del sistema que se intenta reparar. Para modificar tal est�tico y limitado valor, es
			necesario el reinicio de Rainbow.
			
			El modelo de Arco Iris incluye el concepto de Entorno de Ejecuci�n tal cual es descripto en la
			metodolog�a ATAM\cite{ATAM}, el cual permite al usuario especificar distintos entornos de ejecuci�n posibles para el
			sistema, los cuales poseen restricciones asociadas que se chequean continuamente y que, de cumplirse, hacen que se
			considere que el sistema se encuentra en otro entorno de ejecuci�n, afectando as� al algoritmo de decisi�n de
			estrategias de reparaci�n; ya que dicha decisi�n est� condicionada al \emph{scoring} de estrategias de reparaci�n, el
			cual se realiza considerando los pesos relativos que poseen los \emph{concerns} arquitecturales en el entorno de ejecuci�n
			actual.
			
			Lo antedicho, agrega una nueva arista a las variables consideradas al momento de elegir la mejor estrategia de
			reparaci�n, aumentando las probabilidades de que la estrategia seleccionada sea m�s precisa de acuerdo a la situaci�n
			del sistema en tiempo de ejecuci�n.
			
			Si bien en Rainbow el concepto de ``entorno de ejecuci�n'' no existe como tal, uno podr�a encontrarlo indirectamente
			dentro de ciertas condiciones de aplicabilidad embebidas dentro del c�digo de las estrategias de reparaci�n.
			
			Ahora bien, observemos que Rainbow no tiene forma de determinar a priori si alguna de las estrategias configuradas
			por el usuario van a aplicar en el contexto actual de ejecuci�n: dada alguna \emph{constraint} violada, necesita
			inspeccionar la totalidad de las estrategias para verificar si son aplicables en la situaci�n actual de
			\emph{runtime}. Esto �ltimo es una notable diferencia a favor de Arco Iris, ya que en este caso, al estar las
			estrategias de reparaci�n candidatas embebidas en el escenario, aquellos escenarios cuyo entorno no se corresponda
			con el entorno actual de ejecuci�n, no se considerar�n ``rotos'', evitando c�mputos innecesarios y refinando
			as� el m�todo original provisto por Rainbow.

	\subsection{Compatibilidad hacia atr�s: una empresa sin sentido}

		Uno de los objetivos de dise�o m�s deseables en una extensi�n a un \emph{framework} ya existente como Rainbow, es sin
		dudas que la extensi�n sea ``compatible hacia atr�s'' con \emph{Rainbow}, entendiendose �sto como la capacidad de
		Arco Iris de agregar nuevas caracter�sticas manteniendo al 100\% la funcionalidad original. 
		
		En principio, si se considera �nicamente el dise�o de alto nivel de Rainbow, la idea de una extensi�n \emph{backwards
		compatible} parece factible. Pero, lamentablemente, si se consideran las diferencias conceptuales de ambos modelos,
		no es dif�cil notar que el intento de mantener a ambos conviviendo no s�lo no es una tarea sencilla sino que tambi�n
		carece de sentido, ya que, en algunos puntos claves, son enfoques diametralmente opuestos. Veamos algunos de los
		motivos que apoyan esta idea:
		
		\begin{itemize}
			\item \textbf{C�mo se dispara la auto reparaci�n:} El enfoque de Arco Iris hace centro en el concepto de Escenario
			como medio para expresar requerimientos de atributos de calidad para un sistema. Con lo cual, el inicio de todo el
			proceso de auto reparaci�n se da de una manera sincr�nica con respecto al Est�mulo de alg�n escenario(s) ocurrido en
			el sistema en ejecuci�n. Esto se contrapone con el enfoque de Rainbow, en el cual, peri�dicamente se verifican todas
			las restricciones e invariantes de la arquitectura. Observamos que si quisieramos mantener esta �ltima
			caracter�stica de Rainbow, ya el foco absoluto de la auto reparaci�n no estar�a puesto en los Escenarios de
			atributos de calidad sino que paralelamente existir�an dos flujos distintos de auto reparaci�n disputando entre s�.
			Esto llevar�a a un comportamiento excesivamente complejo y poco predecible, lo cual resulta poco deseable para un
			\emph{framework} de auto reparaci�n.
			\item \textbf{C�mo se decide cu�ndo adaptar:} Arco Iris s�lo intenta efectuar adaptaciones acotadas al escenario o
			los escenarios que se dejan de cumplir en un determinado momento, mientras que Rainbow contempla todo el abanico de
			estrategias disponibles en el caso de que detecte que una violaci�n a una restricci�n o invariante ha tenido lugar.
			Evidentemente, los enfoques son distintos, sin mencionar que los algoritmos de \emph{scoring} de estrategias de
			ambos \emph{frameworks} difieren considerablemente, agregando el algoritmo de Arco Iris nuevas variables a la
			f�rmula, tal como se explica en la secci�n \ref{sec:arcoIrisStrategyScoring}.
		\end{itemize}

	\subsection{Aplicabilidad en sistemas reales}

		Luego de haber repasado las caracter�sticas principales de Arco Iris y de haber entendido su mec�nica, es muy probable
		que al lector se le suscite la siguiente pregunta: �Qu� aplicabilidad tiene esta extensi�n de Rainbow en un	sistema real?
		
		Consideramos que Arco Iris se encuentra un paso m�s cerca de ser utilizado en un sistema de \emph{software} real (i.e.
		en un �mbito no acad�mico) que lo que su antecesor, Rainbow, se encontraba. Esta consideraci�n se basa en el hecho de
		que Arco Iris hace foco en la accesibilidad del usuario final para configurar el \emph{framework} de una manera amena
		y m�s flexible que la provista originalmente por Rainbow, incluyendo una interfaz de usuario visual que facilita dicha
		tarea as� tambi�n como el mantenimiento y evoluci�n de configuraci�n existente.
		
		Habiendo dicho lo anterior, tambi�n reconocemos que Arco Iris todav�a puede no ser la herramienta m�s s�lida y madura
		que los sistemas de \emph{software} de la industria necesitan para confiar la compleja y crucial tarea de agregar auto
		reparaci�n a un sistema. Esto tiene como causas diversos motivos, algunos de los m�s importantes son:
		
		\begin{itemize}
			\item Actualmente, no todas las organizaciones que desarrollan \emph{software} poseen un modelo formal de la
			arquitectura, tal como es requerido por Rainbow o Arco Iris. Esto forma parte de una tendencia por parte de La
			industria del \emph{software} a desarrollar sistemas, a veces de envergadura y de misi�n cr�tica, de una manera
			todav�a artesanal. Esta realidad complica la adopci�n de \emph{frameworks} que centren la auto reparaci�n de un
			sistema en el modelo de su arquitectura.
			\item El trabajo de creaci�n de \emph{Gauges} y \emph{Probes} (los cuales usualmente no son reutilizables entre
			distintas aplicaciones) sigue siendo una tarea de complejidad no trivial a cargo del usuario del \emph{framework}.
			\item La informaci�n necesaria para que el \emph{framework} funcione, no obstante las mejoras introducidas en Arco
			Iris descritas en la secci�n \ref{sec:actualizacionDinamicaConfig}, sigue estando dispersa en diversos archivos de
			configuraci�n, en archivos separados de estrategias y t�cticas, en el modelo de la arquitectura, etc. Esto, sumado a
			una incompleta documentaci�n de Rainbow, configura una curva de aprendizaje del \emph{framework} un tanto pronunciada
			para un usuario nuevo. Es necesario seguir trabajando en la centralizaci�n de la configuraci�n (tal cual fue descrito
			en la secci�n \ref{sec:todaLaConfigEnUnSoloArchivo}) y tambi�n en el incremento de la documentaci�n para facilitar el
			uso, tanto de Rainbow como de Arco Iris.
		\end{itemize}

	
	\newpage
	
	\appendix

\todo{JONY: revisar formato y titulos del Apendice}

\section{Arquitectura de Znn modelada en Acme}\label{arquitecturaZNN}

{\scriptsize
\begin{verbatim}
import TargetEnvType.acme;
 
Family ZNewsFam extends EnvType with {

    Port Type HttpPortT extends ArchPortT with {

    }
    Role Type RequestorRoleT extends ArchRoleT with {

    }
    Component Type ProxyT extends ArchElementT with {

        Property deploymentLocation : string <<  default : string = "localhost"; >> ;

        Property load : float <<  default : float = 0.0; >> ;

    }
    Port Type ProxyForwardPortT extends ArchPortT with {

    }
    Component Type ServerT extends ArchElementT with {

        Property deploymentLocation : string <<  default : string = "localhost"; >> ;

        Property load : float <<  default : float = 0.0; >> ;

        Property reqServiceRate : float <<  default : float = 0.0; >> ;

        Property byteServiceRate : float <<  default : float = 0.0; >> ;

        Property fidelity : int <<  HIGH : int = 5; LOW : int = 1; default : int = 5; >> ;

        Property cost : float <<  default : float = 1.0; >> ;

        Property lastPageHit : Record [uri : string; cnt : int; kbytes : float; ];

        Property anotherConstraint : string 
            <<  default : string = "heuristic self.load <= MAX_UTIL;"; >> ;

    }
    Role Type ReceiverRoleT extends ArchRoleT with {

    }
    Connector Type ProxyConnT extends ArchConnT with {
        Role req : RequestorRoleT = new RequestorRoleT extended with {

        }
        Role rec : ReceiverRoleT = new ReceiverRoleT extended with {

        }

    }
    Component Type ClientT extends ArchElementT with {

        Property deploymentLocation : string <<  default : string = "localhost"; >> ;

        Property experRespTime : float <<  default : float = 100.0; >> ;

        Property reqRate : float <<  default : float = 0.0; >> ;
        rule primaryConstraint = invariant self.experRespTime <= MAX_RESPTIME;
        rule reverseConstraint = heuristic self.experRespTime >= MIN_RESPTIME;

    }
    Port Type HttpReqPortT extends ArchPortT with {

    }
    Connector Type HttpConnT extends ArchConnT with {

        Property bandwidth : float <<  default : float = 0.0; >> ;

        Property latency : float <<  default : float = 0.0; >> ;

        Property numReqsSuccess : int <<  default : int = 0; >> ;

        Property numReqsRedirect : int <<  default : int = 0; >> ;

        Property numReqsClientError : int <<  default : int = 0; >> ;

        Property numReqsServerError : int <<  default : int = 0; >> ;

        Property latencyRate : float;
        Role req : RequestorRoleT = new RequestorRoleT extended with {

        }
        Role rec : ReceiverRoleT = new ReceiverRoleT extended with {

        }

    }

    Property MIN_RESPTIME : float;

    Property MAX_RESPTIME : float;

    Property TOLERABLE_PERCENT_UNHAPPY : float;

    Property UNHAPPY_GRADIENT_1 : float;

    Property UNHAPPY_GRADIENT_2 : float;

    Property UNHAPPY_GRADIENT_3 : float;

    Property FRACTION_GRADIENT_1 : float;

    Property FRACTION_GRADIENT_2 : float;

    Property FRACTION_GRADIENT_3 : float;

    Property MIN_UTIL : float;

    Property MAX_UTIL : float;

    Property MAX_FIDELITY_LEVEL : int;

    Property THRESHOLD_FIDELITY : int;

    Property THRESHOLD_COST : float;

    Property SUPPORT_FRACTION_GRADIENT : boolean;
}

System ZNewsSys : ZNewsFam = {


    Property MIN_RESPTIME : float = 100.0;

    Property MAX_RESPTIME : float = 1000.0;

    Property UNHAPPY_GRADIENT_1 : float = 0.1;

    Property UNHAPPY_GRADIENT_2 : float = 0.2;

    Property UNHAPPY_GRADIENT_3 : float = 0.5;

    Property FRACTION_GRADIENT_1 : float = 0.2;

    Property FRACTION_GRADIENT_2 : float = 0.4;

    Property FRACTION_GRADIENT_3 : float = 1.0;

    Property TOLERABLE_PERCENT_UNHAPPY : float = 0.4;

    Property MIN_UTIL : float = 0.1;

    Property MAX_UTIL : float = 0.75;

    Property MAX_FIDELITY_LEVEL : int = 5;

    Property THRESHOLD_FIDELITY : int = 2;

    Property THRESHOLD_COST : float = 4.0;

    Property SUPPORT_FRACTION_GRADIENT : boolean = false;
    Component s1 : ServerT, ArchElementT = new ServerT, ArchElementT extended with {

        Property deploymentLocation = "phoenix";

        Property isArchEnabled = true;

        Property cost = 1.0;

        Property fidelity = 3;

        Property load = 0.891;
        Port http0 : HttpPortT, ArchPortT = new HttpPortT, ArchPortT extended with {

        }

    }
    Component lbproxy : ProxyT = new ProxyT extended with {

        Property deploymentLocation = "127.0.0.1";

        Property isArchEnabled = true;

        Property load = 0.01;
        Port fwd0 : ProxyForwardPortT = new ProxyForwardPortT extended with {

            Property isArchEnabled = true;

        }
        Port fwd1 : ProxyForwardPortT = new ProxyForwardPortT extended with {

        }
        Port fwd2 : ProxyForwardPortT = new ProxyForwardPortT extended with {

        }
        Port fwd3 : ProxyForwardPortT = new ProxyForwardPortT extended with {

        }
        Port http0 : HttpPortT = new HttpPortT extended with {

            Property isArchEnabled = true;

        }
        Port http1 : HttpPortT = new HttpPortT extended with {

            Property isArchEnabled = true;

        }
        Port http2 : HttpPortT = new HttpPortT extended with {

            Property isArchEnabled = true;

        }

    }
    Component s2 : ServerT, ArchElementT = new ServerT, ArchElementT extended with {

        Property deploymentLocation = "127.0.0.3";

        Property isArchEnabled = true;

        Property fidelity = 5;

        Property load = 0.99;

        Property cost = 1.0;
        Port http0 : HttpPortT, ArchPortT = new HttpPortT, ArchPortT extended with {

            Property isArchEnabled = false;

        }

    }
    Component s3 : ServerT, ArchElementT = new ServerT, ArchElementT extended with {

        Property deploymentLocation = "127.0.0.4";

        Property isArchEnabled = true;

        Property cost = 1.0;

        Property fidelity = 3;

        Property load = 0.891;
        Port http0 : HttpPortT, ArchPortT = new HttpPortT, ArchPortT extended with {

        }

    }
    Component s0 : ServerT, ArchElementT = new ServerT, ArchElementT extended with {

        Property deploymentLocation = "oracle";

        Property cost = 0.9;

        Property fidelity = 3;

        Property load = 0.594;

        Property isArchEnabled = true;
        Port http0 : HttpPortT = new HttpPortT extended with {

            Property isArchEnabled = true;

        }

    }
    Component c1 : ClientT = new ClientT extended with {

        Property deploymentLocation = "127.0.0.1";

        Property isArchEnabled = true;

        Property experRespTime = 433.36273;
        Port p0 : HttpReqPortT = new HttpReqPortT extended with {

            Property isArchEnabled = true;

        }

    }
    Component c2 : ClientT = new ClientT extended with {

        Property deploymentLocation = "127.0.0.1";

        Property isArchEnabled = true;

        Property experRespTime = 344.6827;
        Port p0 : HttpReqPortT = new HttpReqPortT extended with {

            Property isArchEnabled = true;

        }

    }
    Component c0 : ClientT = new ClientT extended with {

        Property deploymentLocation = "127.0.0.1";

        Property isArchEnabled = true;

        Property experRespTime = 414.76843;
        Port p0 : HttpReqPortT = new HttpReqPortT extended with {

            Property isArchEnabled = true;

        }

    }
    Connector conn0 : HttpConnT = new HttpConnT extended with {

        Property latencyRate = 0.0;

        Property isArchEnabled = true;
        Role req  = {

            Property isArchEnabled = true;

        }
        Role rec  = {

            Property isArchEnabled = true;

        }

    }
    Connector proxyconn0 : ProxyConnT = new ProxyConnT extended with {

        Property isArchEnabled = true;
        Role req  = {

            Property isArchEnabled = true;

        }
        Role rec  = {

            Property isArchEnabled = true;

        }

    }
    Connector proxyconn1 : ProxyConnT, ArchConnT = new ProxyConnT, ArchConnT extended with {

    }
    Connector proxyconn3 : ProxyConnT, ArchConnT = new ProxyConnT, ArchConnT extended with {

    }
    Connector proxyconn2 : ProxyConnT, ArchConnT = new ProxyConnT, ArchConnT extended with {

    }
    Connector conn : HttpConnT = new HttpConnT extended with {

        Property latencyRate = 0.0;

        Property isArchEnabled = true;
        Role req  = {

            Property isArchEnabled = true;

        }
        Role rec  = {

            Property isArchEnabled = true;

        }

    }
    Connector conn1 : HttpConnT = new HttpConnT extended with {

        Property latencyRate = 0.0;

        Property isArchEnabled = true;
        Role req  = {

            Property isArchEnabled = true;

        }
        Role rec  = {

            Property isArchEnabled = true;

        }

    }
    Attachment lbproxy.fwd0 to proxyconn0.req;
    Attachment s0.http0 to proxyconn0.rec;
    Attachment lbproxy.http0 to conn0.rec;
    Attachment c1.p0 to conn.req;
    Attachment c2.p0 to conn1.req;
    Attachment lbproxy.http2 to conn1.rec;
    Attachment lbproxy.http1 to conn.rec;
    Attachment c0.p0 to conn0.req;
    Attachment s2.http0 to proxyconn2.rec;
    Attachment lbproxy.fwd1 to proxyconn1.req;
    Attachment s1.http0 to proxyconn1.rec;
    Attachment s3.http0 to proxyconn3.rec;
    Attachment lbproxy.fwd3 to proxyconn3.req;
    Attachment lbproxy.fwd2 to proxyconn2.req;
}

\end{verbatim}
}

\newpage
\section{T�cticas de Znn}\label{tacticasZNN}

{\scriptsize
\begin{verbatim}

module newssite.tactics;

import model "ZNewsSys.acme" { ZNewsSys as M, ZNewsFam as T };
import op "znews0.operator.EffectOp" { EffectOp as S };
import op "org.sa.rainbow.stitch.lib.*";


/**
 * Enlist n free servers into service pool.
 * Utility: [v] R; [^] C; [<>] F
 */
tactic enlistServers (int n) {
  condition {
    // some client should be experiencing high response time
    exists c : T.ClientT in M.components | c.experRespTime > M.MAX_RESPTIME;
    // there should be enough available server resources
    Model.availableServices(T.ServerT) >= n;
  }
  action {
    set servers = Set.randomSubset(Model.findServices(T.ServerT), n);
    for (T.ServerT freeSvr : servers) {
      S.activateServer(freeSvr);
    }
  }
  effect {
    // response time decreasing below threshold should result
    forall c : T.ClientT in M.components | c.experRespTime <= M.MAX_RESPTIME;
  }
}

/**
 * Deactivate n servers from service pool into free pool.
 * Utility: [^] R; [v] C; [<>] F
 */
tactic dischargeServers (int n) {
  condition {
    // there should be NO client with high response time
    forall c : T.ClientT in M.components | c.experRespTime <= M.MAX_RESPTIME;
    // there should be enough servers to discharge
    Set.size({ select s : T.ServerT in M.components | s.load < M.MIN_UTIL }) >= n;
  }
  action {
    set lowUtilSvrs = { select s : T.ServerT in M.components | s.load < M.MIN_UTIL };
    set subLowUtilSvrs = Set.randomSubset(lowUtilSvrs, n);
    for (T.ServerT s : subLowUtilSvrs) {
      S.deactivateServer(s);
    }
  }
  effect {
    // still NO client with high response time
    forall c : T.ClientT in M.components | c.experRespTime <= M.MAX_RESPTIME;
  }
}

/**
 * Lowers fidelity by integral steps for percent of requests.
 * Utility: [v] R; [v] C; [v] F
 */
tactic lowerFidelity (int step, float fracReq) {
  condition {
    // some client should be experiencing high response time
    exists c : T.ClientT in M.components | c.experRespTime > M.MAX_RESPTIME;
    // exists server with fidelity to lower
    exists s : T.ServerT in M.components | s.fidelity > step;
  }
  action {
    // retrieve set of servers who still have enough fidelity grade to lower
    set servers = { select s : T.ServerT in M.components | s.fidelity > step };
    for (T.ServerT s : servers) {
      S.setFidelity(s, s.fidelity - step);
    }
  }
  effect {
    // response time decreasing below threshold should result
    forall c : T.ClientT in M.components | c.experRespTime <= M.MAX_RESPTIME;
  }
}

/**
 * Raises fidelity by integral steps for percent of requests.
 * Utility: [^] R; [^] C; [^] F
 */
tactic raiseFidelity (int step, float fracReq) {
  condition {
    // there should be NO client with high response time
    forall c : T.ClientT in M.components | c.experRespTime <= M.MAX_RESPTIME;
    // there exists some client with below low-threshold response time
    exists c : T.ClientT in M.components | c.experRespTime < M.MIN_RESPTIME;
  }
  action {
    // first find the lowest fidelity set
    set servers = { select s : 
        T.ServerT in M.components | s.fidelity <= M.MAX_FIDELITY_LEVEL - step};
   for (T.ServerT s : servers) {
      S.setFidelity(s, java.lang.Math.min(s.fidelity + step, M.MAX_FIDELITY_LEVEL));
    }
  }
  effect {
    // still NO client with high response time
    forall c : T.ClientT in M.components | c.experRespTime <= M.MAX_RESPTIME;
  }
}
\end{verbatim}
}

\newpage
\section{Representaci�n en XML del Escenario de Tiempo de Res\-pues\-ta}\label{scenarioExpRespTimeXML}

{\scriptsize
\begin{verbatim}
    <selfHealingScenario id="0" name="Client Experienced Response Time Scenario" enabled="true" priority="1">
      <concern>RESPONSE_TIME</concern>
      <stimulusSource>Any Client requesting news content</stimulusSource>
      <stimulus>GetNewsContentClientStimulus</stimulus>
      <environments>
        <defaultEnvironment></defaultEnvironment>
      </environments>
      <artifact reference="../../../artifacts/artifact"/>
      <response>Requested News Content</response>
      <responseMeasure>
        <description>Experienced response time is within threshold</description>
        <constraint class="numericBinaryRelationalConstraint" sum="false">
          <artifact reference="../../../../../artifacts/artifact"/>
          <property>experRespTime</property>
          <binaryOperator>LESS_THAN</binaryOperator>
          <constantToCompareThePropertyWith class="int">600</constantToCompareThePropertyWith>
        </constraint>
      </responseMeasure>
      <architecturalDecisions/>
      <repairStrategy></repairStrategy>
    </selfHealingScenario>
\end{verbatim}
}

\newpage
\section{Representaci�n en XML del Escenario de Costo}\label{scenarioCostXML}

{\scriptsize
\begin{verbatim}
    <selfHealingScenario id="1" name="Server Cost Scenario" enabled="true" priority="2">
      <concern>SERVER_COST</concern>
      <stimulusSource>Anyone</stimulusSource>
      <stimulus>ANY</stimulus>
      <environments>
        <defaultEnvironment reference="../../../selfHealingScenario/environments/defaultEnvironment"/>
      </environments>
      <artifact reference="../../../artifacts/artifact[2]"/>
      <response>The proper response for the request</response>
      <responseMeasure>
        <description>Active servers amount is within threshold</description>
        <constraint class="numericBinaryRelationalConstraint" sum="true">
          <artifact reference="../../../../../artifacts/artifact[2]"/>
          <property>cost</property>
          <binaryOperator>LESS_THAN_OR_EQUALS</binaryOperator>
          <constantToCompareThePropertyWith class="int">1</constantToCompareThePropertyWith>
        </constraint>
      </responseMeasure>
      <architecturalDecisions/>
      <repairStrategy></repairStrategy>
    </selfHealingScenario>
\end{verbatim}
}

	
	\newpage
	
	%	Las citas bibliogr�ficas deber�n ser adecuadas al tema y usando el siguiente formato por orden alfab�tico:
%	[AUT/ZZ] Autores, titulo, Publicaci�n, Editorial, A�o.
\def\refname{Bibliograf�a}

\begin{thebibliography}{99}
	\bibitem[GIO/82]{GIO/82} Gioan A. ``Regularized Minimization Under Weaker Hypotheses'', applied Mathematics Optimization, Springer Verlag, Volumen 8 numero1 - pag 59-68.1982.

	\bibitem[GMW99]{GMW99} Garlan, D., Monroe R. T., Wile D. \href{http://www.cs.cmu.edu/afs/cs/project/able/ftp/acme-fcbs/acme-fcbs.pdf}{``Acme: Architectural Description of Component-Based Systems''}.

\end{thebibliography}

\todo{ver svn/doc}\\
\todo{Libros de Arquitecturas (Ver biblio de IS2)}\\
\todo{Art�culos del SEI de ATAM y QAW}\\
\todo{Tesis de Owen Chen}\\
\todo{Todos los art�culos que nos pas� Garlan sobre Self Healing}\\
\todo{Tesis del flaco de la UCA}

\end{document}