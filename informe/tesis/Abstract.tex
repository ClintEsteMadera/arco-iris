\begin{abstract}
  Los sistemas auto-reparables (o aut�nomos) son aquellos que pueden adaptarse din�micamente a las condiciones del
entorno, ayudando as� a asegurar su propia estabilidad y productividad sin depender de la intervenci�n de un usuario
administrador. Existen al momento diversos enfoques en materia de auto-reparaci�n de sistemas, sin embargo, en ninguno
se consider� al usuario como un participante crucial al momento de determinar los requerimientos de auto-reparaci�n de
un sistema. El objetivo principal del presente trabajo es el de proveer una herramienta que permita superar
esa limitaci�n. Se introduce una extensi�n a \emph{Rainbow}, un \emph{framework} de auto reparaci�n basado
en arquitecturas ya existente. Dicha extensi�n, a la cual llamamos ``Arco Iris'', hace foco en el concepto de
``escenario de atributo de calidad'' como medio para que el usuario junto a otros \emph{stakeholders} del sistema puedan
configurar la auto-reparaci�n del sistema de manera colaborativa; estableciendo, entre otras cosas, prioridades entre
distintos escenarios. En este trabajo se establece el marco te�rico necesario para poder estudiar el tema, se explica en
detalle el funcionamiento (de car�cter heur�stico) de ``Arco Iris'', se presenta tambi�n una herramienta visual que
permite al usuario definir los escenarios de manera sencilla e intuitiva, se estudia el funcionamiento de la extensi�n
con algunos casos de uso representativos, se extraen conclusiones sobre los resultados obtenidos y finalmente se
establecen puntos de continuaci�n para el presente trabajo.
\end{abstract}

\def\abstractname{Abstract}
\begin{abstract}

	\todo{SANTI: Traducir Resumen a Ingl�s}

\end{abstract}
