\begin{abstract}

	La auto reparaci�n de sistemas ha experimentado considerables avances en los �ltimos a�os, sin embargo, el usuario se ha visto exclu�do de la posibilidad de participar en el proceso de detecci�n y correcci�n aut�matica de errores. El presente trabajo intentar involucrar al usuario en dicho proceso. La propuesta consiste en combinar dos proyectos existentes: \textbf{Rainbow}, un framework de auto reparaci�n basado en arquitecturas, con una t�cnica que permite especificar atributos de calidad utilizando escenarios, usualmente derivados en sesiones de \textbf{Quality Attribute Workshops}(QAWs). El usuario deber� asignar prioridades a dichos escenarios y pesos a los distintos \emph{concerns} del sistema, toda esta informaci�n ser� analizada por un algoritmo propuesto, el cual escoger� una estrategia para reparar el sistema intentando maximizar la utilidad del mismo. Tambi�n se presentar� una herramienta visual que permitir� al usuario definir los escenarios de manera muy sencilla. Si bien no es necesario un conocimiento t�cnico espec�fico para definir los escenarios, se propone que la definici�n de los mismos sea realizada por todos los \emph{stakeholders} del sistema, tal cual se plantea en los QAWs.
	
\end{abstract}

\def\abstractname{Abstract} 
\begin{abstract}

	\todo{Traducir Resumen a Ingl�s}

\end{abstract}
