\begin{abstract}
	Los sistemas auto-reparables, tambi�n llamados aut�nomos, son aquellos que pueden adaptarse din�micamente a las
	condiciones cambiantes del entorno (contexto, usuarios, hardware) y a las fallas que puedan producirse, para asegurar
	su propia estabilidad y utilidad sin depender de la intervenci�n de un usuario administrador.

	Existen hoy diversos enfoques en materia de auto-reparaci�n. Sin embargo, en ninguno de ellos se considera al usuario
	como un actor crucial en la determinaci�n de requerimientos de auto reparaci�n de un sistema. Con el objetivo de
	superar esa limitaci�n realizamos una extensi�n del framework "Rainbow", creado por investigadores de la Universidad de
	Carnegie Mellon. En este framework el sistema conoce su arquitectura a trav�s de un modelo creado en un ADL (Lenguaje
	de Descripci�n de Arquitectura) y usa ese conocimiento al decidir e implementar la adaptaci�n o reparaci�n. La
	extensi�n realizada permite a los �stakeholders� de la aplicaci�n definir cu�les son los requerimientos de atributos de
	calidad que tiene el sistema, sus prioridades relativas, �tradeoffs� y estrategias y t�cticas de reparaci�n asociadas.

	En s�ntesis, en este trabajo establecemos el marco te�rico para estudiar el tema, extendemos el framework Rainbow para
	que contemple esta nueva funcionalidad, creamos una herramienta visual para facilitarle al usuario la tarea de
	configuraci�n y mostramos c�mo la flexibilidad introducida enriquece a Rainbow y representa un avance en la idea de
	lograr sistemas que se adapten y reparen sin intervenci�n humana.
\end{abstract}

\def\abstractname{Abstract}
\begin{abstract}
	\todo{SANTI: Traducir Resumen a Ingl�s}
\end{abstract}