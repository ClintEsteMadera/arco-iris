\begin{abstract}
	Los sistemas auto-reparables, tambi�n llamados aut�nomos, son aquellos que pueden adaptarse din�micamente a las
	condiciones cambiantes del entorno (contexto, usuarios, hardware) y a las fallas que puedan producirse, para asegurar
	su propia estabilidad y utilidad sin intervenci�n humana.

	Existen hoy diversos enfoques en materia de Auto Reparaci�n. Sin embargo, en ninguno de ellos se considera al usuario
	como un actor crucial en la determinaci�n de requerimientos de auto reparaci�n de un sistema. Con el objetivo de
	superar esa limitaci�n realizamos una extensi�n del \emph{framework} ``Rainbow'', creado por investigadores de la Universidad
	de Carnegie Mellon. En este \emph{framework} el sistema conoce su arquitectura a trav�s de un modelo creado en un ADL
	(Lenguaje de Descripci�n de Arquitectura) y usa ese conocimiento al decidir e implementar la adaptaci�n o reparaci�n.
	La extensi�n realizada permite a los \emph{stakeholders} de la aplicaci�n definir cuales son los requerimientos de
	atributos de calidad que tiene el sistema y sus prioridades relativas dependientes del contexto de ejecuci�n del
	sistema; as� tambi�n como sus estrategias y t�cticas de reparaci�n asociadas.

	En s�ntesis, en este trabajo establecemos el marco te�rico para estudiar el tema, extendemos el \emph{framework}
	Rainbow para que contemple esta nueva funcionalidad, creamos una herramienta visual para facilitarle al usuario la
	tarea de configuraci�n y mostramos c�mo la flexibilidad introducida enriquece a Rainbow y representa un avance en la
	idea de lograr sistemas que se adapten y reparen sin intervenci�n humana.
\end{abstract}

\def\abstractname{Abstract}
\todo{Revisar traducci�n de la correcci�n hecha al 2do p�rrafo}
\begin{abstract}
	Self healing systems, also referred to as Autonomous Systems, are those which can adapt dynamically to changing
	environments (context, users, hardware) and faults, without human intervention, to ensure stability and utility.
	 
	Nowadays there are many different approaches to self-healing. However, none of them considers the users as a critical
	stakeholder in determining the self healing requirements of a system. With the goal of overcoming this limitation, we
	implemented an extension to the framework named ``Rainbow'', created by researchers from Carnegie Mellon University. In
	this framework the system knows its architecture through a model created in an ADL (Architecture Description Language)
	and this knowledge is used when the adaptation or repair has to be decided and implemented. Our extension allows
	stakeholders to define which are the quality attribute requirements that affect the system and their execution
	context-dependent relative priorities, as well as their associated repair strategies and tactics.
	 
	In summary, in this work we establish the theoretical basis for studying the subject, we extend the Rainbow Framework
	in order to make it contemplate these new features, we create a visual tool to make it easier for the user to configure
	these features and we show how the flexibility added makes Rainbow a more powerful framework, and therefore
	represents and advance in the path to achieving systems that can be adapted and repaired without human intervention.
\end{abstract}