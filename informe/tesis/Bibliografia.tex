%	Las citas bibliogr�ficas deber�n ser adecuadas al tema y usando el siguiente formato por orden alfab�tico:
%	[AUT/ZZ] Autores, titulo, Publicaci�n, Editorial, A�o.
\def\refname{Bibliograf�a}

\begin{thebibliography}{99}
	\bibitem{Casuscelli} {Casuscelli, Federico J. Arquitecturas de Software para Sistemas Aut�nomos. Trabajo Final del
	posgrado ``Carrera de Especializaci�n en Ingenier�a del Software'' (CEIS), 2005, Universidad Cat�lica Argentina.}

	\bibitem{GAN/03} Ganek, Alan G. y Corbi, Thomas A. The dawning of the autonomic computing era. IBM Syst. J.,
	42(1):5-18, 2003. ISSN 0018-8670.\\
	\url{http://www.cs.cmu.edu/~garlan/17811/Readings/ganek.pdf}

	\bibitem{GMW99} Garlan, D., Monroe R. T., Wile D.
	\href{http://www.cs.cmu.edu/afs/cs/project/able/ftp/acme-fcbs/acme-fcbs.pdf}{``Acme: Architectural Description of Component-Based Systems''}.

	\bibitem{IBM-AC} P. Horn, ``Autonomic Computing: IBM's Perspective on the State of Information Technology'', Octubre
	2001, IBM Corporation.

	\bibitem{Scenarios} Kazman, Rick and Abowd, Gregory and Bass, Len and Clements, Paul, Scenario-Based Analysis of
	Software Architecture, IEEE Computer Society Press, 1996, Los Alamitos, CA, USA. Disponible on-line:\\
	\url{http://eprints.kfupm.edu.sa/63611/1/63611.pdf}

	\bibitem{ABLE} Architecture Based Languages and Environments, grupo ABLE.\\
	 \url{http://www.cs.cmu.edu/~able/}

	\bibitem{AcmeStudio} Acme Studio Tool, Software Engineering Institute (SEI)\\
	 \url{http://www.cs.cmu.edu/~acme/AcmeStudio/}

	\bibitem{Eclipse} Eclipse Platform\\
	\url{http://www.eclipse.org/}

	\bibitem{TesisOwen} Shang-Wen Cheng, \href{http://owen.tofudo.com/research/phdthesis?action=AttachFile&do=get&target=Thesis-2008-0510-FinalThesis.pdf} {``Rainbow: Cost-Effective Software Ar\-chi\-tec\-tu\-re-Ba\-sed Self-Adap\-tation''}

	\bibitem{ATAM} Software Engineering Institute (SEI),
	\href{http://www.sei.cmu.edu/library/abstracts/reports/00tr004.cfm} {``ATAM: Method for Architecture Evaluation''}

	\bibitem{Observer} Descripci�n del patr�n de dise�o ``Observer'',\\
	\url{http://en.wikipedia.org/wiki/Observer_pattern}	

\end{thebibliography}

\todo{ver svn/doc}

\todo{Libros de Arquitecturas (Ver biblio de IS2)}

\todo{Art�culos del SEI de ATAM y QAW}

\todo{Todos los art�culos que nos pas� Garlan sobre Self Healing}

\todo{Software Arquitecture in Practice de Bass y Clements}