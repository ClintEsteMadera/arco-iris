%	Las citas bibliogr�ficas deber�n ser adecuadas al tema y usando el siguiente formato por orden alfab�tico:
%	[AUT/ZZ] Autores, titulo, Publicaci�n, Editorial, A�o.
\def\refname{Bibliograf�a}

\begin{thebibliography}{99}
	\bibitem{GAN/03} Ganek, Alan G. y Corbi, Thomas A. The dawning of the autonomic computing era. IBM Syst. J., 42(1):5-18, 2003. ISSN 0018-8670.\\
	\url{http://www.cs.cmu.edu/~garlan/17811/Readings/ganek.pdf}

	\bibitem[GIO/82]{GIO/82} Gioan A. ``Regularized Minimization Under Weaker Hypotheses'', applied Mathematics Optimization, Springer Verlag, Volumen 8 numero1 - pag 59-68.1982.

	\bibitem[GMW99]{GMW99} Garlan, D., Monroe R. T., Wile D. \href{http://www.cs.cmu.edu/afs/cs/project/able/ftp/acme-fcbs/acme-fcbs.pdf}{``Acme: Architectural Description of Component-Based Systems''}.

	\bibitem{Scenarios} Kazman, Rick and Abowd, Gregory and Bass, Len and Clements, Paul, Scenario-Based Analysis of Software Architecture, IEEE Computer Society Press, 1996, Los Alamitos, CA, USA. Disponible on-line:\\
	\url{http://eprints.kfupm.edu.sa/63611/1/63611.pdf}

\end{thebibliography}

\todo{ver svn/doc}\\
\todo{Libros de Arquitecturas (Ver biblio de IS2)}\\
\todo{Art�culos del SEI de ATAM y QAW}\\
\todo{Tesis de Owen Chen}\\
\todo{Todos los art�culos que nos pas� Garlan sobre Self Healing}\\
\todo{Tesis del flaco de la UCA}