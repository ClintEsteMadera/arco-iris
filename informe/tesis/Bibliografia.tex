%	Las citas bibliogr�ficas deber�n ser adecuadas al tema y usando el siguiente formato por orden alfab�tico:
%	[AUT/ZZ] Autores, titulo, Publicaci�n, Editorial, A�o.
\def\refname{Bibliograf�a}

\begin{thebibliography}{99}

	\bibitem[BAS/03]{BassClementz} {Bass L., Clements P., Kazman R. ``Software Architecture in Practice''.
	Addison-Wesley Longman Publishing Co., Inc. Boston, MA, USA 2003.}

	\bibitem[CAS/05]{Casuscelli} {Casuscelli, F. ``Arquitecturas de Software para Sistemas Aut�nomos''. Trabajo Final
	del posgrado ``Carrera de Especializaci�n en Ingenier�a del Software'' (CEIS), Universidad Cat�lica Argentina, 2005,
	pp. 16-30}

	\bibitem[GAN/03]{Dawning} {Ganek, Alan G., Corbi, Thomas A. ``The dawning of the autonomic computing era''. IBM
	Syst. J., 42(1):5-18, 2003. \url{http://www.cs.cmu.edu/~garlan/17811/Readings/ganek.pdf}}

	\bibitem[GAR/02]{ModelBasedAdaptation4SelfHealingSystems} {Garlan, D., Schmerl, B. ``Model-based adaptation for
	self-healing systems'', Proceedings of the first workshop on Self-healing systems, Charleston, South Carolina,
	USA, 18-19 Noviembre, 2002. \url{http://portal.acm.org/citation.cfm?id=582134}}

	\bibitem[GAR/00]{ACME} {Garlan, D., Monroe R. T., Wile D. ``Acme: Ar\-chi\-tec\-tu\-ral Des\-crip\-tion of Com\-po\-nent\--Based
	Sys\-tems'', Foundations of Component-Based Systems, Cambridge University Press, pp 47-68, 2000.
	\url{http://www.cs.cmu.edu/afs/cs/project/able/ftp/acme-fcbs/acme-fcbs.pdf}}

	\bibitem[HOR/01]{IBM-AC} {Horn, P. ``Autonomic Computing: IBM's Perspective on the State of Information Technology'',
	IBM Corporation, Octubre 2001. \url{http://www.research.ibm.com/autonomic/manifesto/autonomic_computing.pdf}}

	\bibitem[KAZ/96]{Scenarios} {Kazman, R., Abowd, G., Bass, L., Clements, P. ``Scenario-Based Analysis of Software
	Architecture'', IEEE Computer Society Press, Los Alamitos, CA, USA, 1996. \url{http://eprints.kfupm.edu.sa/63611/1/63611.pdf}}

	\bibitem[KAZ/00]{ATAM} {Kazman R., Klein, M., Clements P. ``ATAM: Method for Architecture Evaluation'', Software
	Engineering Institute (SEI), Agosto 2000. \url{http://www.sei.cmu.edu/library/abstracts/reports/00tr004.cfm}}

	\bibitem[ORI/99]{ArchBasedApproach} {Oriezy P. et al. ``An Architecture-Based Approach to Self-Adaptive
	Software'', IEEE Intelligent Systems, vol. 14, no. 3, 1999, pp. 54-62.
	\url{http://www.ics.uci.edu/~peymano/papers/ieee-is99.pdf}}

	\bibitem[PAN/10]{ADLsVsUML} {Pandey R. ``Architectural description languages (ADLs) vs UML: a review''.
	University Institute of Computer Science and Applications (UICSA) and R. D. University, Jabalpur (M.P.) India, 2010.}

	\bibitem[SHA/08]{TesisOwen} {Shang-Wen C. Rainbow: Cost-Effective Software Ar\-chi\-tec\-ture-Based
	Self-A\-dap\-ta\-tion, School of Computer Science, Carnegie Mellon University, Pittsburgh, PA, USA, Mayo 2008.
	\url{http://tinyurl.com/Owen-2008-0510-FinalThesis}}

	\bibitem[PAN/10]{C&C} {Clements P. et al., ``Documenting Software Architecture: Views and Beyond'', Addison-Wesley,
	2003.}

\end{thebibliography}

\todo{ver svn/doc}

\todo{Libros de Arquitecturas (Ver biblio de IS2)}

\todo{Todos los art�culos que nos pas� Garlan sobre Self Healing}