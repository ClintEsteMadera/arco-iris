\section{Introducci�n}
	\todo{JONY: mergear con todo lo que pusimos en el abstract en la seccion 2}
	\subsection{Motivaci�n para este trabajo}
		La complejidad creciente de los sistemas desaf�a de forma permanente el estado del arte de las Ciencias de la Computaci�n y la Ingenier�a del Software. La velocidad con la que se producen los cambios, la criticidad de las fallas que aparecen y la necesidad de mantener sistemas funcionando de manera continua a pesar de no pertenecer a lo que tradicionalmente se conoce como ``sistemas de misi�n cr�tica'' ha llevado a los investigadores a buscar novedosas formas de resolver estos desaf�os. Una de ellas es la tendencia hacia los sistemas aut�nomos, que recibe distintos nombres como ``Computaci�n Aut�mona'', ``Software consciente'' o ``Sistemas Auto�Reparables'' (o ``Self Healing'' en ingl�s). Existe una cantidad en aumento de especialistas en el mundo \cite{GAN/03} que creen que la necesidad de implementar este tipo de mecanismos est� dando lugar al nacimiento de una nueva era en los sistemas de software.

		La idea subyacente detr�s de estos nombres es que los sistemas incluyan mecanismos para ajustar su comportamiento a partir de fallas o necesidades cambiantes de sus usuarios y/o el entorno en el que operan. De esta forma, un sistema puede repararse u optimizarse sin intervenci�n humana. Una de las formas de implementar estos mecanismos es la llamada ``Adaptaci�n Basada en Arquitecturas''. En este tipo de soluciones, el sistema tiene un m�dulo que conoce su arquitectura, y, sobre la base de este conocimiento y el problema detectado, toma una decisi�n sobre c�mo auto-repararse.

		Si bien ya existen soluciones de este tipo, en ninguna se considera la participaci�n de los \emph{stakeholders} en el proceso de detecci�n y correcci�n autom�tica de errores. Esto nos motiv� para pensar en qu� manera se los pod�a incluir en el proceso, siempre teniendo en cuenta que su participaci�n deb�a contribuir principalmente en las definiciones de los potenciales problemas y sus posibles soluciones. Esto nos llev� a considerar el uso de los Quality Attribute Scenarios (QAS), los cuales nos permiten definir c�mo deber�a responder el sistema ante determinados est�mulos.
		
		Una vez definidos los escenarios por los \emph{stakeholders}, el paso siguiente consist�a en utilizar toda esa informaci�n en tiempo de ejecuci�n para que el sistema sea capaz de auto repararse. All� nos servimos de un \emph{framework} de auto reparaci�n existente llamado Rainbow. Rainbow propone una manera no muy din�mica y muy poco amigable de definir las situaciones en que se deber�a lanzar la auto reparaci�n, as�, al sumarle los escenarios, se logra mantener todo el potencial de Rainbow y a su vez permitir que los \emph{stakeholders} participen en el proceso de auto-reparaci�n y que tomen conocimiento del dinamismo que podr�a llegar a sufrir el sistema debido a la misma.

	\subsection{Organizaci�n del presente trabajo}
		\todo{JONY: Explicar!}