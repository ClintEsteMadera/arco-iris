\section{Introducci�n}
	\todo{Breve referencia del trabajo mencionando la parte novedosa del trabajo.}
	
	\subsection{Self Healing}

	\subsection{Descripci�n de Sistemas basados en Componentes}	
		
		Un problema fundamental de las arquitecturas de los sistemas basados en componentes ha sido encontrar la notaci�n apropiada para definir dichos sistemas.\ 
		
		Un buen lenguaje para la descripci�n de arquitecturas permite generar una documentaci�n clara sobre los componentes del sistema, que luego sirvir� como base a los desarrolladores, permitiendo a su vez razonar sobre las propiedades del sistema y automatizar su an�lisis y hasta puede llegar a utilizarse para la generaci�n autom�tica del sistema.\
		
		Una forma de describir dichas arquitecturas es mediante el modelado de objetos, si bien este m�todo ha sido ampliamente aceptado y utilizado en la industria, tiene varios inconvenientes, el m�s importante y bloqueante es que no proveen un soporte directo para describir propiedades no funcionales, esto hace dificultoso razonar sobre propiedades cr�ticas del sistema, como son la performance y confiabilidad. �sta es la raz�n principal que ha motivado el avance de los ADLs (Architecture Description Language).\cite{GMW99}\
		
		La descripci�n de arquitecturas de sistemas basada en ADLs ha avanzado much�simo en las �ltimas 2 d�cadas, al punto de que ya permiten definir una base formal para la descripci�n y el an�lisis de los mismos.\
					 
		
	\subsection{Acme}
	
	Acme es uno de los ADLs m�s reconocidos y utilizados, ha sido desarrollado en la Carnegie Mellon University, m�s precisamente por el proyecto  \href{http://www.cs.cmu.edu/~able/}{Architecture Based Languages and Environments (Able)}, liderado por el Dr. David Garlan.\
	
	Acme es un pilar fundamental dentro del proyecto Able, ya que todo el proyecto gira en torno a la arquitectura de software de los sistemas, y es Acme quien permite describir dichas arquitecturas, por lo tanto todos los restantes subproyectos dentro de Able utilizan Acme en menor o en mayor medida.
		 
	\subsection{Rainbow}
	
	\subsubsection{Introducci�n a Rainbow}
	
	\subsubsection{T�cticas y Estrategias}

	\subsubsection{ZNN: Testeando Rainbow}
	
	\subsection{Escenarios de Atributos de Calidad y QAW}
	
	\subsection{ATAM}
	
		
	
	
	