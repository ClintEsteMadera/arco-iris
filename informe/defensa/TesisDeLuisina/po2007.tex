\documentclass[table,dvipsnames]{beamer}
\usepackage[spanish]{babel}
\usepackage{beamerthemesplit}

% My Beamer options
\usetheme{Warsaw}
\setbeamercovered{transparent}
\setbeamerfont{subtitle}{size=\tiny}

\title{Coordinando experimentos}
\subtitle{Un ejercicio de coloreo de grafos}
\author{Facundo Carreiro}
\institute{Universidad de Buenos Aires}
\date{1 de Octubre de 2007}

\begin{document}

% Title Page
\begin{frame}
\titlepage
\end{frame}

% Outline
\begin{frame} 
\tableofcontents 
\end{frame}

\section{Introducci\'on}
\subsection{Enunciado}
% Enunciado
\begin{frame}
\frametitle{Enunciado}

Un conjunto de experimentos solares debe ser realizado en distintos
observatorios. Cada experimento empieza en un d\'ia dado del a\~no y
termina en otro d\'ia dado del mismo a\~no.

\vspace{20pt}
Un observatorio puede realizar
un solo experimento por vez. El problema es entonces determinar cual
es el menor n\'umero de observatorios necesarios para realizar un 
conjunto dado de experimentos.

\vspace{20pt}
Modelar el problema como un problema
de coloreo de grafos.
\end{frame}

% Enunciado (cont.)
\begin{frame}
\frametitle{Enunciado (cont.)}

Tomaremos como caso de estudio la siguiente lista de experimentos

\begin{center}
\rowcolors{1}{RoyalBlue!20}{RoyalBlue!5}
\begin{tabular}{lll}
\textbf{Experimento} & \textbf{Inicio} & \textbf{Fin} \\
\hline
A & 02/09 & 25/12 \\
B & 15/10 & 10/12 \\
C & 20/01 & 17/03 \\
D & 23/01 & 30/05 \\
E & 04/04 & 28/07 \\
F & 30/04 & 28/07 \\
G & 24/06 & 30/09 \\
\end{tabular}
\end{center}

\end{frame}

\subsection{Repaso de coloreo de grafos}
\begin{frame}
\begin{block}{Coloreo de un grafo}
El objetivo de colorear un grafo es \alert{asignar colores} (o n\'umeros naturales) a los nodos de manera tal que \alert{ning\'un par de nodos adyacentes compartan el mismo color}.
\end{block}

\vspace{10pt}
\begin{columns}
\column{5cm}
\begin{block}{N\'umero crom\'atico}
Es la menor cantidad de colores con las que se puede colorear un grafo.
\end{block}

\column{4cm}
%\includegraphics[width=4cm]{coloreado.png}
\end{columns}
\end{frame}

% Modelado
\section{Modelado}
\subsection{An\'alisis del enunciado}
\begin{frame}
\frametitle{An\'alisis del enunciado}

\uncover<2-3>{Un conjunto de experimentos solares debe ser realizado en distintos
observatorios. Cada experimento \alert{empieza}<3-> en un d\'ia dado del a\~no y
\alert{termina}<3-> en otro d\'ia dado del mismo a\~no.}

\vspace{20pt}
\uncover<4-5>{Un observatorio puede realizar
\alert{un solo experimento por vez}<5->. El problema es entonces determinar cual
es el \alert{menor n\'umero}<5-> de observatorios necesarios para realizar un 
conjunto dado de experimentos.}

\end{frame}

\subsection{Relaci\'on del mundo con nuestro modelo}
\begin{frame}
\frametitle{Relaci\'on del mundo con nuestro modelo}

Nuestros elementos
\begin{itemize}
\item Conjunto de experimentos y observatorios
\item Debemos asignarle observatorios a los experimentos
\item Bajo la condici\'on de que no est\'en en el mismo observatorio dos experimentos en simult\'aneo
\end{itemize}

\begin{columns}
\column{4cm}
\begin{block}{\Large Modelo}
\begin{itemize}
\item Colores
\item Nodos
\item Ejes
\end{itemize}
\end{block}

\column{4cm}
\begin{block}{\Large Mundo}
\begin{itemize}
\item Experimentos
\item Restricciones
\item Observatorios
\end{itemize}
\end{block}

\end{columns}

\end{frame}

\section{Resoluci\'on}
\subsection{Armando el grafo}
\begin{frame}[t]
\begin{center}

%\includegraphics[height=4cm]{modelo1.png}<1-2>
%\includegraphics[height=4cm]{modelo2.png}<3>
%\includegraphics[height=4cm]{modelo3.png}<4>
%\includegraphics[height=4cm]{modelo4.png}<5>

\vspace{20pt}
%\includegraphics[height=2cm]{gantt.png}<2->

\end{center}

\end{frame}

\subsection{Coloreando el grafo}
\begin{frame}
\begin{center}
%\includegraphics[height=7cm]{modelo4.png}<1>
%\includegraphics[height=7cm]{color1.png}<2>
%\includegraphics[height=7cm]{color2.png}<3>
%\includegraphics[height=7cm]{color3.png}<4>
%\includegraphics[height=7cm]{color4.png}<5>
\end{center}
\end{frame}

\section{}
\begin{frame}
\frametitle{>Qu\'e nos llevamos?}
Ten\'iamos
\pause
\begin{itemize}
\item Enunciado con un conjunto elementos particionado (experimentos, observatorios)
\pause
\item Asignaci\'on entre los conjuntos
\pause
\item Restricciones entre los elementos de uno de los conjuntos
\end{itemize}

\vspace{10pt}
\pause
Modelamos
\pause
\begin{itemize}
\item Los elementos como nodos y colores
\pause
\item La asignaci\'on como el coloreo
\pause
\item Las restricciones como la adyacencia de los nodos
\end{itemize}

\begin{center}

\end{center}
\end{frame}

\begin{frame}
\begin{center}
\Huge{>Preguntas?}
\end{center}
\end{frame}

\end{document}