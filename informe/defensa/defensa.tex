\documentclass[red,table,dvipsnames]{beamer}
\usepackage[spanish, activeacute]{babel}
\usepackage[ansinew]{inputenc}
% descomentar lo siguiente para una presentaci�n en formato widescreen 
%\usepackage[orientation=landscape,size=custom,width=16,height=10,scale=0.5]{beamerposter}

\usetheme{Warsaw}
\usecolortheme{rose}
\setbeamercovered{transparent}
\setbeamerfont{subtitle}{size=\tiny}
\usenavigationsymbolstemplate{}	% borrar los s�mbolos de navegaci�n de abajo a la derecha
%\beamertemplateshadingbackground{red!10}{white!1}

% beamerbasefont.sty defines the commands \Tiny and \TINY for very small font sizes. Redefining \Tiny to avoid warnings
\let\Tiny=\tiny
\setcounter{tocdepth}{1}

\title[Auto Reparaci�n de Sistemas de Software]{Hacia un modelo m�s flexible para la implementaci�n de la auto
reparaci�n de sistemas de software basada en Arquitectura}
\subtitle{Tesis de Licenciatura en Ciencias de la Computaci�n}
\author{Chiocchio, Jonathan \and Tursi, Germ�n Gabriel}
\institute{Universidad de Buenos Aires\\Facultad de Ciencias Exactas y Naturales\\Departamento de Computaci�n}
\date{10 de Junio de 2011}
%\date{\today}

\begin{document}

% Title Page
\begin{frame}[plain]
	\titlepage
\end{frame}

%Outline
\begin{frame}
	\tableofcontents
\end{frame}

\section{Conceptos Preliminares}

	\subsection{Introducci�n a Self Healing}

		\begin{frame}
			\frametitle{\insertsubsection}
		\end{frame}
				
	\subsection{Rainbow}

		\begin{frame}
			\frametitle{\insertsubsection}
			\begin{itemize}
				\item Mencionar Self Healing basado en Arquitectura
				\item Mencionar Acme
				\item Arquitectura:
					\begin{itemize}
						\item Figura 10 y contar cuento de punta a punta 
					\end{itemize}
				\item Explicar qu� es una Estrategia (obviar concepto de T�ctica)
			\end{itemize}
		\end{frame}
	
\section{Descripci�n del Problema y Propuesta}

	\subsection{Mejoras a Rainbow}
	
		\begin{frame}
			\frametitle{\insertsubsection}
			\begin{itemize}
			  \item No considera al usuario como un actor crucial en la determinaci�n de requerimientos
			  \item Poco din�mico, causas:
			  \begin{itemize}
			    \item Configuraci�n compleja
			    \item Ausencia de mecanismo de actualizaci�n en caliente
			   \end{itemize}
			  \item No se adapta al entorno de ejecuci�n
			  \item Duplicaci�n de configuraci�n de restricciones
			  \item etc.
			\end{itemize}
		\end{frame}

	\subsection{Atributos de Calidad y Concerns}

		\begin{frame}
			\frametitle{\insertsubsection}
		\end{frame}
	
	\subsection{QAW y QAS}
	
		\begin{frame}
			\frametitle{\insertsubsection}
			stakeholders participan en la conf de self healing
		\end{frame}
	
	\subsection{Arco Iris}
	
		\subsubsection{�Qu� es Arco Iris?}
		
			\begin{frame}
				\frametitle{\insertsubsubsection}
				\begin{itemize}
				  \item Permite a stakeholders definir los requerimientos de QA y sus prioridades relativas dependientes del
				  contexto de ejecuci�n
				\end{itemize}

				En qu� consiste la extensi�n a Rainbow (Rainbow + Escenarios)
			\end{frame}
	
		\subsubsection{Arco Iris UI}

			\begin{frame}
				\frametitle{\insertsubsubsection}
				(mostrar configuraci�n necesaria para utilizar Arco Iris y mencionar que es dinamica)
			\end{frame}

		\subsubsection{Implementaci�n de Arco Iris}

			\begin{frame}
				\frametitle{\insertsubsubsection}
				\begin{itemize}
					\item Mostrar arquitectura de Arco Iris (haciendo un golpe de efecto con la Arq. De Rainbow)
					\item Explicar importancia de est�mulo, entorno y cuantificaci�n de la rta. para la auto reparaci�n.
					\item Modelo extendido de QAS (explicar SelfHealingScenario: prioridades relativas, estrategias embebidas)
					\item Concepto de Utilidad del Sistema
					\item Puntuaci�n y Selecci�n de estrategias
				\end{itemize}
			\end{frame}

\section{Casos de Prueba}

	\begin{frame}
		\frametitle{\insertsection}
		Utiliza modo simulaci�n de una tesis de doctorado (znn)
	\end{frame}

	\subsection{Caso 1 - Renombrar}

		\begin{frame}
			\frametitle{\insertsubsection}
			(1 escenario - sin estrategias)
		\end{frame}

	\subsection{Caso 3 - Renombrar}
	
		\begin{frame}
			\frametitle{\insertsubsection}
			(Tradeoff entre estrategias)
		\end{frame}

	\subsection{Caso 4 - Renombrar}

		\begin{frame}
			\frametitle{\insertsubsection}
			(Tradeoff seg�n prioridades)
		\end{frame}

	\subsection{Caso 5 - Renombrar}
	
		\begin{frame}
			\frametitle{\insertsubsection}
			(opcional)
		\end{frame}

\section{Trabajo a Futuro y Conclusiones}

	\subsection{Trabajo a Futuro}

		\begin{frame}
			\frametitle{\insertsubsection}
			\begin{itemize}
				\item Ampliaci�n de la recarga din�mica de configuraci�n (ideal: nunca detener Arco Iris para cambiar la config.)
				\item Extensi�n de tipos de restricciones implementadas por defecto
				\item Atributos de Calidad y concerns configurables por el usuario (soluci�n dif�cil)
				\item Optimizaci�n en la selecci�n de la estrategia: la utilidad del sistema es algo limitado pues hace suposiciones
				y simplificaciones sobre el modelo\ldots
				\item Ausencia u obsolescencia del modelo de la arquitectura 
			\end{itemize}
		\end{frame}

	\subsection{Conclusiones}

		\begin{frame}
			\frametitle{\insertsubsection}
			\begin{itemize}
				\item Rapidez en el cambio de configuraci�n (por la UI + recarga dinamica)
				\item Informacion sobre restricciones (fuera del modelo - mas accesible p/ los stakeholders)
				\item Entorno de ejecuci�n m�s copado con respecto al est�tico de Rainbow
				\item Aplicabilidad en sistemas reales (opcional)
				\item Conclusi�n final (secci�n 7.5) 
			\end{itemize}
		\end{frame}	
	
		\begin{frame}
			\begin{center}
				\Huge{�Preguntas?}
			\end{center}
		\end{frame}

\end{document}